\section{Data type compatibility and versioning}\label{sec:dsdl_versioning}

\subsection{Rationale}

Data type definitions may evolve over time as they are refined to better address the needs of their applications.
UAVCAN defines a set of rules that allow data type designers to modify and advance their
data type definitions while ensuring backward compatibility and functional safety.

\subsection{Compatibility}

\subsubsection{Bit compatibility}

A data type or schema $A$ is bit-compatible with a data type or schema $B$ if and only if
the set of serialized representations\footnote{The serialization rules are reviewed
in detail in the section \ref{sec:dsdl_data_serialization}.}
of $A$ is a superset of the set of serialized representations of $B$.

$A$ and $B$ are said to be \emph{mutually bit-compatible} if
their sets of serialized representations are equal.

A \emph{variable-length} data type or schema is a serializable data type or schema
whose set of serialized representations contains bit sequences of different lengths.
Conversely, any data type or schema that is not variable-length is \emph{fixed-length}.

\begin{remark}[breakable]
    The following two definitions are bit-compatible:

    \begin{minted}{python}
        uint32 a
        uint32 b
    \end{minted}

    \begin{minted}{python}
        uint64 c
    \end{minted}

    Consider the following example data type definition; assume that its full data type name is
    \verb|demo.Pair|:

    \begin{minted}{python}
        # demo.Pair
        float16 first
        float16 second
    \end{minted}

    Further, let the following define a data type named \verb|demo.PairVector|:

    \begin{minted}{python}
        # demo.PairVector
        demo.Pair[3] vector
    \end{minted}

    Then the following two definitions are bit-compatible:

    \begin{minted}{python}
        demo.PairVector pair_vector
    \end{minted}

    \begin{minted}{python}
        float16 first_0     # pair_vector.vector[0].first
        float16 second_0    # pair_vector.vector[0].second
        float16 first_1     # pair_vector.vector[1].first
        float16 second_1    # pair_vector.vector[1].second
        float16 first_2     # pair_vector.vector[2].first
        float16 second_2    # pair_vector.vector[2].second
    \end{minted}

    The latter definition in the example above is a flattened unrolled form of the former definition.
    As such, in that particular example, both definitions can be used interchangeably;
    an object serialized using one definition can be deserialized using the other definition.
    However, it is also possible to construct bit-compatible definitions that are not functionally equivalent:

    \begin{minted}{python}
        float16 a
        float32 b
    \end{minted}

    \begin{minted}{python}
        float32 a
        float16 b
    \end{minted}

    Even though the above definitions are bit-compatible, one cannot be substituted with the other.
    The problem of functional equivalency is addressed by the concept of semantic compatibility,
    explored in the section \ref{sec:dsdl_semantic_compatibility}.

    Complicated scenarios are possible when a bit belonging to a primitive-typed field attribute
    is handed over to a constrained field such as an implicit array length field or an implicit union tag field.
    Some interesting examples are shown in the table \ref{table:dsdl_many_compat},
    together with a set of serialized representation patterns.
    Remember that the bits belonging to void-typed field attributes are ignored during deserialization.

    % Please do not remove the hard placement specifier [H], it is needed to keep tables ordered.
    \begin{table}[H]\caption{Complex bit compatibility examples}\label{table:dsdl_many_compat}
        \begin{tabu}{|l|X|X|X|X|X|}
            \hline
            \rowfont{\bfseries}
            &A                  &  B                &  C                &  D                &  E                 \\
            \hline

            \multirow{2}{*}{\textbf{Definition}}
            &\texttt{void1}     &\texttt{bool x}    &\texttt{void1}     &\texttt{bool x}    &\texttt{bool[<5] a} \\
            &\texttt{bool[<3] a}&\texttt{bool[<3] a}&\texttt{bool[<4] a}&\texttt{bool[<4] a}&                    \\
            \hline

            \multirow{8}{*}{%
                \begin{tabular}[x]{@{}l@{}}\textbf{Serialized}\\\textbf{representations}\end{tabular}%
            }
            &\multicolumn{2}{l|}{\texttt{000   }}   &\multicolumn{2}{l|}{\texttt{000    }}  &\texttt{000     } \\
            &\multicolumn{2}{l|}{\texttt{001a  }}   &\multicolumn{2}{l|}{\texttt{001a   }}  &\texttt{001a    } \\
            &\multicolumn{2}{l|}{\texttt{010aa }}   &\multicolumn{2}{l|}{\texttt{010aa  }}  &\texttt{010aa   } \\
            &\multicolumn{2}{l|}{\texttt{      }}   &\multicolumn{2}{l|}{\texttt{011aaa }}  &\texttt{011aaa  } \\
            &\multicolumn{2}{l|}{\texttt{100   }}   &\multicolumn{2}{l|}{\texttt{100    }}  &\texttt{100aaaa } \\
            &\multicolumn{2}{l|}{\texttt{101a  }}   &\multicolumn{2}{l|}{\texttt{101a   }}  &\texttt{        } \\
            &\multicolumn{2}{l|}{\texttt{110aa }}   &\multicolumn{2}{l|}{\texttt{110aa  }}  &\texttt{        } \\
            &\multicolumn{2}{l|}{\texttt{      }}   &\multicolumn{2}{l|}{\texttt{111aaa }}  &\texttt{        } \\
            \hline

            {\begin{tabular}[x]{@{}l@{}}\textbf{Bit-compatible}\\\textbf{with}\end{tabular}}
            &B                  & A                 & A, B, D           & A, B, C           & \emph{(none)}    \\
            \hline
        \end{tabu}
    \end{table}
\end{remark}

\subsubsection{Semantic compatibility}\label{sec:dsdl_semantic_compatibility}

A data type $A$ is semantically compatible with a data type $B$
if an application that correctly uses $A$ exhibits a functionally equivalent behavior to an application
that correctly uses $B$.
The property of semantic compatibility is commutative.

\begin{remark}[breakable]
    Despite using different binary layouts, the following two definitions are semantically compatible
    and also bit-compatible:

    \begin{minted}{python}
        uint16 FLAG_A = 1
        uint16 FLAG_B = 256
        uint16 flags
    \end{minted}

    \begin{minted}{python}
        uint8 FLAG_A = 1
        uint8 FLAG_B = 1
        uint8 flags_a
        uint8 flags_b
    \end{minted}

    Therefore, the definitions can be used interchangeably.
    It should be noted here that due to different set of field and constant attributes,
    the source code auto-generated from the provided definitions may be not drop-in replaceable,
    requiring changes in the application;
    however, source-code-level application compatibility is orthogonal to data type compatibility.
\end{remark}

\subsection{Versioning}

\subsubsection{General assumptions}

The concept of versioning applies only to composite data types.
As such, unless specifically stated otherwise, every reference to ``data type''
in this section implies a composite data type.

A data type is uniquely identified by its full name,
assuming that every root namespace is uniquely named.
There is one or more versions of every data type.

A data type definition is uniquely identified by its full name and the version number pair.
In other words, there may be multiple definitions of a data type differentiated by their version numbers.

\subsubsection{Versioning principles}

Every data type definition has a pair of version numbers ---
a major version number and a minor version number, following the principles of semantic versioning.

For the purposes of the following definitions, a \emph{release} of a data type definition stands for
the disclosure of the data type definition to its intended users or to the general public,
or for the commencement of usage of the data type definition in a production system.

In order to ensure a deterministic application behavior and ensure a robust migration path
as data type definitions evolve, UAVCAN requires that all data type definitions that share the same
full name and the same major version number must be semantically compatible with each other
and mutually bit-compatible with each other.

The versioning rules do not extend to scenarios where the name of a data type is changed,
because that would essentially construe the release of a new data type,
which relieves its designer from all compatibility requirements.
When a new data type is first released,
the version numbers of its first definition must be assigned ``1.0'' (major 1, minor 0).

In order to ensure predictability and functional safety of applications that leverage UAVCAN,
the standard requires that once a data type definition is released,
its DSDL source text, name, version numbers, fixed port ID, and other properties cannot undergo any
modifications whatsoever, with the following exceptions:
\begin{itemize}
    \item Whitespace changes of the DSDL source text are allowed,
    excepting string literals and other semantically sensitive contexts.

    \item Comment changes of the DSDL source text are allowed as long as such changes
    do not affect semantic compatibility of the definition.

    \item A deprecation marker directive (section~\ref{sec:dsdl_directives}) can be added or removed\footnote{%
    Removal is useful when a decision to deprecate a data type definition is withdrawn.}.
\end{itemize}
Addition or removal of the fixed port identifier is not permitted after a data type definition
of a particular version is released.

Therefore, substantial modifications of released data types are only possible by releasing
new definitions of the same data type.
If it is desired and possible to keep the same major version number for a new definition of the data type,
the minor version number of the new definition shall be one greater than the newest existing minor version
number before the new definition is introduced.
Otherwise, the major version number shall be incremented by one and the minor version shall be set to zero.

An exception to the above rules applies when the major version number is zero.
Data type definitions bearing the major version number of zero are not subjected to any compatibility requirements.
Released data type definitions with the major version number of zero are permitted to change in arbitrary
ways without any regard for compatibility.
It is recommended, however, to follow the principles of immutability, releasing every subsequent definition
with the minor version number one greater than the newest existing definition.

For any data type, there shall be at most one definition per version.
In other words, there must be exactly one or zero definitions
per combination of data type name and version number pair.

All data types under the same name must be also of the same kind.
In other words, if the first released definition of a data type is of the message kind,
all other versions must also be of the message kind.

\subsubsection{Port-ID assignment constraints}

The following constraints apply to fixed port ID assignments:
\begin{align*}
    \exists P(x_{a.b})                          &\rightarrow \exists P(x_{a.c})
    &\mid& b < c;\ x \in (M \cup S) \\
    \exists P(x_{a.b})                          &\rightarrow         P(x_{a.b}) =    P(x_{a.c})
    &\mid& b < c;\ x \in (M \cup S) \\
    \exists P(x_{a.b}) \land \exists P(x_{c.d}) &\rightarrow         P(x_{a.b}) \neq P(x_{c.d})
    &\mid& a \neq c;\ x \in (M \cup S) \\
    \exists P(x_{a.b}) \land \exists P(y_{c.d}) &\rightarrow         P(x_{a.b}) \neq P(y_{c.d})
    &\mid& x \neq y;\ x \in T;\ y \in T;\ T = \left\{ M, S \right\} \\
\end{align*}
where $t_{a.b}$ denotes a data type $t$ version $a.b$ ($a$ major, $b$ minor);
$P(t)$ denotes the fixed port-ID value of data type $t$ (which may or may not exist);
$M$ is the set of message types, and $S$ is the set of service types.

\subsubsection{Data type version selection}

DSDL compilers should compile every available data type version separately,
allowing the application to choose from all available major and minor version combination.

When emitting a transfer, the major version of the data type is chosen at the discretion of the application.
The minor version should be the newest available one under the chosen major version.

When receiving a transfer, the node deduces which major version of the data type to use
from its port identifier (either fixed or runtime-assigned).
The minor version should be the newest available one under the deduced major version\footnote{%
Such liberal minor version selection policy poses no compatibility risks since all definitions under the same
major version are compatible with each other.}.

It follows from the above two rules that when a node is responding to a service request,
the major data type version used for the response transfer shall be the same that is used for the request transfer.
The minor versions may differ, which is acceptable due to the major version compatibility requirements.

\begin{remark}[breakable]
    A simple usage example is provided in this intermission.

    Suppose a vendor named \emph{Sirius Cybernetics Corporation} was contracted to design a
    cryopod management data bus for a colonial spaceship \emph{Golgafrincham B-Ark}.
    Having consulted with applicable specifications and standards, an engineer came up with the following
    definition of a cryopod status message type (named \verb|sirius_cyber_corp.golgafrincham_b_ark.cryopod.Status|):

    \begin{minted}{python}
        # sirius_cyber_corp.golgafrincham_b_ark.cryopod.Status.0.1

        float16 internal_temperature    # [kelvin]
        float16 coolant_temperature     # [kelvin]

        # Status flags in the low byte
        uint16 FLAG_COOLING_SYSTEM_A_ACTIVE = 1
        uint16 FLAG_COOLING_SYSTEM_B_ACTIVE = 2
        # Error flags in the high byte
        uint16 FLAG_PSU_MALFUNCTION = 8192
        uint16 FLAG_OVERHEATING     = 16384
        uint16 FLAG_CRYOBOX_BREACH  = 32768
        # Storage for the above defined flags
        uint16 flags
    \end{minted}

    The definition has been deployed to the first prototype for initial lab tests.
    Since the definition was experimental, the major version number was set to zero, to signify the
    tentative nature of the definition.
    Suppose that upon completion of the first trials it was identified that the units must track their power consumption
    in real time, for each of the three redundant power supplies independently.
    The definition has been amended appropriately.

    It is easy to see that the amended definition shown below is neither semantically compatible nor bit-compatible
    with the original definition; however, it shares the same major version number of zero, because the backward
    compatibility rules do not apply to zero-versioned data types to allow for low-overhead experimentation
    before the system is fully deployed and fielded.

    \begin{minted}{python}
        # sirius_cyber_corp.golgafrincham_b_ark.cryopod.Status.0.2

        truncated float16 internal_temperature    # [kelvin]
        truncated float16 coolant_temperature     # [kelvin]

        saturated float32 power_consumption_0     # Power consumption by the redundant PSU 0 [watt]
        saturated float32 power_consumption_1     # likewise for PSU 1
        saturated float32 power_consumption_2     # likewise for PSU 2

        # Status flags in the low byte
        uint16 FLAG_COOLING_SYSTEM_A_ACTIVE = 1
        uint16 FLAG_COOLING_SYSTEM_B_ACTIVE = 2
        # Error flags in the high byte
        uint16 FLAG_PSU_MALFUNCTION = 8192
        uint16 FLAG_OVERHEATING     = 16384
        uint16 FLAG_CRYOBOX_BREACH  = 32768
        # Storage for the above defined flags
        uint16 flags
    \end{minted}

    The last definition was deemed sufficient and deployed to the production system
    under the version number of 1.0: \verb|sirius_cyber_corp.golgafrincham_b_ark.cryopod.Status.1.0|.

    Having collected empirical data from the fielded systems, the Sirius Cybernetics Corporation has
    identified a shortcoming in the v1.0 definition, which was corrected in an updated definition.
    Since the updated definition, which is shown below, is mutually semantically
    compatible\footnote{The topic of data serialization is explored in detail in the section
    \ref{sec:dsdl_data_serialization}.}
    with v1.0, the major version number was kept the same and the minor version number was incremented by one:

    \begin{minted}{python}
        # sirius_cyber_corp.golgafrincham_b_ark.cryopod.Status.1.1

        saturated float16 internal_temperature    # [kelvin]
        saturated float16 coolant_temperature     # [kelvin]

        saturated float32[3] power_consumption    # Power consumption by the PSU

        # Status flags
        uint8 STATUS_FLAG_COOLING_SYSTEM_A_ACTIVE = 1
        uint8 STATUS_FLAG_COOLING_SYSTEM_B_ACTIVE = 2
        uint8 status_flags

        # Error flags
        uint8 ERROR_FLAG_PSU_MALFUNCTION = 5
        uint8 ERROR_FLAG_OVERHEATING     = 6
        uint8 ERROR_FLAG_CRYOBOX_BREACH  = 7
        uint8 error_flags
    \end{minted}

    Since the definitions v1.0 and v1.1 are mutually semantically compatible,
    UAVCAN nodes using either of them can successfully interoperate on the same bus.

    Suppose further that at some point a newer version of the cryopod module was released,
    with higher precision temperature sensors.
    The definition has to be updated accordingly to use \verb|float32| for the temperature fields
    instead of \verb|float16|.
    Seeing as that change breaks the binary compatibility,
    the major version number has to be incremented by one, and the minor version number has to be reset back to zero:

    \begin{minted}{python}
        # sirius_cyber_corp.golgafrincham_b_ark.cryopod.Status.2.0

        float32 internal_temperature    # [kelvin]
        float32 coolant_temperature     # [kelvin]

        float32[3] power_consumption    # Power consumption by the PSU

        # Status flags
        uint8 STATUS_FLAG_COOLING_SYSTEM_A_ACTIVE = 1
        uint8 STATUS_FLAG_COOLING_SYSTEM_B_ACTIVE = 2
        uint8 status_flags

        # Error flags
        uint8 ERROR_FLAG_PSU_MALFUNCTION = 5
        uint8 ERROR_FLAG_OVERHEATING     = 6
        uint8 ERROR_FLAG_CRYOBOX_BREACH  = 7
        uint8 error_flags
    \end{minted}

    Now, nodes using v1.0, v1.1, and v2.0 definitions can still coexist on the same network,
    but they are not guaranteed to understand each other unless they support all of the used data type definitions.

    In practice, nodes that need to maximize their compatibility are likely to employ all existing major versions of
    each used data type.
    If there are more than one minor versions available, the highest minor version within the major version should
    be used, to take advantage of the latest changes in the data type definition.
    It is also expected that in certain scenarios some nodes may resort to publishing the same message type
    using different major versions concurrently to circumvent compatibility issues (in the
    example reviewed here that would be v1.1 and v2.0).
\end{remark}
