\section{Abstract concepts}

This section defines core concepts that are agnostic of a particular transport layer implementation.

\subsection{Transfer}

A \emph{transfer} is an act of data transmission between nodes.

\subsubsection{Broadcast and unicast transfers}

A transfer that is addressed to any interested node except the source node is a \emph{broadcast transfer}.
A transfer that is addressed to one particular node is a \emph{unicast transfer}.

In the case of broadcast transfers, the sending node makes the data widely available on the bus,
allowing any interested node to freely opt-in and process
it\footnote{The word ``broadcast'' should not lead one to believe that every node is required to
process such transfers. The opt-in logic is facilitated by automatic acceptance filtering features
implemented on the transport layer.}.
The decision of whether to process any given transfer or not is made by receiving nodes.

In the case of unicast transfers, the addressing logic is inverted:
the sending node decides which particular remote node should receive the transfer.
All other nodes remain unaffected by such transmission and take no part in the addressing process.

\subsubsection{Message and service transfers}

A \emph{message transfer} is a broadcast transfer that contains a serialized message and its
metadata\footnote{Such as the subject-ID and the source node-ID.}.

A \emph{service transfer} is a unicast transfer that contains either a service request or a service response
with related metadata.

\subsubsection{Single-frame and multi-frame transfers}

Both message and service transfers can be further distinguished between single-frame and multi-frame transfers.

A \emph{single-frame transfer} is a transfer that is entirely contained in a single transport frame.
The amount of data that can be exchanged using single-frame transfers is dependent on the transport protocol in use.

A \emph{multi-frame transfer} is a transfer that has its payload distributed over multiple transport frames.
The UAVCAN protocol stack handles transfer decomposition and reassembly automatically.

The choice between single-frame and multi-frame transfers is made by the UAVCAN protocol logic on
the transmitting node based on the amount of payload data to be transferred.
The application does not have any control over the type of transfer that will be used
except limiting the amount of payload data.
UAVCAN protocol implementations must always choose single-frame transfers if possible;
multi-frame transfers can be used only if all of the requested payload cannot be allocated in one transport frame.

\subsection{Message publication}

Message publication is the main method of communication between UAVCAN nodes.

A published message is carried by a single message transfer that contains the serialized message object.
A published message does not contain any additional fields besides those listed in the table
\ref{table:common_transfer_properties}.

In order to publish a message, the publishing node must have a node-ID that is unique within the network.
An exception applies to \emph{anonymous message publications}.

\subsubsection{Anonymous message publication}\label{sec:transport_anonymous_message_publication}

An anonymous message transfer is a transfer that can be sent from a node that does not have a node-ID.
This kind of message transfer is especially useful for facilitation of \emph{plug-and-play nodes}
(a high-level concept that is reviewed in detail in chapter \ref{sec:application_layer}).

A node that does not have a node-ID is said to be in \emph{passive mode}.
Passive nodes are unable to initiate regular data exchanges,
but they can listen to the transfers exchanged over the bus,
and they can emit anonymous message transfers.

An anonymous message has the same properties as a regular message, except for the source node-ID.

An anonymous transfer can only be a single-frame transfer. Multi-frame anonymous message transfers are not allowed.
This restriction must be kept in mind when designing message data types
intended for use with anonymous message transfers:
when used with anonymous transfers, the whole message must fit into a single transport frame;
however, the same data type can be used with multi-frame regular (non-anonymous) transfers, if desired.

Anonymous messages may require special handling logic depending on the transport layer in use.

\subsubsection{Message timing requirements}

Generally, a message transmission should be aborted if it cannot be completed in 1 second.
Applications are allowed to deviate from this recommendation,
provided that every such deviation is explicitly documented.
It is expected that high-frequency high-priority messages may opt for lower timeout values,
whereas low-priority delayable data may opt for higher timeout values to account for network congestion.

\subsection{Service invocation}

A service invocation sequence consists of two related service transfers:
\emph{service request transfer} and \emph{service response transfer}.

A service request transfer is sent from the invoking node -- \emph{client node} -- to the node
that provides the service -- \emph{server node}.
Upon handling the request, the server node responds to the client node with a service response transfer.
The client will match the response with the corresponding request by comparing the following values:
server node-ID, service-ID, and the transfer-ID.

The tables \ref{table:service_request_transfer_properties} and \ref{table:service_response_transfer_properties}
describe the properties of service request and service response transfers, respectively.

Both the client and the server must have node-ID values that are unique within the network;
service invocation is not available to passive nodes.
The client and the server must be two distinct nodes.

\begin{UAVCANSimpleTable}{Service request transfer properties}{|l X|}\label{table:service_request_transfer_properties}
    Property                        & Description \\
    Payload                         & The serialized service request object. \\
    Service-ID                      & See the table \ref{table:common_transfer_properties}. \\
    Source node-ID                  & The node-ID of the client (the invoking node). \\
    Destination node-ID             & The node-ID of the server (the invoked node). \\
    Priority                        & See the table \ref{table:common_transfer_properties}. \\
    Transfer-ID                     & An integer value that:
        \begin{enumerate}
            \item allows the server to distinguish the request from other requests from the same client;
            \item allows the client to match the response with its request.
        \end{enumerate} \\
\end{UAVCANSimpleTable}

\begin{UAVCANSimpleTable}{Service response transfer properties}{|l X|}\label{table:service_response_transfer_properties}
    Property                        & Description \\
    Payload                         & The serialized service response object. \\
    Service-ID                      & Same value as in the request transfer. \\
    Source node-ID                  & The node-ID of the server (the invoked node). \\
    Destination node-ID             & The node-ID of the client (the invoking node). \\
    Priority                        & Same value as in the request transfer. \\
    Transfer-ID                     & Same value as in the request transfer. \\
\end{UAVCANSimpleTable}

\subsubsection{Service timing requirements}

Applications are recommended to follow the service invocation timing recommendations specified below.
Applications are allowed to deviate from these recommendations,
provided that every such deviation is explicitly documented.

\begin{itemize}
    \item Service transfer transmission should be aborted if does not complete in 1 second.
    \item The client should stop waiting for a response from the server if one has not arrived within 1 second.
\end{itemize}

If the server uses a significant part of the timeout period to process the request,
the client might drop the request before receiving the response.
It is recommended to ensure that the server will be able to process any request in less than 0.5 seconds.

\subsection{Transport layer model}

See the figure \ref{fig:transport_layer_model}.

A \emph{specifier} is a collection of identifiers that together define a category of entities.
Specifiers are auxiliary ephemeral constructs which are needed only for completeness of the model
and for reasoning about the protocol.

% Please do not remove the hard placement specifier [H], it is needed to keep elements ordered.
\begin{figure}[H]
    \centering
    \resizebox{\textwidth}{!}{
        \footnotesize
        \begin{tabu}{|l l l|X[c,2] X[c] X[c]|l|}\hline\rowfont{\bfseries}
            \multicolumn{3}{|c|}{Taxonomy} & Messages & \multicolumn{2}{|c|}{Services} & Description \\\hline

            % TRANSFER PAYLOAD
            \multicolumn{3}{|c|}{Transfer payload} &
            \multicolumn{3}{c|}{\bfseries{} Serialized object} &
            The serialized instance of a specific DSDL data type. \\\hline

            % TRANSFER PRIORITY
            \multicolumn{1}{|c|}{\multirow{7}{*}{\rotatebox[origin=c]{90}{Transfer metadata}}} &
            &
            &
            \multicolumn{3}{c|}{\bfseries{} Transfer priority} &
            Defines the urgency (time sensitivity) of the transferred object.\\\cline{4-7}

            % TRANSFER ID
            \multicolumn{1}{|c|}{} &
            &
            &
            \multicolumn{3}{c|}{\bfseries{} Transfer-ID} &
            An integer that uniquely identifies a transfer within its category.\\\cline{2-7}

            % ROUTE SPECIFIER
            \multicolumn{1}{|c|}{} &
            \multicolumn{1}{c|}{\multirow{5}{*}{\rotatebox[origin=c]{90}{\shortstack{Session \\ specifier}}}} &
            \multirow{2}{*}{\shortstack{Route \\ specifier}} &
            \multicolumn{3}{c|}{\bfseries{} Source node-ID} &
            Source node-ID is not defined for anonymous message transfers. \\\cline{5-6}

            \multicolumn{1}{|c|}{} &
            \multicolumn{1}{c|}{} &
            &
            \multicolumn{1}{c|}{} &
            \multicolumn{2}{c|}{\bfseries{} Destination node-ID} &
            Destination node-ID is not defined for message transfers.\\\cline{3-7}

            % DATA SPECIFIER
            \multicolumn{1}{|c|}{} &
            \multicolumn{1}{c|}{} &
            \multirow{3}{*}{\shortstack{Data \\ specifier}} &
            \multicolumn{3}{c|}{\bfseries{} Transfer kind} &
            Message (subject) or service transfer.\\\cline{4-6}

            \multicolumn{1}{|c|}{} &
            \multicolumn{1}{c|}{} &
            &
            \multicolumn{1}{c|}{\multirow{2}{*}{\bfseries{} Subject-ID}} &
            \multicolumn{2}{c|}{\bfseries{} Service-ID} &
            Port-ID specifies how the serialized object should be processed.\\\cline{5-6}

            \multicolumn{1}{|c|}{} &
            \multicolumn{1}{c|}{} &
            &
            \multicolumn{1}{c|}{} &
            {\bfseries{} Request} &
            {\bfseries{} Response} &
            Request/response specifier applies to services only.\\\hline
        \end{tabu}
    }
    \caption{UAVCAN transport layer model}\label{fig:transport_layer_model}
\end{figure}

\subsection{Transfer priority}\label{sec:transfer_prioritization}

UAVCAN transfers are prioritized by means of the transfer priority property,
which allows at least 8 (eight) different priority levels for all types of transfers
(some transports may support more than eight priority levels).
Transfers with higher priority levels preempt transfers with lower priority levels,
delaying their transmission until there are no more higher priority transfers to exchange.

\begin{remark}[breakable]
    The priority level mnemonics and their usage recommendations are specified in the following list.
    The mapping between the mnemonics and actual numeric identifiers is transport-dependent.

    % https://forum.uavcan.org/t/transfer-priority-level-mnemonics/218/6?u=pavel.kirienko
    \begin{description}
        \item[Exceptional] -- The bus designer can ignore these messages when calculating bus load since they
        should only be sent when a total system failure has occurred.
        For example, a self-destruct message on a rocket would use this priority.
        Another analogy is an NMI on a microcontroller.

        \item[Immediate] -- Immediate is a ``high priority message'' but with additional latency constraints.
        Since exceptional messages are not considered when designing a bus, the latency of immediate messages
        can be determined by considering only immediate messages.

        \item[Fast] -- Fast and immediate are both ``high priority messages'' but with additional latency constraints.
        Since exceptional messages are not considered when designing a bus,
        the latency of fast messages can be determined by considering only immediate and fast messages.

        \item[High] -- High priority messages are more important than nominal messages but have looser
        latency requirements than fast messages. This priority is used so that,
        in the presence of rogue nominal messages, important commands can be received.
        For example, one might envision a failure mode where a temperature sensor starts to
        load a vehicle bus with nominal messages.
        The vehicle remains operational (for a time) because the controller is exchanging fast and
        immediate messages with sensors and actuators.
        A system safety monitor is able to detect the distressed bus and command the vehicle to a
        safe state by sending high priority messages to the controller.

        \item[Nominal] -- This is what all messages should use by default.
        Specifically the heartbeat messages should use this priority.

        \item[Low] -- Low priority messages are expected to be sent on a bus under all conditions but cannot
        prevent the delivery of nominal messages.
        They are allowed to be delayed but latency should be constrained by the bus designer.

        \item[Slow] -- Slow messages are low priority messages that have no time sensitivity at all.
        The bus designer need only ensure that, for all possible system states,
        these messages will eventually be sent.

        \item[Optional] -- These messages might never be sent (theoretically) for some possible system states.
        The system must tolerate never exchanging optional messages in every possible state.
        The bus designer can ignore these messages when calculating bus load.
        This should be the priority used for diagnostic or debug messages that are not required on an
        operational system.
    \end{description}
\end{remark}

\subsection{Transfer descriptor}\label{sec:transfer_descriptor}
% TODO: SESSION SPECIFIER!
Transfer emission and reception processes rely on the concept of \emph{transfer descriptor}.

A transfer descriptor is a set of properties that identify a particular set of transfers that originate
from the same source node, share the same port-ID, same kind (message or service), and are addressed to the same
destination node (the latter applies only to unicast transfers).

The properties that constitute a transfer descriptor are listed below:

\begin{itemize}
    \item Transfer kind (message or service).
    \item Port-ID (subject-ID for message transfers, service-ID for service transfers).
    \item Source node-ID.
    \item Destination node-ID (only for service transfers).
\end{itemize}

For convenience, two derived definitions are introduced.
Their objective is to simplify the description of transfer reception and emission logic that appears later in this
specification.
\begin{description}
    \item[Emitted transfer descriptor] -- a transfer descriptor where the source node-ID equals the local node's ID.
    \item[Received transfer descriptor] -- a transfer descriptor where the destination node-ID equals
    the local node's ID (for service transfers) or is not defined (for message transfers).
\end{description}

\subsubsection{Hard real-time considerations}

Hard real-time applications require a predictable and deterministic data processing time.
The concept of transfer descriptor plays an important role in communication;
hence, its contribution to the worst case data processing load should be carefully analyzed.

\begin{remark}
    From the above definition of transfer descriptor it is easy to derive that for any
    message subject-ID or any service subject-ID the maximum number of transfer descriptors
    that can be observed by the local node will never exceed the number of nodes on the bus minus
    one\footnote{The local node cannot exchange data with itself, hence minus one.}.
    If the number of nodes on the bus cannot be known in advance, it can be considered to equal the maximum
    number of nodes permitted by the used transport layer\footnote{E.g., 128 nodes for the CAN bus transport.}.

    The total number of distinct transfer descriptors that can be observed by a node on any valid UAVCAN bus
    is a product of the number of distinct port-ID values utilized by the node and the number of other nodes on the bus.

    The transport emission and reception logic defined later in this specification relies on data structures
    indexed by transfer descriptor values.
    Elements of such structures can be easily accessed via constant-complexity static look-up tables
    because the worst case number of elements is always statically known.
\end{remark}

\subsection{Transfer-ID computation}\label{sec:transfer_id}

The \emph{transfer-ID} is a small unsigned integer value in the range from 0 to 31, inclusive,
that is provided for every transfer.
This value is crucial for many aspects of UAVCAN communication\footnote{One might be tempted to use the transfer-ID
value for temporal synchronization of parallel message streams originating from the same node,
where messages bearing the same transfer-ID value are supposed to correspond to the same moment in time.
Such use is strongly discouraged because it is impossible to detect if one node is more than
32 messages behind another.
If temporal synchronization is necessary, explicit time stamping should be used instead.};
specifically:
\begin{description}
    \item[Message sequence monitoring] - the continuously increasing transfer-ID allows receiving nodes to
    detect lost messages and detect when a message stream from any remote node is interrupted.

    \item[Service response matching] - when a server responds to a request, it uses the same transfer-ID for the
    response as in the request,
    allowing any node to emit concurrent requests to the same server while being able to
    match each response with the corresponding request.

    \item[Transport frame deduplication] - for single-frame transfers,
    the transfer-ID allows receiving nodes to work around the transport
    frame duplication problem\footnote{This is a well-known issue that can be observed with certain
    transports such as CAN bus -- a frame that appears valid to the receiver may under certain
    (rare) conditions appear invalid to the transmitter, triggering the latter to retransmit the frame,
    in which case it will be duplicated on the side of the receiver.
    Sequence counting mechanisms such as the transfer-ID or the toggle bit (both of which are used in UAVCAN)
    allow applications to circumvent this problem.} (multi-frame transfers combat the frame duplication
    problem using the toggle bit, which is introduced later).

    \item[Multi-frame transfer reassembly] - more info is provided in section \ref{sec:transfer_reception}.

    \item[Automatic management of redundant interfaces] - the transfer-ID parameter allows the UAVCAN protocol
    stack to perform automatic switchover to a back-up interface shall the primary interface fail.
    The switchover logic can be completely transparent to the application, joining several independent
    redundant physical transports into a highly reliable single virtual communication channel.
\end{description}

For message transfers and service request transfers the ID value should be computed as described below.
For service response transfers this value must be directly copied from the corresponding service request transfer.

Every node that is interested in emitting transfers must maintain a mapping
(or a similar functionally equivalent static structure\footnote{For example, simple static variables.})
from emitted transfer descriptors (section \ref{sec:transfer_descriptor}) to transfer-ID counters.
This mapping is referred to as the \emph{emitted transfer-ID map}.

Whenever a node needs to emit a transfer, it will query its transfer-ID map for the appropriate transfer descriptor.
If the map does not contain such entry, a new entry will be created with the transfer-ID counter initialized to zero.
The node will use the current value of the transfer-ID from the map for the transfer,
and then the value stored in the map will be incremented by one.
When the stored transfer-ID exceeds its maximum value, it will roll over to zero.

It is expected that some nodes will need to emit certain transfers aperiodically or on an ad-hoc basis,
thereby creating unused entries in the emitted transfer-ID map.
If such aperiodic or ad-hoc transfers are of interest,
the worst case number of unused entries can be determined statically as a function of the number of
port identifiers used and the number of addressed nodes on the bus (the latter applies to services only).
Nodes are not allowed to remove any entries from the transfer-ID map as long as they are running.

\subsection{Single frame transfers}

If the size of the entire transfer payload does not exceed the space available for payload in a single transport frame,
the whole transfer will be contained in one transport frame.
Such transfer is called a \emph{single-frame transfer}.

Single frame transfers are more efficient than multi-frame transfers in terms of throughput, latency,
and data overhead.

\subsection{Multi-frame transfers}\label{sec:transport_multi_frame_transfers}

\emph{Multi-frame transfers} are used when the size of the transfer payload exceeds the space available
for payload in a single transport frame.

Two new concepts are introduced in the context of multi-frame transfers, both of which are reviewed below in detail:
\begin{samepage}
\begin{itemize}
    \item Transfer CRC\footnote{CRC stands for ``cyclic redundancy check'', an error-detecting code
    added to data transmissions to reduce the likelihood of undetected data corruption.}.
    \item Toggle bit.
\end{itemize}
\end{samepage}

In order to emit a multi-frame transfer, the node must first compute the CRC for the entirety of the transfer payload.
The node appends the resulting CRC value at the end of the transfer payload in the big-endian byte order,
and then emits the resulting byte set in chunks as an ordered sequence of transport frames,
where the first transport frame contains the beginning of the payload bytes,
and the last transport frame contains the last bytes of the payload (possibly none) plus the transfer CRC.

The data field of all transport frames of a multi-frame transfer, except the last one, should be fully utilized.
Applications are allowed to limit the maximum amount of data transferred per transport frame in order to
improve the preemption granularity, thus reducing the worst case latency of higher priority
transfers\footnote{For example, some CAN FD applications may choose to restrict the maximum payload size to 32 bytes
rather than the protocol limit of 64 bytes, as that provides more opportunities for higher-priority frames to
take over the bus. The trade-off is that smaller frames lead to higher transfer fragmentation, increase the bus load,
and increase the overall average latency.}.
Receiving nodes must be prepared to reconstruct multi-frame transfers that utilize the
available payload space partially.

All frames of a multi-frame transfer should be pushed to the transmission queue at once,
in the proper order from the first frame to the last frame.
Explicit gap time between transport frames belonging to the same transfer should not be introduced;
rather, implementations always should strive to minimize it.
Re-ordering of frames belonging to the same multi-frame transfer is prohibited.

\subsubsection{Transfer CRC}\label{sec:transfer_crc}

Transfer CRC allows receiving nodes to ensure that a received multi-frame transfer has been reassembled correctly.

It should be understood that the transfer CRC is not intended for bit-level data integrity checks,
as that must be managed by the transport layer implementation on a per-frame
basis\footnote{Bit-level errors at the transport frame level may compromise the error-detecting
properties of the transfer CRC.}.
As such, the transfer CRC allows receiving nodes to ensure that all of the frames of a multi-frame
transfer were received, all of the received frames were reassembled in the correct order,
and that all of the received frames belong to the same multi-frame transfer.

The transfer CRC is computed over the entire payload of the transfer.
Certain transport implementations\footnote{Such as CAN FD.} may require a short sequence of padding bytes
to be added at the end of the transfer payload due to the low granularity of the frame payload length property;
in that case, the padding bytes must be included in the CRC computation as well,
as if they were part of the useful payload.

\subsection{Redundant interface support}

In configurations with redundant bus interfaces,
nodes are required to submit every outgoing transfer to the transmission queues of
all available redundant interfaces simultaneously.
It is recognized that perfectly simultaneous transmission may not be possible due to different
utilization rates of the redundant interfaces and different phasing of their traffic;
however, that is not an issue for UAVCAN.
If perfectly simultaneous frame submission is not possible, interfaces with lower numerical index
should be handled in the first order.

An exception to the above rule applies if the payload of the transfer depends on some properties
of the interface through which the transfer is emitted.
An example of such a special case is the time synchronization algorithm leveraged by UAVCAN
(documented in chapter \ref{sec:application_layer} of the specification).

Redundant interfaces are used for increased fault tolerance, not for load sharing reasons.
Whenever a node is connected to an interface the likelihood of the interface failing is increased.
This suggests that backup interfaces may only interconnect with mission-critical equipment,
unless a homogeneous network architecture is desired\footnote{Heterogeneous transport configuration
complicates the analysis of the network, which might make it impractical in safety-critical deployments.
In that case, a simpler configuration where each available redundant bus is connected to every node may be
preferred.}.
See section \ref{sec:phy_non_uniform_transport_redundancy}.

\subsection{Payload truncation}

The deterministic nature of UAVCAN in general and DSDL in particular allows implementations to statically determine the
maximum amount of memory that is required to contain a data object of a particular type.
Consequently, an implementation that is interested in receiving data objects of a particular type\footnote{%
    Messages, service requests, or service responses.
}
can statically determine the maximum length of the transfer payload.

Implementations shall be able to handle incoming transfers containing a larger amount of payload data than expected.
In the event of such extra payload being received, a compliant implementation should silently\footnote{%
    Such occurrence is not indicative of a problem so it shall not be reported as such.
}
discard the excessive (unexpected) data at the end of the received payload.
The transfer CRC, if applicable, shall be validated regardless of the presence of the extra payload in the transfer.

The requirement to silently discard the excessive payload data at the end of the transfer is motivated by
the necessity to allow extensibility of data type definitions, as described in chapter \ref{sec:dsdl}.
Additionally, excessive payload data may contain padding bytes if required by the particular transport layer.

\begin{figure}[H]
    $$
    \raisebox{1em}{\footnotesize{\text{first byte}}}
    \overbrace{%
        \underbrace{\huge{%
            \blacksquare\blacksquare\blacksquare\blacksquare\blacksquare\blacksquare%
            \blacksquare\blacksquare\blacksquare\blacksquare\blacksquare\blacksquare%
        }}_{\substack{\text{Expected, accepted} \\ \text{payload}}}%
        \underbrace{\huge{%
            \boxtimes\boxtimes\boxtimes\boxtimes\boxtimes\boxtimes\boxtimes\boxtimes%
        }}_{\substack{\text{Excessive, discarded} \\ \text{payload}}}%
    }^{\substack{%
        \text{Transfer CRC is validated} \\
        \text{for the entire transfer payload} \\
        \text{before the truncation}}
    }
    \raisebox{1em}{\footnotesize{\text{last byte}}}
    $$
    \caption{Transfer payload truncation.\label{fig:transport_payload_truncation}}
\end{figure}

\begin{remark}
    Let node $A$ publish an object of the following type over the subject $X$:

    \begin{minted}{python}
        float32 parameter
        float32 variance
    \end{minted}

    Let node $B$ subscribe to the subject $X$ expecting an object of the following type:

    \begin{minted}{python}
        float32 parameter
    \end{minted}

    The payload truncation requirement guarantees that the two nodes will be able to interoperate despite
    relying on incompatible data type definitions.
    Under this example, the duty of ensuring the semantic compatibility lies on the system integrator.
\end{remark}

Implementations shall not enforce the minimum payload size on received transfers.
In other words, implementations shall not discard a transfer even if it is determined that it contains less payload
data than expected.
Serialization validity constraints are to be enforced by the object deserialization routines
instead of the transport layer.
