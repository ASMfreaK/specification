\section{Grammar}\label{sec:dsdl_grammar}

This section contains the formal definition of the DSDL grammar.
Its notation is introduced beforehand.
The meaning of each element of the grammar and their semantics will be explained in the following sections.

\subsection{Notation}

The notation used in the following definition is a variant of the extended Backus--Naur
form\footnote{This notation is a subset of the notation defined in a Python parsing library titled Parsimonious.
Parsimonious is an MIT-licensed software product authored by Erik Rose;
its sources are available at \url{https://github.com/erikrose/parsimonious}.}.
The rule definition patterns are specified in the table \ref{table:dsdl_grammar_definition_notation}.
The text of the formal definition contains comments which begin with an octothorp and last until the end of the line.

\begin{UAVCANSimpleTable}{Notation used in the formal grammar definition}{|l X|}
    \label{table:dsdl_grammar_definition_notation}
    Pattern & Description \\

    \texttt{"korovan"} &
    Denotes a terminal string of ASCII characters.
    The string is case-sensitive. \\

    \emph{(space)} &
    Concatenation.
    E.g., \texttt{korovan paukan excavator} matches a sequence where the specified tokens
    appear in the defined order. \\

    \texttt{korovan / paukan / excavator} &
    Alternatives.
    The leftmost matching alternative is accepted. \\

    \texttt{korovan?} &
    Optional greedy match. \\

    \texttt{paukan*} &
    Zero or more expressions, greedy match. \\

    \texttt{excavator+} &
    One or more expressions, greedy match. \\

    \texttt{\textasciitilde{}r"regex-pattern"} &
    An IEEE POSIX Extended Regular Expression pattern defined between the double quotes.
    The expression operates on the ASCII character set and is always case-sensitive.
    ASCII escape sequences ``\texttt{\textbackslash{}r}'', ``\texttt{\textbackslash{}n}'', and
    ``\texttt{\textbackslash{}t}'' are used to denote ASCII carriage return (code 13),
    line feed (code 10), and tabulation (code 9) characters, respectively. \\

    \texttt{\textasciitilde{}r'regex-pattern'} &
    As above, with single quotes instead of double quotes. \\

    \texttt{(korovan paukan)} &
    Parentheses are used for grouping. \\
\end{UAVCANSimpleTable}

\subsection{Definition}

At the top level, a DSDL definition file is an ordered collection of statements;
the order is determined by the relative placement of statements inside the DSDL source file:
statements located closer the beginning of the file precede those that are located closer to the end of the file.

From the top level down to the expression rule, the grammar is a valid regular grammar,
meaning that it can be parsed using standard regular expressions.

The grammar definition provided here assumes lexerless parsing;
that is, it applies directly to the unprocessed source text of the definition.

All characters used in the definition belong to the ASCII character set.

\clearpage\inputminted[fontsize=\scriptsize]{python}{dsdl/grammar.parsimonious}

\subsection{Expressions}

Symbols representing operators belong to the ASCII (basic Latin) character set.

Operators of the same precedence level are evaluated from left to right.

The attribute reference operator is a special case: it is defined for an instance of any type
on its left side and an attribute identifier on its right side.
The concept of ``attribute identifier'' is not otherwise manifested in the type system.
The attribute reference operator is not explicitly documented for any data type;
instead, the documentation specifies the set of available attributes for instances of said type,
if there are any.

\begin{UAVCANSimpleTable}{Unary operators}{|l l X|}
    Symbol                             & Precedence & Description \\
    \texttt{\textbf{+}}                         & 3 & Unary plus \\
    \texttt{\textbf{-}} (hyphen-minus)          & 3 & Unary minus \\
    \texttt{\textbf{!}}                         & 8 & Logical not \\
\end{UAVCANSimpleTable}

\begin{UAVCANSimpleTable}{Binary operators}{|l l X|}
    Symbol                                          & Precedence & Description \\
    \texttt{\textbf{.}} (full stop)                          & 1 & Attribute reference
                                                                   (parent object on the left side,
                                                                   attribute identifier on the right side) \\

    \texttt{\textbf{**}}                                     & 2 & Exponentiation
                                                                   (base on the left side, power on the right side) \\

    \texttt{\textbf{*}}                                      & 4 & Multiplication \\
    \texttt{\textbf{/}}                                      & 4 & Division \\
    \texttt{\textbf{\%}}                                     & 4 & Modulo \\

    \texttt{\textbf{+}}                                      & 5 & Addition \\
    \texttt{\textbf{-}} (hyphen-minus)                       & 5 & Subtraction \\

    \texttt{\textbf{|}} (vertical line)                      & 6 & Bitwise or \\
    \texttt{\textbf{\textasciicircum{}}} (circumflex accent) & 6 & Bitwise xor \\
    \texttt{\textbf{\&}}                                     & 6 & Bitwise and \\

    \texttt{\textbf{==}} (dual equals sign)                  & 7 & Equality \\
    \texttt{\textbf{!=}}                                     & 7 & Inequality \\
    \texttt{\textbf{<=}}                                     & 7 & Less or equal \\
    \texttt{\textbf{>=}}                                     & 7 & Greater or equal \\
    \texttt{\textbf{<}}                                      & 7 & Less \\
    \texttt{\textbf{>}}                                      & 7 & Greater \\

    \texttt{\textbf{||}} (dual vertical line)                & 9 & Logical or \\
    \texttt{\textbf{\&\&}}                                   & 9 & Logical and \\
\end{UAVCANSimpleTable}

\subsection{Literals}

Upon its evaluation, a literal yields an object of a particular type depending on the syntax of the literal,
as specified in this section.

\subsubsection{Boolean literals}

A boolean literal is denoted by the keyword ``\verb|true|'' or ``\verb|false|''
represented by an instance of primitive type ``\verb|bool|'' (section \ref{sec:dsdl_primitive_types})
with an appropriate value.

\subsubsection{Numeric literals}

Integer and real literals are represented as instances of type ``\verb|rational|'' (section \ref{sec:dsdl_rational}).

The digit separator character ``\verb|_|'' (underscore) does not affect the interpretation of numeric literals.

The significand of a real literal is formed by the integer part, the optional decimal point,
and the optional fraction part;
either the integer part or the fraction part (not both) can be omitted.
The exponent is optionally specified after the letter ``\verb|e|'' or ``\verb|E|'';
it indicates the power of 10 by which the significand is to be scaled.
Either the decimal point or the letter ``\verb|e|''/``\verb|E|'' with the following exponent
(not both) can be omitted from a real literal.

\begin{remark}
    An integer literal \verb|0x123| is represented internally as $\frac{291}{1}$.

    A real literal \verb|.3141592653589793e+1| is represented internally as
    $\frac{3141592653589793}{1000000000000000}$.
\end{remark}

\subsubsection{String literals}

String literals are represented as instances of type ``\verb|string|'' (section \ref{sec:dsdl_string}).

A string literal is allowed to contain an arbitrary sequence of Unicode characters,
excepting escape sequences defined in the table \ref{table:dsdl_string_literal_escape}
which must follow one of the specified therein forms.
An escape sequence begins with the ASCII backslash character ``\verb|\|''.

\begin{UAVCANSimpleTable}{String literal escape sequences}{|l X|}
    Sequence & Interpretation
    \label{table:dsdl_string_literal_escape} \\

    \texttt{\textbackslash{}\textbackslash{}}   & Backslash, ASCII code 92. Same as the escape character. \\
    \texttt{\textbackslash{}r}                  & Carriage return, ASCII code 13.               \\
    \texttt{\textbackslash{}n}                  & Line feed, ASCII code 10.                     \\
    \texttt{\textbackslash{}t}                  & Horizontal tabulation, ASCII code 9.          \\

    \texttt{\textbackslash{}\textquotesingle{}} &
    Apostrophe (single quote), ASCII code 39. Regardless of the type of quotes around the literal. \\

    \texttt{\textbackslash{}\textquotedbl{}}    &
    Quotation mark (double quote), ASCII code 34. Regardless of the type of quotes around the literal. \\

    \texttt{\textbackslash{}u????} &
    Unicode symbol with the code point specified by a four-digit hexadecimal number.
    The placeholder ``\texttt{?}'' represents a hexadecimal character \texttt{[0-9a-fA-F]}. \\

    \texttt{\textbackslash{}U????????} &
    Like above, the code point is specified by an eight-digit hexadecimal number. \\

\end{UAVCANSimpleTable}

\begin{remark}
    \begin{minted}{python}
        @assert "oh,\u0020hi\U0000000aMark" == 'oh, hi\nMark'
    \end{minted}
\end{remark}

\subsubsection{Set literals}

Set literals are represented as instances of type ``\verb|set|'' (section \ref{sec:dsdl_set})
parametrized by the type of the contained elements which is determined automatically.

A set literal declaration must specify at least one element,
which is used to determine the element type of the set.

The elements of a set literal are defined as DSDL expressions which are evaluated before a set is constructed
from the corresponding literal.

\begin{remark}
    \begin{minted}{python}
        @assert {"cells", 'interlinked'} == {"inter" + "linked", 'cells'}
    \end{minted}
\end{remark}

\subsection{Reserved identifiers}\label{sec:dsdl_reserved_identifiers}

DSDL identifiers and data type name components that match any of the
case-insensitive patterns specified in the table \ref{table:dsdl_reserved_word_patterns}
cannot be used to name new entities.
The semantics of such identifiers is predefined by the DSDL specification,
and as such, they cannot be used for other purposes.
Some of the reserved identifiers do not have any functions associated with them
in this version of the DSDL specification, but this may change in the future.

\begin{UAVCANSimpleTable}{Reserved identifier patterns (POSIX ERE notation, ASCII character set, case-insensitive)}%
    {|l l X|}%
    \label{table:dsdl_reserved_word_patterns}%
    POSIX ERE ASCII pattern                            & Example            & Special meaning \\
    \texttt{truncated}                                 &                    & Cast mode specifier \\
    \texttt{saturated}                                 &                    & Cast mode specifier \\
    \texttt{true}                                      &                    & Boolean literal \\
    \texttt{false}                                     &                    & Boolean literal \\
    \texttt{bool}                                      &                    & Primitive type category \\
    \texttt{u?int\textbackslash{}d*}                   & \texttt{uint8}     & Primitive type category \\
    \texttt{float\textbackslash{}d*}                   & \texttt{float}     & Primitive type category \\
    \texttt{u?q\textbackslash{}d+\_\textbackslash{}d+} & \texttt{q16\_8}    & Primitive type category (future) \\
    \texttt{void\textbackslash{}d*}                    & \texttt{void}      & Void type category \\
    \texttt{optional}                                  &                    & Reserved for future use \\
    \texttt{aligned}                                   &                    & Reserved for future use \\
    \texttt{const}                                     &                    & Reserved for future use \\
    \texttt{struct}                                    &                    & Reserved for future use \\
    \texttt{super}                                     &                    & Reserved for future use \\
    \texttt{template}                                  &                    & Reserved for future use \\
    \texttt{enum}                                      &                    & Reserved for future use \\
    \texttt{self}                                      &                    & Reserved for future use \\
    \texttt{and}                                       &                    & Reserved for future use \\
    \texttt{or}                                        &                    & Reserved for future use \\
    \texttt{not}                                       &                    & Reserved for future use \\
    \texttt{auto}                                      &                    & Reserved for future use \\
    \texttt{type}                                      &                    & Reserved for future use \\
    \texttt{con}                                       &                    & Compatibility with Microsoft Windows \\
    \texttt{prn}                                       &                    & Compatibility with Microsoft Windows \\
    \texttt{aux}                                       &                    & Compatibility with Microsoft Windows \\
    \texttt{nul}                                       &                    & Compatibility with Microsoft Windows \\
    \texttt{com\textbackslash{}d}                      & \texttt{com1}      & Compatibility with Microsoft Windows \\
    \texttt{lpt\textbackslash{}d}                      & \texttt{lpt9}      & Compatibility with Microsoft Windows \\
    \texttt{\_.*\_}                                    & \texttt{\_offset\_}& Special-purpose intrinsic entities \\
\end{UAVCANSimpleTable}
