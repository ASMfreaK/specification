\section{Data serialization}\label{sec:dsdl_data_serialization}

\newcommand{\hugett}[1]{\texttt{\huge{#1}}}

\subsection{General principles}

\subsubsection{Design goals}

The main design principle behind the serialized representations described in this section is
the maximization of compatibility with native representations used by currently existing and
likely future computer microarchitectures.
The goal is to ensure that the serialized representations defined by DSDL match internal data representations of
modern computers, so that, ideally, a typical system will not have to perform any data conversion whatsoever while
exchanging data over a UAVCAN network.

The implicit truncation and implicit zero extension rules introduced in this section are designed to
facilitate structural subtyping and to enable extensibility of data types while retaining backward compatibility.
This is a conscious trade-off between runtime type checking and long-term stability guarantees.
This model assumes that data type compatibility is determined statically and is not, normally, enforced at runtime.

\subsubsection{Bit and byte ordering}

The smallest atomic data entity is a bit.
Eight bits form one byte;
within the byte, the bits are ordered so that the least significant bit is considered first (0-th index),
and the most significant bit is considered last (7-th index).

Numeric values consisting of multiple bytes are arranged so that the least significant byte is encoded first;
such format is also known as little-endian.

\begin{figure}[H]
    $$
    \overset{\text{bit index}}{%
        \underbrace{%
            \overset{\text{M}}{\overset{7}{\hugett{0}}}
            \overset{6}{\hugett{1}}
            \overset{5}{\hugett{0}}
            \overset{4}{\hugett{1}}
            \overset{3}{\hugett{0}}
            \overset{2}{\hugett{1}}
            \overset{1}{\hugett{0}}
            \overset{\text{L}}{\overset{0}{\hugett{1}}}
        }_\text{least significant byte}%
    }
    \hugett{\ldots}
    \overset{\text{bit index}}{%
        \underbrace{%
            \overset{\text{M}}{\overset{7}{\hugett{0}}}
            \overset{6}{\hugett{1}}
            \overset{5}{\hugett{0}}
            \overset{4}{\hugett{1}}
            \overset{3}{\hugett{0}}
            \overset{2}{\hugett{1}}
            \overset{1}{\hugett{0}}
            \overset{\text{L}}{\overset{0}{\hugett{1}}}
        }_\text{most significant byte}%
    }
    $$
    \caption{Bit and byte ordering\label{fig:dsdl_serialization_bit_ordering}}
\end{figure}

\subsubsection{Implicit truncation of excessive data}\label{sec:dsdl_serialization_implicit_truncation}

When a serialized representation is deserialized, implementations shall ignore
any excessive (unused) data or padding bits remaining upon deserialization\footnote{%
    The presence of unused data should not be considered an error.
}.

As a consequence of the above requirement the transport layer can introduce
additional zero padding bits at the end of a serialized representation
to satisfy data size granularity constraints.
Non-zero padding bits are not allowed\footnote{%
    Because padding bits may be misinterpreted as part of the serialized representation.
}.

\begin{remark}
    Because of implicit truncation a serialized representation constructed from an instance of type $B$ can be
    deserialized into an instance of type $A$ as long as $B$ is a structural subtype of $A$.

    Let $x$ be an instance of data type $B$, which is defined as follows:

    \begin{minted}{python}
        float32 parameter
        float32 variance
    \end{minted}

    Let $A$ be a structural supertype of $B$, being defined as follows:

    \begin{minted}{python}
        float32 parameter
    \end{minted}

    Then the serialized representation of $x$ can be deserialized into an instance of $A$.
    The topic of data type compatibility is explored in detail in section~\ref{sec:dsdl_versioning}.
\end{remark}

\subsubsection{Implicit zero extension of missing data}\label{sec:dsdl_serialization_implicit_zero_extension}

For the purposes of deserialization routines,
the serialized representation of any instance of a data type shall \emph{implicitly} end with an
infinite sequence of bits with a value of zero (0).\footnote{%
    This can be implemented by checking for out-of-bounds access during deserialization and returning zeros
    if an out-of-bounds access is detected. This is where the name ``implicit zero extension rule'' is derived
    from.
}.

Despite this rule, implementations are not allowed to intentionally truncate trailing zeros
upon construction of a serialized representation of an object\footnote{%
    Intentional truncation is prohibited because a future revision of the specification may remove the implicit zero
    extension rule.
    If intentional truncation were allowed, removal of this rule would break backward compatibility.
}.

\begin{remark}
    The implicit zero extension rule enables extension of data types by introducing additional fields
    without breaking backward compatibility with existing deployments.
    The topic of data type compatibility is explored in detail in section~\ref{sec:dsdl_versioning}.

    The following example assumes that the reader is familiar with the variable-length array serialization rules,
    explained in section~\ref{sec:dsdl_serialized_variable_length_array}.

    Let the data type $A$ be defined as follows:

    \begin{minted}{python}
        uint8 scalar
    \end{minted}

    Let $x$ be an instance of $A$, where the value of \verb|scalar| is 4.
    Let the data type $B$ be defined as follows:

    \begin{minted}{python}
        uint8[<256] array
    \end{minted}

    Then the serialized representation of $x$ can be deserialized into an instance of $B$ where the field
    \verb|array| contains a sequence of four zeros: $0, 0, 0, 0$.
\end{remark}

\subsubsection{Error handling}

Correct UAVCAN types shall have no serialization error states.

A deserialization process may encounter a serialized representation that does not belong to the
set of serialized representations of the data type at hand.
In such case, the invalid serialized representation shall be discarded and the implementation
shall explicitly report its inability to complete the deserialization process for the given input.
Correct UAVCAN types shall have no other deserialization error states.

\subsection{Void types}\label{sec:dsdl_serialized_void}

The serialized representation of a void-typed field attribute is constructed as a sequence of zero bits.
The length of the sequence equals the numeric suffix of the type name.

When a void-typed field attribute is decoded, the values of respective bits are ignored;
in other words, any bit sequence of correct length is a valid serialized representation
of a void-typed field attribute.
This behavior facilitates usage of void fields as placeholders for non-void fields
introduced in newer versions of the data type (section~\ref{sec:dsdl_versioning}).

\begin{remark}
    The following data type will be serialized as a sequence of three zero bits $000_2$:
    \begin{minted}{python}
        void3
    \end{minted}
    The following bit sequences are valid serialized representations of the type:
    $000_2$,
    $001_2$,
    $010_2$,
    $011_2$,
    $100_2$,
    $101_2$,
    $110_2$,
    $111_2$.

    Shall the padding field be replaced with a non-void-typed field in a future version of the data type,
    nodes utilizing the newer definition may be able to retain compatibility with nodes using older types,
    since the specification guarantees that padding fields are always initialized with zeros:

    \begin{minted}{python}
        # Version 1.1
        float64 a
        void64
    \end{minted}

    \begin{minted}{python}
        # Version 1.2
        float64 a
        float32 b  # Messages v1.1 will be interpreted such that b = 0.0
        void32
    \end{minted}
\end{remark}

\subsection{Primitive types}

\subsubsection{General principles}

Implementations where native data formats are incompatible with those adopted by UAVCAN shall perform
conversions between the native formats and the corresponding UAVCAN formats during
serialization and deserialization.
Implementations shall avoid or minimize information loss and/or distortion caused by such conversions.

Serialized representations of instances of the primitive type category that are longer than one byte (8 bits)
are constructed as follows.
First, only the least significant bytes that contain the used bits of the value are preserved;
the rest are discarded following the lossy assignment policy selected by the specified cast mode.
Then the bytes are arranged in the least-significant-byte-first order\footnote{Also known as ``little endian''.}.
If the bit width of the value is not an integer multiple of eight (8) then the next value in the type will begin
starting with the next bit in the current byte. If there are no further values then the remaining bits
shall be zero (0).

\begin{remark}
    The value $1110\,1101\,1010_2$ (3802 in base-10) of type \verb|uint12| is encoded as follows.
    The bit sequence is shown in the base-2 system, where bytes (octets) are comma-separated:
    $$
        \overset{\text{byte 0}}{%
            \underbrace{%
                \overset{7}{\hugett{1}}
                \overset{6}{\hugett{1}}
                \overset{5}{\hugett{0}}
                \overset{4}{\hugett{1}}
                \overset{3}{\hugett{1}}
                \overset{2}{\hugett{0}}
                \overset{1}{\hugett{1}}
                \overset{0}{\hugett{0}}
            }_{\substack{\text{Least significant 8} \\ \text{bits of }3802_{10}}}%
        }%
        \hugett{,}%
        \overset{\text{byte 1}}{%
            \underbrace{
                \overset{7}{\hugett{?}}
                \overset{6}{\hugett{?}}
                \overset{5}{\hugett{?}}
                \overset{4}{\hugett{?}}
            }_{\substack{\text{Next object} \\ \text{or zero} \\ \text{padding bits}}}%
            \underbrace{
                \overset{3}{\hugett{1}}
                \overset{2}{\hugett{1}}
                \overset{1}{\hugett{1}}
                \overset{0}{\hugett{0}}
            }_{\substack{\text{Most} \\ \text{significant} \\ \text{4 bits of} \\ \text{3802}_{10}}}%
        }
    $$
\end{remark}

\subsubsection{Boolean types}\label{sec:dsdl_serialized_bool}

The serialized representation of a value of type \verb|bool| is a single bit.
If the value represents falsity, the value of the bit is zero (0); otherwise, the value of the bit is one (1).

\subsubsection{Unsigned integer types}\label{sec:dsdl_serialized_unsigned_integer}

The serialized representation of an unsigned integer value of length $n$ bits
(which is reflected in the numerical suffix of the data type name)
is constructed as if the number were to be written in base-2 numerical system
with leading zeros preserved so that the total number of binary digits would equal $n$.

\begin{remark}
    The serialized representation of integer 42 of type \verb|uint7| is $0101010_2$.
\end{remark}

\subsubsection{Signed integer types}

The serialized representation of a non-negative value of a signed integer type is constructed as described
in section~\ref{sec:dsdl_serialized_unsigned_integer}.

The serialized representation of a negative value of a signed integer type is computed by
applying the following transformation:
$$2^n + x$$
where $n$ is the bit length of the serialized representation
(which is reflected in the numerical suffix of the data type name)
and $x$ is the value whose serialized representation is being constructed.
The result of the transformation is a positive number,
whose serialized representation is then constructed as described in section~\ref{sec:dsdl_serialized_unsigned_integer}.

The representation described here is widely known as \emph{two's complement}.

\begin{remark}
    The serialized representation of integer -42 of type \verb|int7| is $1010110_2$.
\end{remark}

\subsubsection{Floating point types}

The serialized representation of floating point types follows the IEEE 754 series of standards as follows:

\begin{itemize}
    \item \verb|float16| --- IEEE 754 binary16;
    \item \verb|float32| --- IEEE 754 binary32;
    \item \verb|float64| --- IEEE 754 binary64.
\end{itemize}

Implementations that model real numbers using any method other than IEEE 754 shall be able to model
positive infinity, negative infinity, signaling NaN\footnote{%
    Per the IEEE 754 standard, NaN stands for
    ``not-a-number'' -- a set of special bit patterns that represent lack of a meaningful value.
}, and quiet NaN.

\subsection{Array types}

\subsubsection{Fixed-length array types}

Serialized representations of a fixed-length array of $n$ elements of type $T$ and
a sequence of $n$ field attributes of type $T$ are equivalent.

\begin{remark}
    Serialized representations of the following two data type definitions are equivalent:

    \begin{minted}{python}
        AnyType[3] array
    \end{minted}

    \begin{minted}{python}
        AnyType item_0
        AnyType item_1
        AnyType item_2
    \end{minted}
\end{remark}

\subsubsection{Variable-length array types}\label{sec:dsdl_serialized_variable_length_array}

A serialized representation of a variable-length array consists of two segments:
the implicit length field followed by the array elements.

The implicit length field is of an unsigned integer type.
The serialized representation of the implicit length field
is injected in the beginning of the serialized representation of its array.
The bit length of the unsigned integer value is first determined as follows:

$$b=\lceil{}\log_2 (c + 1)\rceil{}$$

where $c$ is the capacity (i.e., the maximum number of elements) of the variable-length array and
$b$ is the minimum number of bits needed to encode $c$ as an unsigned integer. An additional transformation
of $b$ ensures byte alignment of this implicit field when serialized\footnote{Future updates to the specification
may allow this second step to be modified but the default action will always be to byte-align the implicit
length field.}:

$$2^{\lceil{}\log_2 (\text{max}(8, b))\rceil{}}$$

The number of elements $n$ contained in the variable-length array is encoded
in the serialized representation of the implicit length field
as described in section~\ref{sec:dsdl_serialized_unsigned_integer}.
By definition, $n \leq c$; therefore, bit sequences where the implicit length field contains values
greater than $c$ do not belong to the set of serialized representations of the array.

The rest of the serialized representation is constructed as if the variable-length array was
a fixed-length array of $n$ elements\footnote{%
    Observe that the implicit array length field, per its definition,
    is guaranteed to never break the alignment of the following array elements.
}.

\begin{remark}
    Data type authors must take into account that variable-length arrays with a capacity of $\leq{}255$ elements will
    consume an additional 8 bits of the serialized representation
    (where a capacity of $\leq 65535$ will consume 16 bits and so on).
    For example:

    \begin{minted}{python}
        uint8 first
        uint8[<=6] second              # The implicit length field is 8 bits wide
        @assert _offset_.max / 8 <= 7  # This would fail.
    \end{minted}

    In the above example the author attempted to fit the message into a single Classic CAN frame but
    did not account for the implicit length field. The correct version would be:

    \begin{minted}{python}
        uint8 first
        uint8[<=5] second              # The implicit length field is 8 bits wide
        @assert _offset_.max / 8 <= 7  # This would pass.
    \end{minted}

    If the array contained three elements, the resulting set of its serialized representations would
    be equivalent to that of the following definition:

    \begin{minted}{python}
        uint8 first
        uint8 implicit_length_field  # Set to 3, because the array contains three elements
        uint8 item_0
        uint8 item_1
        uint8 item_2
    \end{minted}
\end{remark}

\subsection{Composite types}\label{sec:dsdl_serialization_composite}

\subsubsection{Final structure}

A serialized representation of an object of a final composite type that is not a tagged union
is a sequence of serialized representations of its field attribute values joined into a bit sequence,
separated by padding if such is necessary to satisfy the alignment requirements.
The ordering of the serialized representations of the field attribute values follows the order
of field attribute declaration.

\begin{remark}
    Consider the following definition,
    where the fields are assigned runtime values shown in the comments:

    \begin{minted}{python}
        #                          decimal           bit sequence   comment
        truncated uint12 first   # +48858     1011_1110_1101_1010   overflow, MSB truncated
        saturated  int3  second  #     -1                     111   two's complement
        saturated  int4  third   #     -5                    1011   two's complement
        saturated  int2  fourth  #     -1                      11   two's complement
        truncated uint4  fifth   #   +136               1000_1000   overflow, MSB truncated
        @final
    \end{minted}

    It can be seen that the bit layout is rather complicated because the field boundaries do not align with byte
    boundaries, which makes it a good case study.
    The resulting serialized byte sequence is shown below in the base-2 system:
    $$
        \underbrace{%
            \overbrace{%
                \underset{7}{\overset{7}{\hugett{1}}}%
                \underset{6}{\overset{6}{\hugett{1}}}%
                \underset{5}{\overset{5}{\hugett{0}}}%
                \underset{4}{\overset{4}{\hugett{1}}}%
                \underset{3}{\overset{3}{\hugett{1}}}%
                \underset{2}{\overset{2}{\hugett{0}}}%
                \underset{1}{\overset{1}{\hugett{1}}}%
                \underset{0}{\overset{0}{\hugett{0}}}%
            }^{\texttt{first}}%
        }_{\texttt{byte 0}}%
        \hugett{,}%
        \underbrace{%
            \overbrace{%
                \underset{7}{\overset{0}{\hugett{1}}}%
            }^{\texttt{third}}%
            \overbrace{%
                \underset{6}{\overset{2}{\hugett{1}}}%
                \underset{5}{\overset{1}{\hugett{1}}}%
                \underset{4}{\overset{0}{\hugett{1}}}%
            }^{\texttt{second}}%
            \overbrace{%
                \underset{3}{\overset{11}{\hugett{1}}}%
                \underset{2}{\overset{10}{\hugett{1}}}%
                \underset{1}{\overset{9}{\hugett{1}}}%
                \underset{0}{\overset{8}{\hugett{0}}}%
            }^{\texttt{first}}%
        }_{\texttt{byte 1}}%
        \hugett{,}%
        \underbrace{%
            \overbrace{%
                \underset{7}{\overset{2}{\hugett{0}}}%
                \underset{6}{\overset{1}{\hugett{0}}}%
                \underset{5}{\overset{0}{\hugett{0}}}%
            }^{\texttt{fifth}}%
            \overbrace{%
                \underset{4}{\overset{1}{\hugett{1}}}%
                \underset{3}{\overset{0}{\hugett{1}}}%
            }^{\texttt{fourth}}%
            \overbrace{%
                \underset{2}{\overset{3}{\hugett{1}}}%
                \underset{1}{\overset{2}{\hugett{0}}}%
                \underset{0}{\overset{1}{\hugett{1}}}%
            }^{\texttt{third}}%
        }_{\texttt{byte 2}}%
        \hugett{,}%
        \underbrace{%
            \overbrace{%
                \underset{7}{\overset{?}{\hugett{?}}}%
                \underset{6}{\overset{?}{\hugett{?}}}%
                \underset{5}{\overset{?}{\hugett{?}}}%
                \underset{4}{\overset{?}{\hugett{?}}}%
                \underset{3}{\overset{?}{\hugett{?}}}%
                \underset{2}{\overset{?}{\hugett{?}}}%
                \underset{1}{\overset{?}{\hugett{?}}}%
            }^{\substack{\text{Next object or} \\ \text{zero padding bits}}}
            \overbrace{%
                \underset{0}{\overset{3}{\hugett{1}}}%
            }^{\texttt{fifth}}%
        }_{\texttt{byte 3}}%
    $$

    Note that some of the complexity of the above illustration stems from the modern convention of representing
    numbers with the most significant components on the left moving to the least significant component of the
    number of the right. If you were to reverse this convention the bit sequences for each type in the composite
    would seem to be continuous as they crossed byte boundaries. Using this reversed representation, however, is
    not recommended because the convention is deeply ingrained in most readers, tools, and technologies.
\end{remark}

\subsubsection{Final tagged union}

Similar to variable-length arrays, a serialized representation of a final tagged union consists of two segments:
the implicit \emph{union tag} value followed by the selected field attribute value.

The implicit union tag is an unsigned integer value whose serialized representation
is implicitly injected in the beginning of the serialized representation of its tagged union.
The bit length of the implicit union tag is determined as follows:
$$b=\lceil{}\log_2 n\rceil{}$$
where $n$ is the number of field attributes in the union, $n \geq 2$ and $b$ is the minimum number of bits needed
to encode $n$ as an unsigned integer. An additional transformation of $b$ ensures byte alignment of this implicit
field when serialized\footnote{Future updates to the specification may allow this second step to be modified but
the default action will always be to byte-align the implicit length field.}:

$$2^{\lceil{}\log_2 (\text{max}(8, b))\rceil{}}$$

Each of the tagged union field attributes is assigned an index according to the order of their definition;
the order follows that of the DSDL statements (see section~\ref{sec:dsdl_grammar} on statement ordering).
The first defined field attribute is assigned the index 0 (zero),
the index of each following field attribute is incremented by one.

The index of the field attribute whose value is currently held by the tagged union is encoded
in the serialized representation of the implicit union tag as described in section
\ref{sec:dsdl_serialized_unsigned_integer}.
By definition, $i < n$, where $i$ is the index of the current field attribute;
therefore, bit sequences where the implicit union tag field contains values
that are greater than or equal $n$ do not belong to the set of serialized representations of the tagged union.

The serialized representation of the implicit union tag is immediately followed by
the serialized representation of the currently selected field attribute value\footnote{%
    Observe that the implicit union tag field, per its definition,
    is guaranteed to never break the alignment of the following field.
}.

\begin{remark}
    Consider the following example:

    \begin{minted}{python}
        @final
        @union           # In this case, the implicit union tag is one byte wide
        uint16 FOO = 42  # A regular constant attribute
        uint16 a         # Field index 0
        uint8 b          # Field index 1
        uint32 BAR = 42  # Another regular constant attribute
        float64 c        # Field index 2
    \end{minted}

    In order to encode the field \verb|b|, the implicit union tag shall be assigned the value 1.
    The following type will have an identical layout:

    \begin{minted}{python}
        @final
        uint8 implicit_union_tag  # Set to 1
        uint8 b                   # The actual value
    \end{minted}

    Suppose that the value of \verb|b| is 7.
    The resulting serialized representation is shown below in the base-2 system:
    $$%
    \overset{\text{byte 0}}{%
        \underbrace{\hugett{00000001}}_{\substack{\text{union} \\ \text{tag}}}%
    }%
    \hugett{,}%
    \overset{\text{byte 1}}{%
        \underbrace{\hugett{00000111}}_{\text{field }\texttt{b}}%
    }
    $$

\end{remark}

\begin{remark}
    Let the following data type be defined under the short name \verb|Empty| and version 1.0:

    \begin{minted}{python}
        # Empty. The only valid serialized representation is an empty bit sequence.
        @final
    \end{minted}

    Consider the following union:

    \begin{minted}{python}
        @final
        @union
        Empty.1.0 none
        AnyType.1.0 some
    \end{minted}

    The set of serialized representations of the union given above is equivalent to
    that of the following variable-length array:

    \begin{minted}{python}
        @final
        AnyType.1.0[<=1] maybe_some
    \end{minted}
\end{remark}

\subsubsection{Delimited types}\label{sec:dsdl_serialization_composite_non_final}

Objects of delimited (non-final) composite types that are nested inside other objects\footnote{%
    Of any type, not necessarily composite; e.g., arrays.
}
are serialized into opaque containers that consist of two parts:
the fixed-length \emph{delimiter header},
immediately followed by the serialized representation of the object as if it was of a final type.

Objects of delimited composite types that are \emph{not} nested inside other objects (i.e., top-level objects)
are serialized as if they were of a final type (without the delimiter header).

\begin{remark}
    Top-level objects do not require the delimiter header because the change in their length does not necessarily
    affect the backward compatibility thanks to the implicit truncation rule
    (section \ref{sec:dsdl_serialization_implicit_truncation}) and the implicit zero extension rule
    (section \ref{sec:dsdl_serialization_implicit_zero_extension}).
    The delimiter header, therefore, logically belongs to the container object rather than the contained one.
\end{remark}

The delimiter header is an implicit field of type \verb|uint32| that encodes the length of the following
serialized representation in bytes\footnote{%
    Remember that by virtue of the padding requirement (section \ref{sec:dsdl_composite_alignment_cumulative_bls}),
    the length of the serialized representation of a composite type is always an integer number of bytes.
}.
During deserialization, if the length of the serialized representation reported by its delimiter header
does not match the expectation of the deserializer,
the implicit truncation (section \ref{sec:dsdl_serialization_implicit_truncation})
and the implicit zero extension (section \ref{sec:dsdl_serialization_implicit_zero_extension})
rules apply.

It is allowed for a final composite type to nest delimited composite types, and vice versa.
No special rules apply in such cases.

\begin{remark}
    The resulting serialized representation of a delimited composite is identical to \verb|uint8[<2**32]|.
    The implicit array length prefix here is like the delimiter header,
    and the array content is the serialized representation of the composite as if it was final.

    The following illustrates why this is necessary for robust extensibility.
    Suppose that some composite $C$ contains two fields whose types are $A$ and $B$.
    The fields of $A$ are $a_0,\ a_1$;
    likewise, $B$ contains $b_0,\ b_1$.

    Suppose that $C^\prime$ is modified such that $A^\prime$ contains an extra field $a_2$.
    If $A$ (and $A^\prime$) were final, this would result in the breakage of compatibility between $C$ and $C^\prime$
    as illustrated in figure \ref{fig:dsdl_final_non_extensibility} because the positions of the fields of $B$
    (which is final) would be shifted by the size of $a_2$.

    The use of opaque containers allows the implicit truncation and the implicit zero extension rules to apply
    at any level of nesting, enabling agents expecting $C$ to truncate $a_2$ away,
    and enabling agents expecting $C^\prime$ to zero-extend $a_2$
    if it is not present, as shown in figure \ref{fig:dsdl_non_final_extensibility},
    where $H_A$ is the delimiter header of $A$.
    Observe that it is irrelevant whether $C$ (same as $C^\prime$) is final or not.

    \begin{figure}[H]
        \centering
        \begin{tabular}{r c c c c c}
            \cline{2-5}
            $C$ &
            \multicolumn{1}{|c|}{$a_0$} & \multicolumn{1}{c|}{$a_1$}
            &\multicolumn{1}{c|}{$b_0$} & \multicolumn{1}{c|}{$b_1$} &
            \\\cline{2-5}
            & $\checkmark$ & $\checkmark$ & $\times$ & $\times$ & $\times$ \\
            \cline{2-6}
            $C^\prime$ &
            \multicolumn{1}{|c|}{$a_0$} & \multicolumn{1}{c|}{$a_1$} & \multicolumn{1}{c|}{$a_2$}
            &\multicolumn{1}{c|}{$b_0$} & \multicolumn{1}{c|}{$b_1$}
            \\\cline{2-6}
        \end{tabular}
        \caption{Non-extensibility of final types}
        \label{fig:dsdl_final_non_extensibility}
    \end{figure}

    \begin{figure}[H]
        \centering
        \begin{tabular}{r c c c c c c}
            \cline{2-7}
            $C$ &
            \multicolumn{1}{|c|}{$H_A$} & \multicolumn{1}{c|}{$a_0$} & \multicolumn{1}{c|}{$a_1$}
            &\multicolumn{1}{c|}{\footnotesize{$\ldots$}}
            &\multicolumn{1}{c|}{$b_0$} & \multicolumn{1}{c|}{$b_1$}
            \\\cline{2-7}
            & $\checkmark$ & $\checkmark$ & $\checkmark$ & $\checkmark$ & $\checkmark$ & $\checkmark$ \\
            \cline{2-7}
            $C^\prime$ &
            \multicolumn{1}{|c|}{$H_A$} & \multicolumn{1}{c|}{$a_0$} & \multicolumn{1}{c|}{$a_1$} &
            \multicolumn{1}{c|}{$a_2$}
            &\multicolumn{1}{c|}{$b_0$} & \multicolumn{1}{c|}{$b_1$}
            \\\cline{2-7}
        \end{tabular}
        \caption{Extensibility of delimited types with the help of the delimiter header}
        \label{fig:dsdl_non_final_extensibility}
    \end{figure}

    The design decision to make the delimiter header of a fixed width may not be obvious so it's worth explaining.
    There are two alternatives: making it variable-length and making the length a function of the extent
    (section \ref{sec:dsdl_composite_extent_and_finality}).
    The first option does not align with the rest of the specification because DSDL does not make use of
    variable-length integers (unlike some other formats, like Google Protobuf, for example),
    and because a variable-length length prefix would have somewhat complicated the bit length set computation.
    The second option would make nested hierarchies (composites that nest other composites) possibly highly fragile
    because the change of the extent of a deeply nested type may inadvertently move the delimiter header of an
    outer type into a different length category, which would be disastrous for compatibility and hard to spot.
    There is an in-depth discussion of this issue (and other related matters) on the forum.

    The fixed-length delimiter header may be considered large,
    but delimited types tend to also be complex, which makes the overhead comparatively insignificant,
    whereas final types that tend to be compact and overhead-sensitive do not contain the delimiter header.
\end{remark}
