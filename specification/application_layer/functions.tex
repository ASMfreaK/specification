\section{Application-level functions}\label{sec:application_level_functions}

This section documents the high-level functionality defined by UAVCAN.
The common high-level functions defined by the specification span across different application domains.
All of the functions defined in this section are optional,
except for the node heartbeat feature (section \ref{sec:application_node_heartbeat}),
which is mandatory for all UAVCAN nodes.

The detailed specifications for each function are provided in the DSDL comments of the data type definitions
they are built upon, whereas this section serves as a high-level overview or an index.

\subsection{Node initialization}

UAVCAN does not require that nodes undergo any specific initialization upon connection to the bus ---
a node is free to begin functioning immediately once it is powered up.
The operating mode of the node (such as initialization, normal operation, maintenance, and so on)
is to be reflected via the mandatory heartbeat message described in section \ref{sec:application_node_heartbeat}.

\subsection{Node heartbeat}\label{sec:application_node_heartbeat}

Every UAVCAN node must report its status and presence by publishing messages of type
\DSDLReference{uavcan.node.Heartbeat} at a fixed rate specified in the message definition.

This is the only high-level protocol function that UAVCAN nodes are required to support.
All other data types and application-level functions are optional.

\DSDL{uavcan.node.Heartbeat}

\subsection{Generic node information}

The service \DSDLReference{uavcan.node.GetInfo} can be used to obtain generic information about the node,
such as the structured name of the node (which includes the name of its vendor),
a 128-bit globally unique identifier, the version information about its hardware and software,
version of the UAVCAN specification implemented on the node, and the optional certificate of authenticity.

While the service is, strictly speaking, optional, omitting its support is highly discouraged,
since it is instrumental for network discovery, firmware update, and various maintenance and diagnostic needs.

\DSDL{uavcan.node.GetInfo}

\subsection{Bus data flow monitoring}

The combination of the following three services defined in the namespace \DSDLReference{uavcan.node.port}
(see table \ref{table:dsdl:uavcan.node.port}) enables a highly capable tool of network inspection and monitoring:
\begin{itemize}
    \item \DSDLReference{uavcan.node.port.List} --- designed for obtaining the full set of subjects and services
    implemented by the server node.

    \item \DSDLReference{uavcan.node.port.GetInfo} --- returns the static (unchanging or infrequently changing)
    information about the selected subject or service.

    \item \DSDLReference{uavcan.node.port.GetStatistics} --- returns the transfer event counters of
    the selected subject or service.
\end{itemize}

The first service \verb|List| allows the caller to construct a list of all subjects and services used by the
server node (i.e., the node that the request was sent to).
The second service \verb|GetInfo| allows the caller to map each subject or service to a particular data type,
and understand the role of the server node in relation to said subject or service
(publisher, subscriber, or server).

By comparing the data obtained with the help of these two services from each node on the bus,
the caller can reconstruct the data exchange graph for the entire bus,
thus enabling advanced network monitoring and diagnostics
(assuming that every node on the bus supports the services in question; they are not mandatory).

The last service \verb|GetStatistics| can be used to sample the number of transfers and errors observed
on the specified port.
When invoked periodically, this service allows the caller to observe the real time intensity of data exchange
for each port independently.
In combination with the data exchange graph reconstruction described earlier,
this service allows the caller to build a sophisticated real-time view of the bus.

\DSDL{uavcan.node.port.* --index-only}

\subsection{Network-wide time synchronization}

UAVCAN provides a simple and robust method of time synchronization%
\footnote{The ability to accurately synchronize time between nodes is instrumental for building distributed
real-time data processing systems such as various robotic applications, autopilots, autonomous driving solutions,
and so on.} that is built upon the work
``Implementing a Distributed High-Resolution Real-Time Clock using the CAN-Bus''
published by M.~Gergeleit and H.~Streich%
\footnote{Proceedings of the 1st international CAN-Conference 94, Mainz,
13.-14. Sep. 1994, CAN in Automation e.V., Erlangen.}.
The detailed specification of the time synchronization algorithm is provided in the documentation
for the message type \DSDLReference{uavcan.time.Synchronization}.

An important service type \DSDLReference{uavcan.time.GetSynchronizationMasterInfo}
is designed to provide nodes with information about the currently used time system
and related data such as the number of leap seconds.

Redundant time synchronization masters are supported for the benefit of high-reliability applications.

\begin{remark}
      Time synchronization with explicit sensor feed timestamping should be preferred over inferior alternatives
      such as sensor lag reporting that are sometimes found in simpler systems because such alternatives
      are difficult to scale and they do not account for the delays introduced by communication interfaces.

      It is the duty of every node that publishes timestamped data to account for its own internal delays;
      for example, if a data latency of a local sensor is known,
      it needs to be accounted for in the reported timestamp value.
\end{remark}

\DSDL{uavcan.time.* --index-only}

\subsection{Primitive types and physical quantities}

The namespaces \DSDLReference{uavcan.si} and \DSDLReference{uavcan.primitive}
included in the standard data type set are designed to provide a very generic and flexible,
albeit neither bandwidth-efficient nor low-latency, method of real-time data exchange.

Generally, applications where the bus bandwidth and latency are important should minimize their reliance
on these generic data types and favor more specialized types instead that are custom-designed for their
particular use cases; e.g., vendor-specific types or application-specific types, either
designed in-house, published by third parties\footnote{As long as the license permits.}, or supplied by
vendors of COTS equipment used in the application.

Vendors of COTS equipment should always ensure that at least some minimal functionality is available
via these generic types without reliance on their vendor-specific types (if there are any).
This is important for reusability, because it is expected that some of the systems where such COTS nodes are
to be integrated into may not be able to easily support vendor-specific types.

\subsubsection{SI namespace}

The \verb|si| namespace is named after the International System of Units (SI).
The namespace contains a collection of scalar and vector value types that describe most commonly used
physical quantities in the SI system of units; for example, velocity, mass, energy, angle, time, and so on.
The objective of these types is to permit construction of arbitrarily complex distributed control systems without
reliance on any particular vendor-specific data types.

Each message type defined in the SI namespace contains a short overflowing timestamp field of type
\DSDLReference{uavcan.time.SynchronizedAmbiguousTimestamp}.
Every emitted message should be timestamped in order to allow subscribers to identify which of the messages
relate to the same event or to the same instant.
Messages that are emitted in bulk in relation to the same event or the same instant should contain
exactly the same value of the timestamp,
in order to simplify the task of timestamp matching for the subscribers.

The exact strategy of matching related messages by timestamp employed by subscribers is entirely
implementation-defined.
The specification does not concern itself with this matter because it is expected that different applications
will opt for different design trade-offs and different tactics to suit their constraints best.
Such diversity is not harmful, because its effects are always confined to the local node and cannot affect
operation of other nodes or their compatibility.

The table \ref{table:dsdl:uavcan.si} provides a high-level overview of the SI namespace.
Please follow the references for details.

\DSDL{uavcan.si.* --index-only}

\subsubsection{Primitive namespace}

The primitive namespace contains a collection of primitive types:
integer types, floating point types, bit flag, string, raw block of bytes, and an empty value.
Integer, floating point, and bit flag types are available in two categories: scalar and array;
the latter are limited so that their serialized representation is never larger than 257 bytes.

The primitive types are designed to complement the SI namespace with an even simpler set of basic types
that do not make any assumptions about the meaning of the data they describe.
The primitive types provide a very high degree of flexibility, but due to their lack of semantic information,
their usage carries the risk of creating suboptimal interfaces that are difficult to use, validate, and scale.

Normally, the usage of primitive types should be limited to very basic vendor-neutral interfaces for COTS
equipment and software, debug and diagnostic purposes, and whenever there is a need to exchange data the
type of which cannot be determined statically (an example of the latter use case is the register protocol
described in section \ref{sec:application_register_interface}).

The table \ref{table:dsdl:uavcan.primitive} provides a high-level overview of the primitive namespace.
Please follow the references for details.

\DSDL{uavcan.primitive.* --index-only}

\subsection{Remote file system interface}\label{sec:application_file_system}

The set of standard data types contains a collection of services for manipulation of remote file systems
(namespace \DSDLReference{uavcan.file}, see the table \ref{table:dsdl:uavcan.file}).
All basic file system operations are supported, including file reading and writing,
directory listing, metadata retrieval (size, modification time, etc.), moving, renaming, creation, and deletion.

The set of supported operations may be extended in future versions of the protocol.

Implementers of file servers are strongly advised to always support services \verb|Read| and \verb|GetInfo|,
as that allows clients to make assumptions about the minimal degree of available service.
If write operations are required, all of the defined services should be supported.

\DSDL{uavcan.file.* --index-only}

\subsection{Generic node commands}\label{sec:application_generic_node_commands}

Commonly used node-level application-agnostic auxiliary commands
(such as: restart, power off, factory reset, emergency stop, etc.)
are supported via the standard service \DSDLReference{uavcan.node.ExecuteCommand}.
The service also allows vendors to define vendor-specific commands alongside the standard ones.

It is recommended to support this service in all nodes.

\subsection{Node software update}

A simple software\footnote{Or firmware -- UAVCAN does not distinguish between the two.}
update protocol is defined on top of the remote file system interface (section \ref{sec:application_file_system})
and the generic node commands (section \ref{sec:application_generic_node_commands}).

The process of software update involves the following data types:

\begin{itemize}
    \item \DSDLReference{uavcan.node.ExecuteCommand} -- used to initiate the software update process.
    \item \DSDLReference{uavcan.file.Read} -- used to transfer the software image file (or multiple files)
          from the file server to the updated node.
\end{itemize}

The software update protocol logic is described in detail in the documentation for the data type
\DSDLReference{uavcan.node.ExecuteCommand}.
The protocol is considered simple enough to be usable in embedded bootloaders with
small memory-constrained microcontrollers.

\subsection{Register interface}\label{sec:application_register_interface}

UAVCAN defines the concept of \emph{named register} -- a scalar, vector, or string value with an associated
human-readable name that is stored on a UAVCAN node locally and is accessible via
UAVCAN\footnote{And, possibly, other interfaces.} for reading and/or modification
by other nodes on the bus.

Named registers are designed to serve the following purposes:
\begin{description}
    \item[Node configuration parameter management] --- Named registers can be used to expose persistently stored
          values that define behaviors of the local node.

    \item[Diagnostics and monitoring] --- Named registers can be used to expose internal states (variables) of
          the node's decision-making and data processing logic (implemented in software or hardware) to provide
          insights about its inner processes.

    \item[Advanced node information reporting] --- Named registers can store any invariants provided by the vendor,
          such as calibration coefficients or unique identifiers.

    \item[Special functions] --- Non-persistent named registers can be used to trigger specific behaviors or
          commence predefined operations when written.

    \item[Advanced debugging] --- Registers following a specific naming pattern can be used to provide direct read
          and write access to the local node's application software to facilitate in-depth debugging and monitoring.
\end{description}

The register protocol rests upon two service types defined in the namespace \DSDLReference{uavcan.register};
the namespace index is shown in the table \ref{table:dsdl:uavcan.register}.
Data types supported by the register protocol are defined in the nested data structure
\DSDLReference{uavcan.register.Value}.

The UAVCAN specification defines several standard naming patterns to facilitate cross-vendor compatibility
and provide a framework of common basic functionality.

\DSDL{uavcan.register.* --index-only}

\subsection{Diagnostics and event logging}

The message type \DSDLReference{uavcan.diagnostic.Record} is designed to facilitate emission of
human-readable diagnostic messages and event logging,
both for the needs of real-time display\footnote{E.g., messages displayed to a human operator in real time.}
and for long-term storage\footnote{E.g., flight data recording.}.

\subsection{Plug-and-play nodes}

Every UAVCAN node must have a node-ID that is unique within the network.
Normally, such identifiers are assigned by the network designer, integrator, some automatic external tool,
or another entity that is external to the network.
However, there exist circumstances where such manual assignment is either difficult or undesirable.

Nodes that can join any UAVCAN network automatically without any prior manual configuration
are called \emph{plug-and-play nodes} (or \emph{PnP nodes} for brevity).

Plug-and-play nodes automatically obtain a node-ID and deduce all necessary parameters of the physical layer
such as the bit rate.

UAVCAN defines an automatic node-ID allocation protocol that is built on top of the data types defined in the
namespace \DSDLReference{uavcan.pnp} (where \emph{pnp} stands for ``plug-and-play'')
(see table \ref{table:dsdl:uavcan.pnp}).
The protocol is described in the documentation for the data type \DSDLReference{uavcan.pnp.NodeIDAllocationData}.

The plug-and-play node-ID allocation protocol relies on anonymous messages reviewed in the section
\ref{sec:transport_anonymous_message_publication}.
Remember that the plug-and-play feature is entirely optional and it is expected that some applications where a
high degree of determinism and robustness is expected are unlikely to benefit from it.

This feature derives from the work
``In search of an understandable consensus algorithm''%
\footnote{Proceedings of USENIX Annual Technical Conference, p. 305-320, 2014.}
by Diego Ongaro and John Ousterhout.

\DSDL{uavcan.pnp.* --index-only}

\subsection{Internet/LAN forwarding interface}

Data types defined in the namespace \DSDLReference{uavcan.internet} (see table \ref{table:dsdl:uavcan.internet})
are designed for establishing robust direct connectivity between local UAVCAN nodes and hosts on the Internet
or on a local area network (LAN) by means of so called \emph{modem nodes}%
\footnote{Normally, a modem node would be implemented using on-board cellular, radio frequency,
or satellite communication hardware.}
(possibly redundant).

This basic support for world-wide communication provided at the protocol level allows any component
of a vehicle equipped with modem nodes to reach external resources or exchange arbitrary data globally
without dependency on application-specific means of communication%
\footnote{Information security and other security-related concerns are outside of the scope of this specification.}.

The set of supported Internet/LAN protocols may be extended in future revisions of the specification.

Some of the major applications for this feature are as follows:

\begin{enumerate}
    \item Direct telemetry transmission from UAVCAN nodes.
    \item Implementation of remote API for on-board equipment (e.g., web interface).
    \item Reception of real-time correction data streams (e.g., RTCM RC-104) for precise positioning applications.
    \item Automatic upgrades directly from the vendor's Internet resources.
\end{enumerate}

\DSDL{uavcan.internet.* --index-only}
