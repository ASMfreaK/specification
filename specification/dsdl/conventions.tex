\section{Conventions and recommendations}

This section is dedicated to conventions and recommendations
intended to help data type designers maintain a consistent style across the ecosystem
and avoid some common pitfalls.
Some of the conventions are mandatory to follow, others are optional.

\subsection{Standard namespace}

This specification defines a set of standard regulated DSDL data types located under
the root namespace named ``\verb"uavcan"'' (section~\ref{sec:list_of_standard_data_types}).

Vendor-specific, user-specific, or any other data types not defined by this specification
must not be defined inside the standard root namespace.
Vendor-specific or user-specific root namespaces must be used instead.

\subsection{Naming recommendations}

The DSDL naming recommendations follow those that are widely accepted in the general software development industry.

\begin{itemize}
    \item Namespaces and field attributes should be named in the \verb|snake_case|.
    \item Constant attribute should be named in the \verb|SCREAMING_SNAKE_CASE|.
    \item Data types (excluding their namespaces) should be named in the \verb|PascalCase|.
    \item Names of message types should form a declarative phrase or a noun. For example,
    \verb|BatteryStatus| or \verb|OutgoingPacket|.
    \item Names of service types should form an imperative phrase or a verb. For example,
    \verb|GetInfo| or \verb|HandleIncomingPacket|.
    \item Avoid short names, unnecessary abbreviations, and uncommon acronyms.
\end{itemize}

\subsection{Comments}

Every data type definition file should begin with a header comment that provides an exhaustive description
of the data type, its purpose, semantics, usage patterns, any related data exchange patterns,
assumptions, constraints, and all other information that may be necessary or generally useful for the usage of the
data type definition.

Every attribute of the data type definition, and especially every field attribute of it,
should have an associated comment explaining the purpose of the attribute, its semantics, usage patterns,
assumptions, constraints, and any other pertinent information.
Exception applies to attributes supplied with sufficiently descriptive and unambiguous names.

% Field comment placement https://forum.uavcan.org/t/dsdl-documentation-comments/407

\subsection{Optional value representation}

Data structures may include optional field attributes that are not necessarily always populated.

The recommended approach for representing optional field attributes
is to use variable-length arrays with the capacity of one element,
prefixed with padding bits as necessary to retain byte alignment.

Alternatively, such one-element variable-length arrays can be replaced with two-field unions,
where the first field is empty and the second field contains the desired optional value.
The described layout is bit-compatible and semantically compatible with the one-element array described above,
provided that the field attributes are not swapped.

Floating-point-typed field attributes may be assigned the value of not-a-number (NaN) per IEEE 754
to indicate that the value is not specified;
however, this pattern is discouraged because some nodes may not support IEEE 754 NaN values internally,
the value would still have to be transferred over the bus even if not populated,
and special case values undermine type safety.

\begin{remark}[breakable]
    Array-based optional field:

    \begin{minted}{python}
        void7                           # It is recommended to ensure byte alignment.
        MyType[<=1] optional_field
    \end{minted}

    Union-based optional field:

    \begin{minted}{python}
        @union                          # The implicit tag is one bit long.
        uavcan.primitive.Empty none     # Represents lack of value, unpopulated field.
        MyType some                     # The field of interest; field ordering is important.
    \end{minted}

    The defined above union can be used as follows (suppose it is named \verb|MaybeMyType|):

    \begin{minted}{python}
        void7                           # It is recommended to ensure byte alignment.
        MaybeMyType optional_field
    \end{minted}

    The shown approaches are mutually bit-compatible and semantically compatible.
\end{remark}

\subsection{Bit flag representation}

The recommended approach to defining a set of bit flags is to dedicate a \verb|bool|-typed field attribute for each.
Representations based on an integer sum of powers of two\footnote{Which are popular in programming.}
are discouraged due to their obscurity and failure to express the intent clearly.

\begin{remark}
    Recommended approach:

    \begin{minted}{python}
        void5
        bool flag_foo
        bool flag_bar
        bool flag_baz
    \end{minted}

    Not recommended:

    \begin{minted}{python}
        uint8 flags             # Not recommended
        uint8 FLAG_BAZ = 1
        uint8 FLAG_BAR = 2
        uint8 FLAG_FOO = 4
    \end{minted}
\end{remark}
