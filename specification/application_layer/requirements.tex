\section{Application-level requirements}\label{sec:application_level_requirements}

This section describes a set of high-level rules that must be obeyed by all UAVCAN implementations.

\subsection{Port identifier distribution}

An overview of related concepts is provided in chapter \ref{sec:basic_concepts}.

The subject and service identifier values are segregated into three ranges:
unregulated port identifiers that can be freely chosen by users and integrators (both fixed and non-fixed);
regulated fixed identifiers for non-standard data type definitions
that are assigned by the UAVCAN maintainers for publicly released data types;
and regulated identifiers of the standard data types that are an integral part of the UAVCAN specification.

More information on the subject of data type regulation is provided in section
\ref{sec:basic_concepts_data_type_regulation}.

The ranges are summarized in the table \ref{table:application_port_id_distribution}.
Unused gaps are reserved for future expansion of adjacent ranges.

\begin{UAVCANSimpleTable}{Port identifier distribution}{|l l X|}\label{table:application_port_id_distribution}
    Subject-ID          & Service-ID        & Purpose \\
    $[0, 24575]$        & $[0, 127]$        & Unregulated identifiers (both fixed and non-fixed). \\
    $[28672, 29695]$    & $[256, 319]$      & Non-standard fixed regulated identifiers (i.e., vendor-specific). \\
    $[31744, 32767]$    & $[384, 511]$      & Standard fixed regulated identifiers. \\
\end{UAVCANSimpleTable}

\subsection{Standard namespace}

An overview of related concepts is provided in chapter \ref{sec:dsdl}.

This specification defines a set of standard regulated DSDL data types located under
the root namespace named ``\verb"uavcan"'' (section~\ref{sec:list_of_standard_data_types}).

Vendor-specific, user-specific, or any other data types not defined by this specification
must not be defined inside the standard root namespace\footnote{Custom data type definitions shall be located
inside vendor-specific or user-specific namespaces instead.}.
