\section{Application-level functions}\label{sec:application_level_functions}

This section documents the high-level functionality defined by UAVCAN.
The common high-level functions defined by the specification span across different application domains.
All of the functions defined in this section are optional,
except for the node heartbeat feature (section \ref{sec:application_node_heartbeat}),
which is mandatory for all UAVCAN nodes.

The detailed specifications for each function are provided in the DSDL comments of the data type definitions
they are built upon, whereas this section serves as a high-level overview or an index.

\subsection{Node initialization}

UAVCAN does not require that nodes undergo any specific initialization upon connecting to the bus ---
a node is free to begin functioning immediately once it is powered up.
However, the transitioning to the operating state (even if there is no other state to transition from)
is to be reflected via the mandatory heartbeat message, described in section \ref{sec:application_node_heartbeat}.

\subsection{Node heartbeat}\label{sec:application_node_heartbeat}

Every UAVCAN node must report its status and presence by publishing messages of type
\DSDLReference{uavcan.node.Heartbeat} at a fixed rate specified in the message definition.

This is the only high-level protocol function that UAVCAN nodes are required to support.
All other data types and application-level functions are optional.

\DSDL{uavcan.node.Heartbeat}

\subsection{Generic node information}

The service \DSDLReference{uavcan.node.GetInfo} can be used to obtain generic information about the node,
such as the structured name of the node (which includes the name of its vendor),
a 128-bit globally unique identifier, the version information about its hardware and software,
version of the UAVCAN specification implemented on the node, and the optional certificate of authenticity.

While the service is, strictly speaking, optional, omitting its support is highly discouraged,
since it is instrumental for network discovery, firmware update, and various maintenance and diagnostic needs.

\DSDL{uavcan.node.GetInfo}

\subsection{Bus data flow monitoring}

The combination of the following three services defined in the namespace \DSDLReference{uavcan.node.port}
enables a highly capable tool of network inspection and monitoring:
\begin{itemize}
    \item \DSDLReference{uavcan.node.port.List} --- designed for obtaining the full set of subjects and services
    implemented by the server node.

    \item \DSDLReference{uavcan.node.port.GetInfo} --- returns the static (unchanging or slowly changing)
    information about the selected subject or service.

    \item \DSDLReference{uavcan.node.port.GetStatistics} --- returns transfer event counters of
    the selected subject or service.
\end{itemize}

The first service \verb|List| allows the caller to construct a list of all subjects and services used by each
node on the bus.
The second service \verb|GetInfo| allows the caller to map each subject or service to a particular data type,
and understand the role of each node in relation to said subject or service
(publisher, subscriber, or server).
By comparing the data obtained with the help of these two services from each node on the bus,
the caller can reconstruct the data exchange graph for the entire bus,
thus enabling advanced network monitoring and diagnostic purposes.

The last service \verb|GetStatistics| can be used to sample the number of transfers and errors observed
on the specified port.
When invoked periodically, this service allows the caller to observe the real time intensity of data exchange
for each port on each node independently.
In combination with the data exchange graph reconstruction,
this option allows the caller to build a sophisticated real-time view of the bus.

\subsection{Network-wide time synchronization}

UAVCAN provides a simple and robust method of time synchronization%
\footnote{The ability to accurately synchronize time between nodes is instrumental for building distributed
real-time data processing systems such as various robotic applications, autopilots, autonomous driving solutions,
and so on.} that builds upon the work
``Implementing a Distributed High-Resolution Real-Time Clock using the CAN-Bus''
published by M. Gergeleit and H. Streich%
\footnote{Proceedings of the 1st international CAN-Conference 94, Mainz,
13.-14. Sep. 1994, CAN in Automation e.V., Erlangen.}.
The detailed specification of the time synchronization algorithm is provided in the documentation
for the message type \DSDLReference{uavcan.time.Synchronization}.

An important service type \DSDLReference{uavcan.time.GetSynchronizationMasterInfo}
is designed to provide nodes with information about the currently used time system
and related data such as the number of leap seconds.

Redundant time synchronization masters are supported for the benefit of high-reliability applications.

\subsection{Primitive types and physical quantities}

The namespaces \DSDLReference{uavcan.primitive} and \DSDLReference{uavcan.si}
included in the standard data type set are designed to provide a very generic and flexible,
albeit not bandwidth-efficient nor low-latency, method of communication.

Generally, applications where the bus bandwidth and latency are important should minimize their reliance
on these generic data types and favor more specialized types instead that are custom-designed for their
particular application domain; e.g., vendor-specific types or application-specific types, either
designed in-house, published by third parties\footnote{As long as the license permits.}, or supplied by
vendors of COTS equipment used in the application.

Vendors of COTS equipment should always ensure that at least some minimal functionality is available
via these generic types without reliance on their vendor-specific types (if there are any).
This is important for reusability, because it is expected that some of the systems where such COTS nodes are
to be integrated into may not be able to easily support vendor-specific types.

\subsubsection{SI namespace}

The \verb|si| namespace is named after the International System of Units (SI).
The namespace contains a collection of scalar and vector value types that describe most commonly used
physical quantities in the SI system of units; for example, velocity, mass, energy, angle, time, and so on.
The objective of these types is to permit construction of arbitrarily complex data exchange systems without
reliance on any particular vendor-specific data types.

Each message type defined in the SI namespace contains a short overflowing timestamp field of type
\DSDLReference{uavcan.time.SynchronizedAmbiguousTimestamp}.
Every emitted message should be timestamped in order to allow subscribers to identify which of the messages
relate to the same event or to the same instant.
Messages that are emitted in bulk in relation to the same event or the same instant should contain
exactly the same value of the timestamp,
in order to simplify the task of timestamp matching for the subscribers.

The exact strategy of matching related messages by timestamp employed by subscribers is entirely
implementation-defined.
The specification does not concern itself with this matter because it is expected that different applications
will opt for different design trade-offs and different tactics to suit their constraints best.
Such diversity is not harmful, because its effects are always constrained to the local node and cannot affect
operation of other nodes or their compatibility.

The table \ref{table:dsdl:uavcan.si} provides a high-level overview of the SI namespace.
Please follow the references for details.

\DSDL{uavcan.si.* --index-only}

\subsubsection{Primitive namespace}

The primitive namespace contains a collection of primitive types:
integer types, floating point types, bit flag, string, raw block of bytes, and an empty value.
Integer, floating point, and bit flag types are available in two categories: scalar and array;
the latter are limited so that their serialized representation is never larger than 257 bytes.

The primitive types are designed to complement the SI namespace with an even simpler set of basic types
that do not make any assumptions about the data they contain.
The primitive types provide a very high degree of flexibility,
but due to their lack of semantic information,
their usage carries the risk of creating suboptimal interfaces that are difficult to use, validate, and scale.

Normally, the usage of primitive types should be limited to very basic vendor-neutral interfaces for COTS
equipment and software, debug and diagnostic purposes, and whenever there is a need to exchange data the
type of which cannot be determined statically (an example of the latter use case is the register protocol
described in section \ref{sec:application_register_interface}).

The table \ref{table:dsdl:uavcan.primitive} provides a high-level overview of the primitive namespace.
Please follow the references for details.

\DSDL{uavcan.primitive.* --index-only}

\subsection{File transfer}

\subsection{Generic node commands}

\subsection{Register interface}\label{sec:application_register_interface}

\subsection{Diagnostics and event logging}

The message type \DSDLReference{uavcan.diagnostic.Record} is designed to facilitate emission of
human-readable diagnostic messages and event logging,
both for the needs of real-time display\footnote{E.g., messages displayed to a human operator in real time.}
and for long-term storage\footnote{E.g., flight data recording.}.

\subsection{Plug-and-play nodes}


