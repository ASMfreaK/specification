\section{Language specification}

\subsection{Data types and namespaces}

Every data type is located in a \emph{namespace}.
Namespaces may be included into higher-level namespaces, forming a tree hierarchy.

A namespace that is at the root of the tree hierarchy (i.e., not nested within another one)
is called a \emph{root namespace}.
A namespace that is located inside another namespace is called a \emph{nested namespace}.

A data type is uniquely identified by its namespaces and its \emph{short name}.
The short name of a data type is the name of the type itself excluding the containing namespaces.

A \emph{full name} of a data type consists of its short name and all of its namespace names.
The short name and the namespace names included in a full name are called \emph{name components}.
Name components are ordered: the root namespace is always the first component of the name,
followed by the nested namespaces, if there are any, in the order of their nesting;
the short name is always the last component of the full name.
The full name is formed by joining its name components via the ASCII dot character ``\verb|.|'' (ASCII code 46).

A \emph{full namespace} name is the full name without the short name and its component separator.

A \emph{sub-root namespace} is a nested namespace that is located immediately under its root namespace.
Data types that reside directly under their root namespace do not have a sub-root namespace.

The name structure is illustrated on the figure \ref{fig:dsdl_data_type_name_structure}.

\begin{figure}[H]
    $$
    \overbrace{
        \underbrace{
            \underbrace{\texttt{\huge{uavcan}}}_{\substack{\text{root} \\ \text{namespace}}}%
            \texttt{\huge{.}}%
            \underbrace{\texttt{\huge{node}}}_{\substack{\text{nested, also} \\ \text{sub-root} \\ \text{namespace}}}%
            \texttt{\huge{.}}%
            \underbrace{\texttt{\huge{port}}}_{\substack{\text{nested} \\ \text{namespace}}}%
        }_{\text{full namespace}}%
        \texttt{\huge{.}}%
        \underbrace{\texttt{\huge{GetInfo}}}_{\text{short name}}
    }^{\text{full name}}
    $$
    \caption{Data type name structure.\label{fig:dsdl_data_type_name_structure}}
\end{figure}

Data type names are case-sensitive.
However, data type names that differ only in letter case are not permitted\footnote{%
Because that may cause problems with case-insensitive file systems.}.

A name component consists of alphanumeric ASCII characters (which are: \verb|A-Z|, \verb|a-z|, and \verb|0-9|)
and underscore (``\verb|_|'', ASCII code 95).
An empty string is not a valid name component.
The first character of a name component must not be a digit.
A name component must not match any of the reserved word patterns,
which are listed in the table \ref{table:dsdl_reserved_word_patterns}.

The length of a full data type name must not exceed 50
characters\footnote{This includes the name component separators.}.

Every data type definition is assigned a major and minor version number pair.
In order to uniquely identify a data type definition, its version numbers must be specified.
In the following text, the term \emph{version} without a majority qualifier refers to
a pair of major and minor version numbers.

Valid data type version numbers range from 0 to 255, inclusively.
A data type version where both major and minor components are zero is not allowed.

\subsection{File hierarchy}

DSDL data type definitions are contained in UTF-8 encoded text files with a file name extension \verb|.uavcan|.

One file defines exactly one version of a data type,
meaning that each combination of major and minor version numbers must be unique per data type name.
There may be an arbitrary number of versions of the same data type defined alongside each other,
provided that each version is defined no more than once.
Version number sequences can be non-contiguous,
meaning that it is allowed to skip version numbers or remove existing definitions that are neither oldest nor newest.

A data type definition may have an optional fixed port ID\footnote{Chapter \ref{sec:basic_concepts}.} value specified.

The name of a data type definition file is constructed from the following entities
joined via the ASCII dot character ``\verb|.|'' (ASCII code 46), in the specified order:
\begin{itemize}
    \item Fixed port ID in decimal notation, unless a fixed port ID is not defined for this definition.
    \item Short name of the data type (mandatory, always non-empty).
    \item Major version number in decimal notation (mandatory).
    \item Minor version number in decimal notation (mandatory).
    \item File name extension ``\verb|uavcan|'' (mandatory).
\end{itemize}

\begin{figure}[H]
    $$
    \overbrace{%
        \underbrace{\texttt{\huge{432}}}_{\substack{\text{fixed} \\ \text{port ID}}}%
        \texttt{\huge{.}}%
    }^{\text{optional}}%
    \overbrace{%
        \underbrace{\texttt{\huge{GetInfo}}}_{\substack{\text{short name}}}%
        \texttt{\huge{.}}%
        \underbrace{\texttt{\huge{1.0}}}_{\substack{\text{version} \\ \text{numbers}}}%
        \texttt{\huge{.}}%
        \underbrace{\texttt{\huge{uavcan}}}_{\text{file extension}}%
    }^{\text{mandatory}}
    $$
    \caption{Data type definition file name structure.\label{fig:dsdl_definition_file_name_structure}}
\end{figure}

DSDL namespaces are represented as directories, where one directory defines exactly one namespace, possibly nested.
The name of the directory defines the name of its data type name component.
A directory defining a namespace will always define said namespace in its entirety,
meaning that the contents of a namespace cannot be spread across different directories sharing the same name.
One directory cannot define more than one level of
nesting\footnote{For example, ``\texttt{foo.bar}'' is not a valid directory name.
The valid representation would be ``\texttt{bar}'' nested in ``\texttt{foo}''.}.

An example DSDL directory structure is shown on the figure \ref{fig:dsdl_definition_file_name_structure}.

\begin{figure}[H]
    \begin{tabu}{|l|X|} \hline
        \rowfont{\bfseries}
        Directory tree & Entry description \\\hline

        \texttt{vendor\_x/} &
        Root namespace \texttt{vendor\_x}. \\\cline{2-2}

        \texttt{\qquad{}foo/} &
        Nested namespace (also sub-root) \texttt{vendor\_x.foo}. \\\cline{2-2}

        \texttt{\qquad{}\qquad{}100.Run.1.0.uavcan} &
        Data type definition v1.0 with fixed service ID 100. \\\cline{2-2}

        \texttt{\qquad{}\qquad{}100.Status.1.0.uavcan} &
        Data type definition v1.0 with fixed subject ID 100. \\\cline{2-2}

        \texttt{\qquad{}\qquad{}ID.1.0.uavcan} &
        Data type definition v1.0 without fixed port ID. \\\cline{2-2}

        \texttt{\qquad{}\qquad{}ID.1.1.uavcan} &
        Data type definition v1.1 without fixed port ID. \\\cline{2-2}

        \texttt{\qquad{}\qquad{}bar\_42/} &
        Nested namespace \texttt{vendor\_x.foo.bar\_42}. \\\cline{2-2}

        \texttt{\qquad{}\qquad{}\qquad{}101.List.1.0.uavcan} &
        Data type definition v1.0 with fixed service ID 101. \\\cline{2-2}

        \texttt{\qquad{}\qquad{}\qquad{}102.List.2.0.uavcan} &
        Data type definition v2.0 with fixed service ID 102. \\\cline{2-2}

        \texttt{\qquad{}\qquad{}\qquad{}ID.1.0.uavcan} &
        Data type definition v1.0 without fixed port ID. \\\hline
    \end{tabu}
    \caption{DSDL directory structure example.}\label{fig:dsdl_directory_structure_example}
\end{figure}

\subsection{Syntax}

DSDL uses a simple regular grammar which can be parsed using standard POSIX regular expressions.
The following grammar and syntax specification will rely on the standard IEEE POSIX ERE\footnote{IEEE Std 1003.1.}
regular expression notation.



