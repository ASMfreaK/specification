\chapter{CAN bus transport layer}\label{sec:can_bus_transport_layer}

This chapter defines the CAN bus based transport layer of UAVCAN,
both for the legacy CAN 2.0 and for CAN FD.
The differences between the two transports are minor,
so by default all of the information provided in this chapter applies to both,
unless specifically stated otherwise.

\section{The concept of transfer}

A \emph{transfer} is an act of data transmission between nodes.
A transfer that is addressed to all nodes except the source node is a \emph{broadcast transfer}.
A transfer that is addressed to one particular node is a \emph{unicast transfer}.
UAVCAN defines the following types of transfers:

\begin{description}
    \item[Message transfer] - a broadcast transfer that contains a serialized message.
    \item[Service transfer] - a unicast transfer that contains either a service request or a service response.
\end{description}

Both message and service transfers can be further distinguished between:

\begin{description}
    \item[Single-frame transfer] - a transfer that is entirely contained in a single CAN frame.
    \item[Multi-frame transfer] - a transfer that has its payload distributed over multiple CAN frames.
    The UAVCAN protocol stack handles transfer decomposition and reassembly automatically.
\end{description}

The following properties are common to all types of transfers:

\begin{UAVCANSimpleTable}{Common transfer properties}{|l X|}\label{table:common_transfer_properties}
    Property        & Description \\
    Payload         & The serialized data structure. \\
    Data type ID    & A numerical identifier that indicates how the data structure should be interpreted. \\
    Data type major version number & Semantic major version number of the data type definition. \\
    Source node ID  & The node ID of the transmitting node (excepting anonymous message transfers). \\
    Priority        & A non-negative integer value that defines the message urgency (0 is the highest priority).
                      Higher priority transfers can preempt lower priority transfers. \\
    Transfer ID     & A small overflowing integer that increments with every transfer
                      of this type of message from a given node. \\
\end{UAVCANSimpleTable}

\subsection{Message broadcasting}

\subsubsection{Regular message broadcasting}

Message broadcasting is the main method of communication between UAVCAN nodes.

A broadcast message is carried by a single message transfer that contains the serialized message data structure.
A broadcast message does not contain any additional fields besides those listed in the table
\ref{table:common_transfer_properties}.

In order to broadcast a message, the broadcasting node must have a node ID that is unique within the network.
An exception applies to \emph{anonymous message broadcasts}.

\subsubsection{Anonymous message broadcasting}

An anonymous message transfer is a transfer that originates from a node that does not have a node ID.
This sort of message transfer is especially useful for \emph{dynamic node ID allocation}
(a high-level concept that is reviewed in detail in the chapter \ref{sec:application_layer}).

A node that does not have a node ID is said to be in \emph{passive mode}.
Passive nodes are unable to initiate regular data exchanges,
but they can listen to the data exchanged over the bus,
and they can emit anonymous message transfers.

An anonymous message has the same properties as a regular message, except for the source node ID,
which in the case of anonymous message transfers is always assumed to be zero.

An anonymous transfer can only be a single-frame transfer. Multi-frame anonymous message transfers are not allowed.

Note that anonymous messages require specific arbitration rules and have restrictions on the acceptable
data type ID values. The details are explained later in this chapter.

\subsubsection{Message timing requirements}

Generally, a message transmission should be aborted if it cannot be completed in 1 second.
Applications are allowed to deviate from this recommendation,
provided that every such deviation is explicitly documented.
It is expected that high-frequency high-priority messages may opt for lower timeout values,
whereas low-priority data may opt for higher timeout values to account for CAN bus congestion.

\subsection{Service invocation}

A service invocation sequence consists of two related service transfers:

\begin{description}
    \item[Service request transfer] - from the node that invokes the service - the \emph{client} - to the node that
    provides the service - the \emph{server}.

    \item[Service response transfer] - once the \emph{server node} receives the service request and processes it,
    it sends a response transfer back to the \emph{client node}.
\end{description}

The tables \ref{table:service_request_transfer_properties} and \ref{table:service_response_transfer_properties}
describe the properties of service request and service response transfers, respectively.

Both the client and the server must have node ID values that are unique within the network;
service invocation is not available to passive nodes.

\begin{UAVCANSimpleTable}{Service request transfer properties}{|l X|}\label{table:service_request_transfer_properties}
    Property                        & Description \\
    Payload                         & The serialized service request data structure. \\
    Data type ID                    & See the table \ref{table:common_transfer_properties}. \\
    Data type major version number  & See the table \ref{table:common_transfer_properties}. \\
    Source node ID                  & The node ID of the client (the invoking node). \\
    Destination node ID             & The node ID of the server (the invoked node). \\
    Priority                        & See the table \ref{table:common_transfer_properties}. \\
    Transfer ID                     & An integer value that:
        \begin{enumerate}
            \item allows the server to distinguish the request from other requests from the same client;
            \item allows the client to match the response with its request.
        \end{enumerate} \\
\end{UAVCANSimpleTable}

\begin{UAVCANSimpleTable}{Service response transfer properties}{|l X|}\label{table:service_response_transfer_properties}
    Property                        & Description \\
    Payload                         & The serialized service response data structure. \\
    Data type ID                    & Same value as in the request transfer. \\
    Data type major version number  & Same value as in the request transfer. \\
    Source node ID                  & The node ID of the server (the invoked node). \\
    Destination node ID             & The node ID of the client (the invoking node). \\
    Priority                        & Same value as in the request transfer. \\
    Transfer ID                     & Same value as in the request transfer. \\
\end{UAVCANSimpleTable}

\subsubsection{Service timing requirements}

Applications should follow the service invocation timing recommendations specified below.
Applications are allowed to deviate from these recommendations,
provided that every such deviation is explicitly documented.

\begin{itemize}
    \item Service transfer transmission should be aborted if does not complete in 1 second.
    \item The client should stop waiting for a response from the server if one has not arrived within 1 second.
    \item The server should be able to process any request in under 0.5 seconds.
\end{itemize}

\subsection{Transfer prioritization}\label{sec:transfer_prioritization}

UAVCAN transfers are prioritized by means of the transfer priority property,
which allows eight different priority levels for all types of transfers.
The priority levels and the corresponding numerical identifiers are specified
in the table \ref{table:transfer_priority_levels}.
Observe that due to the specifics of the CAN bus,
lower numerical values correspond to higher priority levels.
The human-friendly mnemonics are introduced in order to prevent confusion due to the inverted nature of the
priority level identifiers.

\begin{minipage}{0.6\textwidth}
\begin{UAVCANSimpleTable}{Transfer priority levels}{|l X|}\label{table:transfer_priority_levels}
    Numerical ID            & Mnemonic \\
    0 (highest priority)    & Emergency \\
    1                       & Critical \\
    2                       & Urgent \\
    3                       & High \\
    4                       & Normal \\
    5                       & Low \\
    6                       & Diagnostic \\
    7 (lowest priority)     & Background \\
\end{UAVCANSimpleTable}
\end{minipage}

Transfers with higher priority levels (i.e., numerically lower priority ID)
preempt transfers with lower priority levels, delaying their transmission
until there are no more higher priority transfers to exchange.

Shall there be multiple transfers of different types at the same priority level contesting for the bus access,
UAVCAN ensures the following precedence, from higher priority to lower priority:

\begin{enumerate}
    \item Message transfers.
    \item Service response transfers.
    \item Service request transfers.
\end{enumerate}

Message transfers take precedence over service transfers because message broadcasting is the primary method of
communication in UAVCAN networks.
Service responses take precedence over service requests in order to make service invocations more atomic
and reduce the number of pending states in the system.

Within the same type and the same priority level,
transfers are prioritized according to the data type ID:
transfers with lower data type ID values preempt those with higher data type ID values.

\section{Transfer emission}

\subsection{Transfer ID computation}

The \emph{transfer ID} is a small unsigned integer value that is provided for every outgoing
transfer.
This value is crucial for many aspects of UAVCAN communication; specifically:
\begin{description}
    \item[Message sequence monitoring] - the continuously increasing transfer ID allows receiving nodes to
    detect lost messages and detect when a message stream from any remote node is interrupted.

    \item[Service response matching] - when a server responds to a request, it uses the same transfer ID for the
    response as in the request,
    allowing any node to emit concurrent requests to the same server while being able to
    match each response with the corresponding request.

    \item[CAN frame deduplication] - for single-frame transfers,
    the transfer ID allows receiving nodes to work around the CAN bus
    frame duplication problem\footnote{This is a well-known issue that can be observed even on a properly
    fielded CAN bus caused by the fact that a frame that appears valid to the receiver may under certain
    (rare) conditions appear invalid to the transmitter, triggering the latter to retransmit the frame,
    in which case it will be duplicated on the side of the receiver.
    Sequence counting mechanisms such as the transfer ID or the toggle bit (both of which are used in UAVCAN)
    allow applications to circumvent this problem.} (multi-frame transfers combat the frame duplication
    problem using the toggle bit).

    \item[Multi-frame transfer reassembly] - more info is provided in the section \ref{sec:transfer_reception}.

    \item[Automatic management of redundant interfaces] - the transfer ID parameter allows the UAVCAN protocol
    stack to perform automatic switchover to a back-up interface shall the primary interface fail.
    The switchover logic can be completely transparent to the application, joining several independent
    redundant physical CAN transports into a highly reliable single virtual communication channel.
\end{description}

For message transfers and service request transfers the ID value should be computed as described below.
For service response transfers this value must be directly copied from the corresponding service request transfer.

The logic to compute the transfer ID relies on the concept of \emph{transfer descriptor}.
A transfer descriptor is a set of properties that identify a particular set of transfers that originate
from the same node, share the same data type ID, same data type major version number, and the same type.
The properties that constitute a transfer descriptor are listed below:
\begin{itemize}
    \item Transfer type (message broadcast, service request, etc.).
    \item Data type ID.
    \item Data type major version number.
    \item Source node ID.
    \item Destination node ID (only for unicast\footnote{I.e., service requests and service responses.} transfers).
\end{itemize}

Every non-passive node must maintain a mapping from transfer descriptors to transfer ID counters.
This mapping is referred to as the \emph{transfer ID map}.

Whenever a node needs to emit a transfer, it will query its transfer ID map for the appropriate transfer descriptor.
If the map does not contain such entry, a new entry will be created with the transfer ID counter initialized to zero.
The node will use the current value of the transfer ID from the map for the transfer,
and then the value stored in the map will be incremented by one.
When the stored transfer ID exceeds its maximum value, it will roll over to zero.

It is expected that some nodes will need to publish certain transfers aperiodically or on an ad-hoc basis,
thereby creating unused entries in the transfer ID map.
In order to avoid keeping unused entries in the map, the nodes are allowed, but not required,
to remove entries from the map that were not used for more than 2 seconds.
Therefore, it is possible that a node may publish a transfer with an out-of-order transfer ID value,
if the previous transfer of the same descriptor has been published more than 2 seconds earlier.

\subsection{Single frame transfers}

If the size of the entire transfer payload does not exceed the space available for payload in a single CAN frame,
the whole transfer will be contained in one CAN frame.
Such transfer is called a \emph{single-frame transfer}.

Single frame transfers are more efficient than multi-frame transfers in terms of throughput, latency,
and data overhead.

\subsection{Multi-frame transfers}

\emph{Multi-frame transfers} are used when the size of the transfer payload exceeds the space available
for payload in a single CAN frame.

Two new concepts are introduced in the context of multi-frame transfers, both of which are reviewed below in detail:
\begin{itemize}
    \item Transfer CRC\footnote{CRC stands for "cyclic redundancy check", an error-detecting code
    added to data transmissions to reduce the likelihood of undetected data corruption.}.
    \item Toggle bit.
\end{itemize}

In order to emit a multi-frame transfer, the node must first compute the CRC for the entirety of the transfer payload.
The node appends the CRC value at the end of the transfer payload,
and then emits the resulting byte set in chunks as an ordered sequence of CAN frames
(i.e. the last CAN frame contains the last bytes of the payload and the transfer CRC).
The data field of all CAN frames of a multi-frame transfer, except the last one, must be fully utilized.

All frames of a multi-frame transfer should be pushed to the transmission queue at once,
in the proper order from the first frame to the last frame.
Explicit gap time between CAN frames belonging to the same transfer should not be introduced.

\subsubsection{Transfer CRC}

The objective of the transfer CRC is to allow receiving nodes to validate correctness of
multi-frame transfer reassembly.
It should be understood that the transfer CRC is not intended for bit-level data integrity checks,
as that is managed by the CAN bus transport automatically.
As such, the transfer CRC allows receiving nodes to ensure that all of the frames of a multi-frame
transfer were received, all of the received frames were reassembled in the correct order,
and that all of the received frames belong to the same multi-frame transfer.

The transfer CRC is computed over the entire payload of the transfer.
Certain transports, such as CAN FD, may require a short sequence of padding bytes to be added
at the end of the transfer payload due to low granularity of the frame payload length property;
in that case, the padding bytes are not to be included in the CRC computation.

The resulting CRC value is appended to the transfer in the \emph{big-endian byte order}
(most significant byte first),
in order to take advantage of the CRC residue check intrinsic to this algorithm.

The transfer CRC algorithm specification is provided in the table \ref{table:transfer_crc_params}.

\begin{minipage}{0.7\textwidth}
\begin{UAVCANSimpleTable}{Transfer CRC algorithm parameters}{|ll|}\label{table:transfer_crc_params}
    Property        & Value \\
    Name            & CRC-16/CCITT-FALSE \\
    Initial value   & \texttt{0xFFFF} \\
    Polynomial      & \texttt{0x1021} \\
    Reverse         & No \\
    Output XOR      & $0$ \\
    Residue         & $0$ \\
    Check           & $\left(49, 50, \ldots, 56, 57\right) \rightarrow \mathtt{0x29B1}$ \\
\end{UAVCANSimpleTable}
\end{minipage}

The following code snippet provides an implementation of the transfer CRC algorithm in C++.

\begin{minipage}{0.9\textwidth}
\begin{minted}{cpp}
// UAVCAN transfer CRC algorithm implementation in C++.
// License: CC0, no copyright reserved.

#include <iostream>
#include <cstdint>
#include <cstddef>

class TransferCRC
{
    std::uint16_t value_ = 0xFFFFU;

public:
    void add(std::uint8_t byte)
    {
        value_ ^= static_cast<std::uint16_t>(byte) << 8U;
        for (std::uint8_t bit = 8; bit > 0; --bit)
        {
            if ((value_ & 0x8000U) != 0)
            {
                value_ = (value_ << 1U) ^ 0x1021U;
            }
            else
            {
                value_ = value_ << 1U;
            }
        }
    }

    void add(const std::uint8_t* bytes, std::size_t length)
    {
        while (length-- > 0)
        {
            add(*bytes++);
        }
    }

    [[nodiscard]] std::uint16_t get() const { return value_; }
};

int main()
{
    TransferCRC crc;
    crc.add(reinterpret_cast<const std::uint8_t*>("123456789"), 9);
    std::cout << std::hex << "0x" << crc.get() << std::endl;  // Outputs 0x29B1
    return 0;
}
\end{minted}
\end{minipage}

\subsubsection{Toggle bit}

The toggle bit is a property defined at the CAN frame level.
Its purpose is to detect and avoid CAN frame duplication errors in multi-frame
transfers\footnote{In single-frame transfers, CAN frame deduplication is based on the transfer ID counter.}.

The toggle bit of the first CAN frame of a multi-frame transfer must be set to one.
The toggle bits of the following CAN frames of the transfer must alternate,
i.e., the toggle bit of the second CAN frame must be zero,
the toggle bit of the third CAN frame must be one, and so on.

For single-frame transfers, the toggle bit must be set to one.

Transfers where the initial value of the toggle bit is zero must be ignored.
The initial state of the toggle bit may be inverted in the future revisions of the protocol
to facilitate automatic protocol version detection.

\subsection{Redundant interface support}

In configurations with redundant CAN bus interfaces,
nodes are required to submit every outgoing transfer to the transmission queues of
all available redundant interfaces simultaneously.
It is understood that perfectly simultaneous transmission may not be possible due to different
utilization rates of the redundant interfaces and different phasing of their traffic;
however, that is not an issue for UAVCAN.
If perfectly simultaneous frame submission is not possible, interfaces with lower numerical index
should be handled in the first order.

An exception to the above rule applies if the payload of the transfer depends on some properties
of the interface through which the transfer is emitted.
An example of such a special case is the time synchronization algorithm leveraged by UAVCAN
(documented in the chapter \ref{sec:application_layer} of the specification).

\section{CAN frame format}

UAVCAN utilizes only extended CAN frames with 29-bit identifiers.
UAVCAN can share the same bus with other protocols based on standard (non-extended) CAN frames with 11-bit identifiers.
However, future revisions of UAVCAN may utilize 11-bit identifiers as well;
therefore, backward compatibility with other protocols is not guaranteed.

\subsection{CAN ID field structure}

UAVCAN utilizes three different CAN ID formats for different types of transfers:
message transfers, service transfers, and anonymous message transfers.
The structure is summarized in the table \ref{table:can_id_structure}.

% Please do not remove the hard placement specifier [H], it is needed to keep tables ordered.
\begin{table}[H]\caption{CAN ID field structure}\label{table:can_id_structure}
\NoLeftSkip
\begin{tabu}{|c|X[c]|X[c]|X[c]|c|}
    \hline
    \rowfont{\bfseries}
    Bit & Service & Message & Anonymous message & Bit \\\hline

    28 & \multicolumn{3}{c|}{\multirow{3}{*}{Transfer priority}} & 28 \\
    27 & \multicolumn{3}{c|}{} & 27 \\
    26 & \multicolumn{3}{c|}{} & 26 \\
    \hline

    25 & \multicolumn{2}{c|}{Service not message} & \multirow{4}{*}{Reserved, required =0} & 25 \\\cline{2-3}
    24 & Request not response & \multirow{16}{*}{Message data type ID} & & 24 \\\cline{2-2}
    23 & \multirow{8}{*}{Service data type ID} & & & 23 \\
    22 & & & & 22 \\\cline{4-4}
    21 & & & \multirow{3}{*}{Message DTID modulo 8} & 21 \\
    20 & & & & 20 \\
    19 & & & & 19 \\\cline{4-4}
    18 & & & \multirow{10}{*}{Payload discriminator} & 18 \\
    17 & & & & 17 \\
    16 & & & & 16 \\\cline{2-2}
    15 & \multirow{7}{*}{Destination node ID} & & & 15 \\
    14 & & & & 14 \\
    13 & & & & 13 \\
    12 & & & & 12 \\
    11 & & & & 11 \\
    10 & & & & 10 \\
    9 & & & & 9 \\
    \hline

    8 & \multicolumn{3}{c|}{\multirow{7}{*}{Source node ID}} & 8 \\
    7 & \multicolumn{3}{c|}{} & 7 \\
    6 & \multicolumn{3}{c|}{} & 6 \\
    5 & \multicolumn{3}{c|}{} & 5 \\
    4 & \multicolumn{3}{c|}{} & 4 \\
    3 & \multicolumn{3}{c|}{} & 3 \\
    2 & \multicolumn{3}{c|}{} & 2 \\
    \hline

    1 & \multicolumn{3}{c|}{\multirow{2}{*}{Data type major version number modulo 4}} & 1 \\
    0 & \multicolumn{3}{c|}{} & 0 \\
    \hline
    \rowfont{\bfseries}
    Bit & Service & Message & Anonymous message & Bit \\\hline
\end{tabu}
\end{table}

The fields are described in detail in the following sections.
The tables \ref{table:can_id_fields_message_transfer},
\ref{table:can_id_fields_anonymous_message_transfer}, and \ref{table:can_id_fields_service_transfer}
summarize the purpose of the fields and their permitted values
for message transfers, anonymous message transfers, and service transfers, respectively.
The following acronyms are used for brevity:
\begin{description}
    \item[DTID] - data type ID.
    \item[DTMVN] - data type major version number.
\end{description}

\begin{UAVCANSimpleTable}{CAN ID fields for message transfers}{|l l l X|}
    \label{table:can_id_fields_message_transfer}
    Field               & Width & Permitted values  & Description \\
    Transfer priority   & 3     & [0, 7] (any)      & \\
    Service not message & 1     & 0                 & Always zero for message transfers. \\
    Message DTID        & 16    & [0, 65535] (any)  & Data type ID of the encoded message data structure. \\
    Source node ID      & 7     & [1, 127]          & Node ID of the origin. \\
    Message DTMVN       & 2     & [0, 3] (any)      & Major version number of the data type, modulo 4. \\
\end{UAVCANSimpleTable}

\begin{UAVCANSimpleTable}{CAN ID fields for anonymous message transfers}{|l l l X|}
    \label{table:can_id_fields_anonymous_message_transfer}
    Field               & Width & Permitted values  & Description \\
    Transfer priority   & 3     & [0, 7] (any)      & \\
    Reserved field      & 4     & 0                 & Set to zero when emitting. When receiving, ignore the
                                                      frame if this field is not zero. \\
    Message DTID modulo 8& 3    & [0, 7] (any)      & Three least significant bits of the data type ID of the
                                                      encoded message data structure. Message types where DTID is
                                                      greater than 7 cannot be used with anonymous message transfers. \\
    Payload discriminator & 10  & [0, 1023] (any)   & Used for CAN ID conflict avoidance; more info below. \\
    Source node ID      & 7     & 0                 & Set to zero. This field is used to distinguish anonymous message
                                                      transfers from regular message transfers. \\
    Message DTMVN       & 2     & [0, 3] (any)      & Major version number of the data type, modulo 4. \\
\end{UAVCANSimpleTable}

\begin{UAVCANSimpleTable}{CAN ID fields for service transfers}{|l l l X|}
    \label{table:can_id_fields_service_transfer}
    Field               & Width & Permitted values  & Description \\
    Transfer priority   & 3     & [0, 7] (any)      & \\
    Service not message & 1     & 1                 & Always one for service transfers. \\
    Request not response& 1     & \{0, 1\} (any)    & 1 for service request, 0 for service response. \\
    Service DTID        & 8     & [0, 255] (any)    & Data type ID of the encoded service data structure
                                                      (request or response). \\
    Destination node ID & 7     & [1, 127]          & Node ID of the destination
                                                      (i.e., server for requests, client for responses). \\
    Source node ID      & 7     & [1, 127]          & Node ID of the origin
                                                      (i.e., client for requests, server for responses). \\
    Service DTMVN       & 2     & [0, 3] (any)      & Major version number of the data type, modulo 4. \\
\end{UAVCANSimpleTable}

\subsubsection{Transfer priority}

Valid values for priority range from 0 to 7, inclusively,
where 0 corresponds to the highest priority, and 7 corresponds to the lowest priority.
Mnemonics for transfer priority levels are provided in the table \ref{table:transfer_priority_levels}.

In multi-frame transfers, the value of the priority field must be identical for all frames of the transfer.

\subsubsection{Data type ID}

A higher-level review of the concept of data type ID is available in the chapter \ref{sec:dsdl}.

Valid values of message type ID belong to the range [0, 65535].
For anonymous message transfers, however, the range of usable message type ID values
is limited to [0, 7];
messages with data type ID outside of this range cannot be used with anonymous message
transfers\footnote{This is considered to be an acceptable limitation because anonymous transfers
are intended for an extremely limited set of use cases.}.

Valid values of service type ID belong to the range [0, 255].

More information on the data type ID value ranges is available in the chapter \ref{sec:application_layer}.

\subsubsection{Data type major version number}

\subsubsection{Node ID}

\subsubsection{Payload discriminator}

\section{Transfer reception}\label{sec:transfer_reception}

\section{CAN bus requirements}



