%
% Copyright (c) 2018  Pavel Kirienko <pavel.kirienko@zubax.com>
%

\documentclass{uavcandoc}

\title{Specification v1.0}

\hbadness=10000

\begin{document}
\frontmatter

\begin{titlepage}

\section*{Overview}

UAVCAN is an open lightweight protocol designed for reliable communication in aerospace and robotic applications via CAN bus.

Features:

\begin{itemize}
    \item Democratic network - no bus master, no single point of failure.
    \item Publish/subscribe and request/response (RPC\footnote{Remote procedure call}) exchange semantics.
    \item Efficient exchange of large data structures with automatic decomposition and reassembly.
    \item Lightweight, deterministic, easy to implement, and easy to validate.
    \item Suitable for deeply embedded, resource constrained, hard real-time systems.
    \item Doubly- or triply- redundant CAN bus support.
    \item Supports high-precision network-wide time synchronization.
    \item The specification and high quality reference implementations in popular programming languages are free and open source.
\end{itemize}

\BeginRightColumn

\section*{Documentation and support}

Information, documentation, and discussions related to UAVCAN are available via the official website at
\href{http://uavcan.org}{uavcan.org}.

\section*{Legal statement}

UAVCAN is an interface standard open to everyone.
No copyrights are reserved and no licensing is necessary for its implementation, distribution, or use.

In no event shall the authors of the standard be liable for any damage arising, directly or indirectly, from its use.

\end{titlepage}

\tableofcontents
\BeginRightColumn
\listoffigures
\listoftables

\mainmatter

\chapter{Overview}

\end{document}
