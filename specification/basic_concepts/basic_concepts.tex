\chapter{Basic concepts}\label{sec:basic_concepts}

\section{Main principles}

\subsection{Communication}

\subsubsection{Architecture}

A UAVCAN network is a decentralized peer network, where each peer (node) has a unique
numeric identifier\footnote{Here and elsewhere in this specification, \emph{ID} and \emph{identifier} are used
interchangeably unless specifically indicated otherwise.}
--- \emph{node-ID} --- ranging from 0 up to a transport-specific upper boundary which is guaranteed to be
not less than 127.
Nodes of a UAVCAN network can communicate using the following communication methods:

\begin{description}
    \item[Message publication] --- The primary method of data exchange with one-to-many publish/subscribe semantics.
    \item[Service invocation] --- The communication method for one-to-one request/response
    interactions\footnote{Like remote procedure call (RPC).}.
\end{description}

For each type of communication, a predefined set of data types is used,
where each data type has a unique name.
Additionally, every data type definition has a pair of major and minor version numbers,
which enable data type definitions to evolve in arbitrary ways while ensuring a well-defined
migration path if backward-incompatible changes are introduced.
Some data types are standard and defined by the protocol specification (of which only a small
subset are required); others may be specific to a particular application or vendor.

\subsubsection{Subjects and services}\label{sec:basic_subjects_and_services}

Message exchanges between nodes are grouped into \emph{subjects} by the semantic meaning of the message.
Message exchanges belonging to the same subject use same message data type down to the major version
(minor versions are allowed to differ) and pertain to same function or process within the system.

Request/response exchanges between nodes are grouped into \emph{services} by the semantic meaning
of the request and response,
like messages are grouped into subjects.
Requests and their corresponding responses that belong to the same service use same service data type down to
the major version (minor versions are allowed to differ; as a consequence, the minor data type version number
of a service response may differ from that of its corresponding request) and pertain to the same function.

Each message subject is identified by a unique natural number -- a \emph{subject-ID};
likewise, each service is identified by a unique \emph{service-ID}.
An umbrella term \emph{port-ID} is used to refer either to a subject-ID or to a service-ID
(port identifiers have no direct manifestation in the construction of the protocol,
but they are convenient for discussion).
The sets of subject-ID and service-ID are orthogonal.

Port identifiers are assigned to various functions, processes, or data streams within the network
at the system definition time.
Generally, a port identifier can be selected arbitrarily by a system integrator
by changing relevant configuration parameters of connected nodes,
in which case such port identifiers are called \emph{non-fixed port identifiers}.
It is also possible to permanently associate any data type definition with a particular port identifier
at a data type definition time,
in which case such port identifiers are called \emph{fixed port identifiers};
their usage is governed by rules and regulations described in later sections.

A port-ID used in a given UAVCAN network shall not be shared between functions, processes, or data streams
that have different semantic meaning.

A port-ID used in a given UAVCAN network shall not be used with different data type definitions unless they
share the same full name and the same major version number\footnote{More on data type versioning in section
\ref{sec:dsdl_versioning}.}.

A data type of a given major version can be used simultaneously with
an arbitrary number of non-fixed different port identifiers,
but not more than one fixed port identifier.

\subsection{Data types}

\subsubsection{Data type definitions}

Message and service data types
are defined using the \emph{data structure description language} (DSDL) (chapter \ref{sec:dsdl}).
A DSDL definition specifies the name, major version, minor version, the data schemas,
and an optional fixed port-ID of the data type among other less important properties.
Message data types always define exactly one data schema, whereas
service data types contain two independent schema definitions: one for request, and the other for response.

\subsubsection{Regulation}\label{sec:basic_concepts_data_type_regulation}

Data type definitions can be created by the UAVCAN specification maintainers or by its users,
such as equipment vendors or application designers.
Irrespective of the origin, data types can be included into the set of data type definitions maintained
and distributed by the UAVCAN specification maintainers;
definitions belonging to this set are termed \emph{regulated data type definitions}.
The specification maintainers undertake to keep regulated definitions well-maintained and may occasionally
amend them and release new versions, if such actions are believed to benefit the protocol.
User-created (i.e., vendor-specific or application-specific) data type definitions that are
not included into the aforementioned set are called \emph{unregulated data type definitions}.

Unregulated definitions that are made available for reuse by others are called
\emph{unregulated public data type definitions};
those that are kept closed-source for private use by their authors are called
\emph{(unregulated) private data type definitions}\footnote{The word \emph{``unregulated''} is redundant
because private data types cannot be regulated, by definition.
Likewise, all regulated definitions are public, so the word \emph{``public''} can be omitted.}.

Data type definitions authored by the specification maintainers for the purpose of supporting and advancing
this specification are called \emph{standard data type definitions}.
All standard data type definitions are regulated.

Fixed port identifiers can be used only with regulated data type definitions or with private definitions.
Fixed port identifiers must not be used with public unregulated data types,
since that is likely to cause unresolvable port identifier collisions\footnote{%
    Any system that relies on data type definitions with fixed port identifiers provided by
    an external party (i.e., data types and the system in question are designed by different parties)
    runs the risk of encountering port identifier conflicts that cannot be resolved without resorting to help
    from said external party since the designers of the system do not have control over their fixed port identifiers.
    Because of this, the specification strongly discourages the use of fixed unregulated private port identifiers.
    If a data type definition is ever disclosed to any other party (i.e., a party that did not author it)
    or to the public at large it is important that the data type \emph{not} include a fixed port-identifier.
}.
This restriction shall be followed at all times by all compliant implementations and
systems\footnote{%
    In general, private unregulated fixed port identifiers are collision-prone by their nature, so they should
    be avoided unless there are very strong reasons for their usage and the authors fully understand the risks.
}.

\begin{UAVCANSimpleTable}{Data type taxonomy}{|l|X|X|}
    & Regulated & Unregulated \\
    \bfseries{Public}
    &
    Standard and contributed (e.g., vendor-specific) definitions.\newline
    Fixed port identifiers are allowed; they are called \emph{``regulated port-ID''}.
    &
    Definitions distributed separately from the UAVCAN specification.\newline
    Fixed port identifiers are \emph{not allowed}.
    \\

    \bfseries{Private}
    &
    Nonexistent category.
    &
    Definitions that are not available to anyone except their authors.\newline
    Fixed port identifiers are permitted (although not recommended);
    they are called \emph{``unregulated fixed port-ID''}.
    \\
\end{UAVCANSimpleTable}

DSDL processing tools shall prohibit unregulated fixed port identifiers by default,
unless they are explicitly configured otherwise.

Each of the two sets of valid port identifiers (which are subject identifiers and service identifiers) are
segregated into three categories (the ranges are documented in chapter \ref{sec:application_layer}):

\begin{itemize}
    \item Application-specific port identifiers.
    These can be assigned by changing relevant configuration parameters of the connected nodes
    (in which case they are called \emph{non-fixed} or runtime-assigned),
    or at the data type definition time (in which case they are called \emph{fixed unregulated},
    and they generally should be avoided due to the risks of collisions as explained earlier).

    \item Regulated non-standard fixed port identifiers.
    These are assigned by the specification maintainers for non-standard contributed
    vendor-specific public data types.

    \item Standard fixed port identifiers. These are assigned by the specification maintainers
    for standard regulated public data types.
\end{itemize}

Data type authors that want to release regulated data type definitions or contribute to the standard data
type set should contact the UAVCAN maintainers for coordination.
The maintainers will choose unoccupied fixed port identifiers for use with the new definitions, if necessary.
Since the set of regulated definitions is maintained in a highly centralized manner,
it can be statically ensured that no identifier collisions will take place within it;
also, since the identifier ranges used with regulated definitions are segregated,
regulated port-IDs will not conflict with any other compliant UAVCAN node or system\footnote{%
    The motivation for the prohibition of fixed port identifiers in unregulated public data types is
    derived directly from the above: since there is no central repository of unregulated definitions,
    collisions would be likely.
}.

\subsubsection{Serialization}

A DSDL description can be used to automatically generate the serialization and deserialization code
for every defined data type in a particular programming language.
Alternatively, a DSDL description can be used to construct appropriate serialization code manually by a human.
DSDL ensures that the worst case memory footprint and computational complexity per data type
are constant and easily predictable.

Serialized message and service objects\footnote{%
    Here and elsewhere, an \emph{object} means a value that is an instance of a well-defined type.
}
are exchanged by means of the transport
layer (chapter \ref{sec:transport_layer}), which implements automatic decomposition of
long transfers into several transport frames\footnote{Here and
elsewhere, a \emph{transport frame} means a block of data that can be atomically exchanged
over the transport layer network, e.g., a CAN frame.} and reassembly from these transport frames
back into a single atomic data block, allowing nodes to exchange serialized objects of
arbitrary size (DSDL guarantees, however, that the minimum and maximum size of the serialized representation
of any object of any data type is always known statically).

\subsection{High-level functions}

On top of the standard data types, UAVCAN defines a set of standard high-level functions including:
node health monitoring, node discovery, time synchronization, firmware update,
plug-and-play node support, and more.
For more information see chapter \ref{sec:application_layer}.

\begin{figure}[hbt]
    \centering
    \begin{tabular}{|c|c|l|c|l|c|}
        \hline
        \multicolumn{6}{|c|}{Applications} \\ \hline

        \qquad{} & Required functions &
        \qquad{} & Standard functions &
        \qquad{} & Custom functions \\
        \cline{2-2} \cline{4-4} \cline{6-6}

        \multicolumn{2}{|c|}{Required data types} &
        \multicolumn{2}{c|}{Standard data types} &
        \multicolumn{2}{c|}{Custom data types} \\ \hline

        \multicolumn{6}{|c|}{Serialization} \\ \hline

        \multicolumn{6}{|c|}{Transport} \\ \hline
    \end{tabular}
    \caption{UAVCAN architectural diagram.\label{fig:architecture}}
\end{figure}

\section{Message publication}

Message publication refers to the transmission of a serialized message object over the network to other nodes.
This is the primary data exchange mechanism used in UAVCAN;
it is functionally similar to raw data exchange with minimal overhead,
additional communication integrity guarantees, and automatic decomposition and reassembly of long payloads
across multiple transport frames.
Typical use cases may include transfer of the following kinds of data (either cyclically or on an ad-hoc basis):
sensor measurements, actuator commands, equipment status information, and more.

Information contained in a published message is summarized in table \ref{table:basic_message}.

\begin{UAVCANSimpleTable}{Published message properties}{|l X|}\label{table:basic_message}
    Property        & Description \\
    Payload         & The serialized message object. \\
    Subject-ID      & Numerical identifier that indicates how the information should be interpreted. \\
    Source node-ID  & The node-ID of the transmitting node (excepting anonymous messages). \\
    Transfer-ID     & An integer value that is used for message sequence monitoring,
                      multi-frame transfer reassembly, deduplication, automatic management of redundant transports,
                      and other purposes. \\
\end{UAVCANSimpleTable}

\subsection{Anonymous message publication}

Nodes that don't have a unique node-ID can publish only \emph{anonymous messages}.
An anonymous message is different from a regular message in that it doesn't contain a source node-ID,
and that it can't be decomposed across several transport frames.

UAVCAN nodes will not have an identifier initially until they are assigned one,
either statically (which is generally the preferred option for applications where a high degree of
determinism and high safety assurances are required) or automatically (i.e., plug-and-play).
Anonymous messages are particularly useful for the plug-and-play feature,
which is explored in detail in chapter~\ref{sec:application_layer}.

Anonymous messages cannot be decomposed into multiple transport frames,
meaning that their payload capacity is limited to that of a single transport frame.
More info is provided in chapter~\ref{sec:transport_layer}.

\section{Service invocation}

Service invocation is a two-step data exchange operation between exactly two nodes: a client and a server.
The steps are\footnote{The request/response semantic is facilitated by means of hardware (if available)
or software acceptance filtering and higher-layer logic.
No additional support or non-standard transport layer features are required.}:

\begin{enumerate}
    \item The client sends a service request to the server.
    \item The server takes appropriate actions and sends a response to the client.
\end{enumerate}

Typical use cases for this type of communication include:
node configuration parameter update, firmware update, an ad-hoc action request, file transfer,
and other functions of similar nature.

Information contained in service requests and responses is summarized in table \ref{table:basic_service}.
Both the request and the response contain same values for all listed fields except payload,
where the content is application-defined.

\begin{UAVCANSimpleTable}{Service request/response properties}{|l X|}\label{table:basic_service}
    Property        & Description \\
    Payload         & The serialized request/response object. \\
    Service-ID      & Numerical identifier that indicates how the service should be handled. \\
    Client node-ID  & Source node-ID during request transfer, destination node-ID during response transfer. \\
    Server node-ID  & Destination node-ID during request transfer, source node-ID during response transfer. \\
    Transfer-ID     & An integer value that is used for request/response matching,
                      multi-frame transfer reassembly, deduplication, automatic management of redundant transports,
                      and other purposes. \\
\end{UAVCANSimpleTable}
