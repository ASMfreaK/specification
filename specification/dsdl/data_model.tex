\section{Data model}

\subsection{Kinds}

All versions of a data type definition must be of the same kind\footnote{%
For example, if a data type version 0.1 is of a message kind, all later versions of it must be messages, too.}.

\subsubsection{Message}

A message data type definition contains the description of the message data structure without any additional entities.
The data structure description may be empty, i.e. it may define no fields or constants.

\subsubsection{Service}

A service data type definition contains the description of two data structures:
the service request structure followed by the service response structure.
The two structures are separated by the service response marker (``\verb|---|'') on a separate line.
Either or both of the structures may be empty.

The presence of the service response marker indicates that the data type definition at hand is of the service kind.
There may be no more than one service response marker in a given definition.

\subsection{Types}

\subsubsection{Void types}

Void types are used for padding purposes.
As will be explained in later sections, it is desirable to align fields at byte boundaries;
void types can be used to facilitate that.

Void-typed fields are set to zero when a data structure is serialized and ignored when it is deserialized.
Void types can be used to reserve space in data type definitions for possible use in later versions of the data type.

The DSDL name pattern for void types is as follows: ``\verb|void[1-9]\d?|'',
where the trailing one- or two-digit integer represents the length of the void field, in bits,
ranging from 1 to 64, inclusive.

Void types can be referred to directly by their name from any namespace.

\subsubsection{Primitive types}

Primitive types are defined here. They are assumed to be known to DSDL processing tools a priori,
and as such, they need not be defined by the user.
Primitive types can be referred to directly by their name from any namespace.

The following text defines DSDL name patterns using the POSIX Extended Regular Expression (ERE) notation.

The hierarchy of primitive types is documented below.

\begin{itemize}
    \item \textbf{Boolean types.} A boolean-typed value represents a variable of the Boolean algebra.
    A Boolean-typed value can have two values: true and false.
    The corresponding DSDL data type name is ``\verb|bool|''.

    \item \textbf{Algebraic types.} Those are types for which conventional algebraic operators are defined.
    \begin{itemize}
        \item \textbf{Integer types} are used to represent signed and unsigned integer values.
        See table \ref{table:dsd_integer_properties}.
        \begin{itemize}
            \item \textbf{Signed integer types.} These are used to represent values which can be negative.
            The corresponding DSDL data type name pattern is ``\verb|int[1-9]\d?|'',
            where the trailing one- or two-digit integer represents the length of the
            encoded representation of the value, in bits, ranging from 2 to 64, inclusive.

            \item \textbf{Unsigned integer types.} These are used to represent non-negative values.
            The corresponding DSDL data type name pattern is ``\verb|uint[1-9]\d?|'',
            where the trailing one- or two-digit integer represents the length of the
            encoded representation of the value, in bits, ranging from 2 to 64, inclusive.
        \end{itemize}

        \item \textbf{Floating point types} are used to approximately represent real values.
        The underlying encoded representation follows the IEEE 754 standard.
        The corresponding DSDL data type name pattern is ``\verb~float(16|32|64)~'', where the trailing
        integer represents the type of the IEEE 754 representation.
        See table \ref{table:dsd_floating_point_properties}.
    \end{itemize}
\end{itemize}

\begin{UAVCANSimpleTable}{Properties of integer types}{|l X l|}
    Category &
    DSDL names &
    Range, $X$ is bit length
    \label{table:dsd_integer_properties} \\

    Signed integers &
    \texttt{int2}, \texttt{int3}, \texttt{int4} \ldots{} \texttt{int62}, \texttt{int63}, \texttt{int64} &
    $\left[-\frac{2^{X}}{2},\frac{2^{X}}{2}-1\right]$ \\

    Unsigned integers &
    \texttt{uint2}, \texttt{uint3}, \texttt{uint4} \ldots{} \texttt{uint62}, \texttt{uint63}, \texttt{uint64} &
    $\left[0,2^{X}-1\right]$ \\
\end{UAVCANSimpleTable}

\begin{UAVCANSimpleTable}{Properties of floating point types}{|X X X X|}
    DSDL name        & Representation    & Approximate epsilon   & Approximate range
    \label{table:dsd_floating_point_properties} \\

    \texttt{float16} & IEEE 754 binary16 & $0.001$               & $\pm{}65504$      \\
    \texttt{float32} & IEEE 754 binary32 & $10^{-7}$             & $\pm{}10^{39}$    \\
    \texttt{float64} & IEEE 754 binary64 & $2 \times{} 10^{-16}$ & $\pm{}10^{308}$   \\
\end{UAVCANSimpleTable}

An exhaustive list of all void and primitive types ordered by bit length is provided below for reference.

\immediate\write18{rm -f ../latex.tmp}
\immediate\write18{../render_list_of_void_and_primitive_types.py > ../latex.tmp}
\immediate\input{../latex.tmp}

\subsubsection{Array types}

\subsubsection{Compound types}

\subsection{Fields}

% padding fields have no name

\subsection{Constants}
