\section{Directives}\label{sec:dsdl_directives}

Per the DSDL grammar specification (section~\ref{sec:dsdl_grammar}),
a directive may or may not have an associated expression.
In this section, it is assumed that a directive does not expect an expression unless explicitly stated otherwise.

If the expectation of an associated directive expression or lack thereof is violated,
the containing DSDL definition is malformed.

\subsection{Tagged union marker}

The identifier of the tagged union marker directive is ``\verb|union|''.
Presence of this directive in a data type definition indicates that the
data schema definition containing this directive belongs to the tagged union category
(section~\ref{sec:dsdl_composite_tagged_unions}).

Usage of this directive is subject to the following constraints:
\begin{itemize}
    \item The directive shall not be used more than once per data schema definition.
    \item The directive shall be placed before the first composite type attribute definition in the current data schema.
\end{itemize}

\begin{remark}
    \begin{minted}{python}
        uint8[<64] name # Request is not a union
        ---
        @union          # Response is a union
        uint64 natural
        #@union         # Would fail - @union is not allowed after the first attribute definition
        float64 real
    \end{minted}
\end{remark}

\subsection{Deprecation marker}

The identifier of the deprecation marker directive is ``\verb|deprecated|''.
Presence of this directive in a data type definition indicates that the current version of the data type definition
is nearing the end of its life cycle and may be removed soon.
The data type versioning principles are explained in section~\ref{sec:dsdl_versioning}.

Code generation tools should use this directive to reflect the impending removal of the current data type version
in the generated code.

Usage of this directive is subject to the following constraints:
\begin{itemize}
    \item The directive shall not be used more than once per data type definition.
    \item The directive shall be placed before the first composite type attribute definition in
          the first data schema definition.
\end{itemize}

\begin{remark}
    \begin{minted}{python}
        @deprecated         # Applies to the entire definition
        uint8 FOO = 123
        #@deprecated        # Would fail - shall be placed before the first attribute definition
        ---
        #@deprecated        # Would fail - shall be placed in the first data schema definition
    \end{minted}

    A C++ class generated from the above definition could be annotated with the \verb|[[deprecated]]| attribute.

    A Rust structure generated from the above definition could be annotated with the \verb|#[deprecated]| attribute.

    A Python class generated from the above definition could raise \verb|DeprecationWarning| upon usage.
\end{remark}

\subsection{Assertion check}

The identifier of the assertion check directive is ``\verb|assert|''.
The assertion check directive expects an expression which shall yield a value of type
``\verb|bool|'' (section~\ref{sec:dsdl_primitive_types}) upon its evaluation.

If the expression yields truth, the assertion check directive has no effect.

If the expression yields falsity, a value of type other than ``\verb|bool|'', or fails to evaluate,
the containing DSDL definition is malformed.

\begin{remark}
    \begin{minted}{python}
        float64 real
        @assert _offset_ == {32}    # Would fail: {64} != {32}
    \end{minted}
\end{remark}

\subsection{Print}

The identifier of the print directive is ``\verb|print|''.
The print directive may or may not be provided with an associated expression.

If the expression is not provided, the behavior is implementation-defined.

If the expression is provided, it is evaluated and its result is displayed by the DSDL processing tool in
a human-readable implementation-defined form.
Implementations should strive to produce textual representations that form valid DSDL expressions themselves,
so that they would produce the same value if evaluated by a DSDL processing tool.

If the expression is provided but cannot be evaluated, the containing DSDL definition is malformed.

\begin{remark}
    \begin{minted}{python}
        float64 real
        @print _offset_ / 6                  # Possible output: {32/3}
        @print uavcan.node.Heartbeat.1.0     # Possible output: uavcan.node.Heartbeat.1.0
        @print bool[<4]                      # Possible output: saturated bool[<=3]
        @print float64                       # Possible output: saturated float64
        @print {123 == 123, false}           # Possible output: {true, false}
        @print 'we all float64 down here\n'  # Possible output: 'we all float64 down here\n'
    \end{minted}
\end{remark}

