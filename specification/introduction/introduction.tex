\chapter{Introduction}\label{sec:introduction}

This chapter covers the basic concepts that govern development and maintenance of the specification.
The actual specification is contained in the following chapters.

The reader should have a solid understanding of the main concepts and operating principles of the CAN bus.

\section{Core design goals}

UAVCAN is designed to adhere to the following set of basic principles.

\begin{description}
    \item[Democratic network] - There should be no master node.
    All nodes in the network should have the same communication rights; there should be no single point of failure.

    \item[Nodes can exchange long payloads] - Nodes must be provided with a simple way to exchange large
    data structures that cannot fit into a single CAN\footnote{Here and in the following parts of the
    specification, "CAN" implies both CAN 2.0 and CAN FD, unless specifically noted otherwise.}
    frame (such as GNSS solutions, 3D vectors, etc.).
    UAVCAN should perform automatic transfer decomposition and reassembly at the protocol level,
    hiding the related complexity from the application.

    \item[Support for redundant interfaces and redundant nodes] - This is a common requirement for
    safety-critical applications.

    \item[High throughput, low latency communication] - Applications that are dependent on high-frequency,
    hard real-time control loops, require a low-latency, high-throughput communication method.

    \item[Simple logic, low computational requirements] - UAVCAN targets a wide variety of embedded systems,
    from high-performance embedded on-board computers for intensive data processing
    (e.g., a high-performance GNU/Linux-powered machine) to extremely resource-constrained microcontrollers.
    The latter imposes severe restrictions on the amount of logic needed to implement the protocol.

    \item[Common high-level functions should be clearly defined] - UAVCAN defines standard services
    and messages for common high-level functions, such as network discovery, node configuration,
    node software update, node status monitoring (which naturally grows into a vehicle-wide health monitoring),
    network-wide time synchronization, dynamic node ID allocation (a.k.a. plug-and-play node support), etc.

    \item[Open specification and reference implementations] - The UAVCAN specification is open and
    freely available; the reference implementations are distributed under the terms of the permissive MIT License.
\end{description}

\section{Capabilities}

This section summarizes the capabilities of the UAVCAN protocol.

UAVCAN-based networks can accommodate up to 127 nodes on the same logical bus.
More nodes can be added to the system by separating the bus into several independent logical segments
interconnected via gateway nodes.

UAVCAN supports up to 32768 distinct message data types and up to 256 distinct service data types.
Part of those are reserved for the standard data types defined by the specification;
the rest are available for vendor- and application-specific data types.
More information is provided in the chapter \ref{sec:application_layer}.

UAVCAN supports eight distinct communication priority levels,
defined in the section \ref{sec:transfer_prioritization}.
Within each priority level, different types of transfers and different data types are
prioritized in a well-defined deterministic manner.

UAVCAN supports both CAN 2.0 and CAN FD transports.
CAN FD should be considered the primary transport, whereas CAN 2.0 is supported as legacy.
Non-redundant, doubly-redundant and triply-redundant transports are supported.
More information on the transport properties and standardized physical connectivity options
is provided in the chapter \ref{sec:hardware_design_recommendations}.

\section{Specification update and approval process}

The UAVCAN development team is charged with advancing the specification based on the input from adopters.
This feedback is gathered via the official discussion
forum\footnote{Please refer to \href{http://uavcan.org}{uavcan.org}.},
which is open to everyone.

The set of standard data type definitions is one of the cornerstone concepts of the specification
(the data structure description language (DSDL) and related concepts are described in section \ref{sec:dsdl}).
Within the same major version, the specification can be extended only in the following ways:

\begin{itemize}
    \item A new data type can be added, possibly with default data type ID,
    as long as the default data type ID doesn't conflict with one of the existing data types.

    \item An existing data type can be modified, as long as the modification doesn't break backward compatibility.

    \item A new version of an existing data type can be added.

    \item An existing data type can be declared deprecated.
    \begin{itemize}
        \item Once declared deprecated, the data type will be maintained for at least two more years.
        After this period its default data type ID may be reused for an incompatible data type.

        \item Deprecation will be announced via the discussion forum, and indicated in the form of a comment
        in its DSDL definition.
    \end{itemize}
\end{itemize}

A link to the repository containing the set of default DSDL definitions can be found on the official
website\footnote{\href{http://uavcan.org}{uavcan.org}}.

\section{Referenced sources}

The UAVCAN specification contains references to the following sources:

\begin{itemize}
    \item CiA 801 - Application note - Automatic bit rate detection.
    \item CiA 103 - Intrinsically safe capable physical layer.
    \item CiA 303 - Recommendation - Part 1: Cabling and connector pin assignment.
    \item IEEE 754 - Standard for binary floating-point arithmetic.
    \item ISO 11898-1 - Controller area network (CAN) - Part 1: Data link layer and physical signaling.
    \item ISO 11898-2 - Controller area network (CAN) - Part 2: High-speed medium access unit.
    \item ISO/IEC 10646 - Universal Coded Character Set (UCS).
    \item ISO/IEC 14882 - Programming Language C++.
    \item "Implementing a Distributed High-Resolution Real-Time Clock using the CAN-Bus", M. Gergeleit and H. Streich.
    \item "In Search of an Understandable Consensus Algorithm (Extended Version)", Diego Ongaro and John Ousterhout.
    \item \href{http://semver.org}{semver.org} - Semantic versioning specification.
\end{itemize}
