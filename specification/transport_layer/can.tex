\section{CAN bus transport layer specification}

This section specifies the CAN-based transport layer of UAVCAN.

Here and in the following parts of this section,
"CAN" implies both CAN 2.0 and CAN FD, unless specifically noted otherwise.
CAN FD should be considered the primary transport protocol.

UAVCAN utilizes only extended CAN frames with 29-bit identifiers.
UAVCAN can share the same bus with other protocols based on standard (non-extended) CAN frames with 11-bit identifiers.
However, future revisions of UAVCAN may utilize 11-bit identifiers as well;
therefore, backward compatibility with other protocols is not guaranteed.

\subsection{CAN ID structure}

UAVCAN utilizes two different CAN ID formats for message transfers and service transfers.
The structure is summarized on the figure~\ref{fig:can_id_structure}.

The fields are described in detail in the following sections.
The tables \ref{table:can_id_fields_message_transfer} and \ref{table:can_id_fields_service_transfer}
summarize the purpose of the fields and their permitted values
for message transfers and service transfers, respectively.

% Please do not remove the hard placement specifier [H], it is needed to keep elements ordered.
\begin{figure}[H]
    \centering
    \resizebox{\textwidth}{!}{
        \footnotesize
        \begin{tabular}{|l|c|c|c|c|c|c|c|c|c|c|c|c|c|c|c|c|c|c|c|c|c|c|c|c|c|c|c|c|c|} \hline
            %
            % Message transfer
            %
            \multirow{2}{*}{\textbf{Message}} &
            \multicolumn{4}{c|}{Service, not message} &
            \multicolumn{5}{c|}{Anonymous message} &
            \multicolumn{12}{c|}{\multirow{2}{*}{Subject ID}} &
            \multicolumn{8}{c|}{Protocol version}
            \\\cline{2-4} \cline{7-10} \cline{23-29}

            &
            \multicolumn{3}{c|}{Priority}
            &
            &
            &
            \multicolumn{16}{c|}{} &
            \multicolumn{7}{c|}{Source node ID} &
            \\

            \textbf{Values} &
            \multicolumn{3}{c|}{$[0, 7]$} &
            $0$ &
            $\mathbb{B}$ &
            \multicolumn{4}{c}{} &
            \multicolumn{12}{c|}{$[0, 32767]$} &
            \multicolumn{7}{c|}{$[0, 127]$} &
            $1$
            \\\hline

            \textbf{CAN ID bit} &
            28 & 27 & 26 & 25 & 24 & 23 & 22 & 21 & 20 & 19 & 18 & 17 & 16 & 15 &
            14 & 13 & 12 & 11 & 10 &  9 &  8 &  7 &  6 &  5 &  4 &  3 &  2 &  1 &  0
            \\\hline

            \textbf{CAN ID byte} &
            \multicolumn{5}{c|}{3} & \multicolumn{8}{c|}{2} & \multicolumn{8}{c|}{1} & \multicolumn{8}{c|}{0}
            \\\hline

            \multicolumn{30}{c}{} \\ \hline % Table separator

            %
            % Service transfer
            %
            \multirow{2}{*}{\textbf{Service}} &
            \multicolumn{4}{c|}{Service, not message} &
            \multicolumn{5}{c|}{Request, not response} &
            \multicolumn{5}{c|}{} &
            \multicolumn{7}{c|}{\multirow{2}{*}{Destination node ID}} &
            \multicolumn{8}{c|}{Protocol version}
            \\\cline{2-4} \cline{7-10} \cline{23-29}

            &
            \multicolumn{3}{c|}{Priority} &
            \cellcolor{red} &
            &
            \multicolumn{9}{c|}{Service ID} &
            \multicolumn{7}{c|}{} &
            \multicolumn{7}{c|}{Source node ID} &
            \\

            \textbf{Values} &
            \multicolumn{3}{c|}{$[0, 7]$} &
            $1$ &
            $\mathbb{B}$ &
            \multicolumn{9}{c|}{$[0, 511]$} &
            \multicolumn{7}{c|}{$[0, 127]$} &
            \multicolumn{7}{c|}{$[0, 127]$} &
            $1$
            \\\hline

            \textbf{CAN ID bit} &
            28 & 27 & 26 & 25 & 24 & 23 & 22 & 21 & 20 & 19 & 18 & 17 & 16 & 15 &
            14 & 13 & 12 & 11 & 10 &  9 &  8 &  7 &  6 &  5 &  4 &  3 &  2 &  1 &  0
            \\\hline

            \textbf{CAN ID byte} &
            \multicolumn{5}{c|}{3} & \multicolumn{8}{c|}{2} & \multicolumn{8}{c|}{1} & \multicolumn{8}{c|}{0}
            \\\hline
        \end{tabular}
    }
    \caption{CAN ID structure}\label{fig:can_id_structure}
\end{figure}

\begin{UAVCANSimpleTable}{CAN ID fields for message transfers}{|l l l X|}
    \label{table:can_id_fields_message_transfer}
    Field               & Width & Permitted values  & Description \\

    Transfer priority   & 3     & $[0, 7]$ (any)    & Section \ref{sec:transfer_prioritization}. \\

    Service not message & 1     & $0$               & Always zero for message transfers. \\

    Anonymous message   & 1     & $\{0, 1\}$ (any)  & Zero for regular (non-anonymous) message transfers.
                                                      One for anonymous message transfers. \\

    Subject ID          & 16    & $[0, 32767]$ (any) & Subject identifier of the current message.
                                                       The most significant bit is always zero. \\

    Source node ID      & 7     & $[0, 127]$ (any)  & Node ID of the origin.
                                                      For anonymous transfers, this field contains a pseudo-ID instead,
                                                      as described in section \ref{sec:can_source_node_pseudo_id}. \\

    Protocol version    & 1     & $1$               & Protocol version, ignore frame
                                                      (or use according to a newer specification)
                                                      if this field is not 1. \\
\end{UAVCANSimpleTable}

\begin{UAVCANSimpleTable}{CAN ID fields for service transfers}{|l l l X|}
    \label{table:can_id_fields_service_transfer}
    Field               & Width & Permitted values  & Description \\

    Transfer priority   & 3     & $[0, 7]$ (any)    & Section \ref{sec:transfer_prioritization}. \\

    Service not message & 1     & $1$               & Always one for service transfers. \\

    Request not response& 1     & $\{0, 1\}$ (any)  & 1 for service request, 0 for service response. \\

    Service ID          & 9     & $[0, 511]$ (any)  & Service ID of the encoded service data structure
                                                      (request or response). \\

    Destination node ID & 7     & $[0, 127]$ (any)  & Node ID of the destination
                                                      (i.e., server for requests, client for responses). \\

    Source node ID      & 7     & $[0, 127]$ (any)  & Node ID of the origin
                                                      (i.e., client for requests, server for responses). \\

    Protocol version    & 1     & $1$               & Protocol version, ignore frame
                                                      (or use according to a newer specification)
                                                      if this field is not 1. \\
\end{UAVCANSimpleTable}

\subsubsection{Transfer priority}

Valid values for priority range from 0 to 7, inclusively,
where 0 corresponds to the highest priority, and 7 corresponds to the lowest priority.

In multi-frame transfers, the value of the priority field must be identical for all frames of the transfer.

When multiple transfers of different types with the same priority contest for bus access
the following precedence is ensured, from higher priority to lower priority:

\begin{samepage}
\begin{enumerate}
    \item Message transfers.
    \item Anonymous message transfers.
    \item Service response transfers.
    \item Service request transfers.
\end{enumerate}
\end{samepage}

Message transfers take precedence over service transfers because message publication is the primary method of
communication in UAVCAN networks.
Service responses take precedence over service requests in order to make service invocations more atomic
and reduce the number of pending states in the system.

Within the same type and the same priority level,
transfers are prioritized according to the port ID\footnote{Subject ID or service ID.}:
transfers with lower port ID values preempt those with higher port ID values.

\begin{remark}
    Mnemonics for transfer priority levels are provided in section \ref{sec:transfer_prioritization},
    and their mapping to the priority field for CAN bus is shown in the table \ref{table:can_id_priority_field}.

    \begin{UAVCANSimpleTable}{CAN transfer priority level mapping}{|l X|}
        \label{table:can_id_priority_field}
        Priority field value & Mnemonic name \\
        0 & Exceptional \\
        1 & Immediate \\
        2 & Fast \\
        3 & High \\
        4 & Nominal \\
        5 & Low \\
        6 & Slow \\
        7 & Optional \\
    \end{UAVCANSimpleTable}
\end{remark}

\subsubsection{Source node ID field in anonymous message transfers}\label{sec:can_source_node_pseudo_id}

CAN bus does not allow different nodes to transmit CAN frames with different data field values under the same CAN ID.
Owing to the fact that the CAN ID includes the node ID value of the transmitting node,
this restriction does not affect regular UAVCAN transfers.
However, anonymous message transfers would violate this restriction,
because they don't have a unique node ID.

In order to work around this problem,
UAVCAN requires that the source node ID field of anonymous messages\footnote{Source node identifier
is not defined for anonymous message transfers; see table \ref{table:common_transfer_properties}.}
is initialized with a pseudorandom \emph{pseudo-ID value},
and defines special logic for handling CAN bus errors during transmission of anonymous frames.

The source of the pseudorandom data used for the pseudo-ID must be likely to produce different values
for different payloads\footnote{As explained in section \ref{sec:transport_anonymous_message_publication},
anonymous message transfers are always single-frame transfers.}.
A possible way of initializing the source node pseudo-ID value is to apply the transfer CRC function
(as defined in the section \ref{sec:transfer_crc})
to the payload of the anonymous message, and then use the seven least significant bits of the result.
Nodes that adopt this approach will be using the same pseudo-ID value for identical messages,
which is acceptable since this will not trigger an error on the bus.

Since the pseudo-ID is only seven bits long (128 possible values),
a collision where multiple nodes emit CAN frames with different data but same CAN ID is likely to happen
despite the randomization measures described earlier.
Therefore, the protocol must account for possible errors on the CAN bus triggered by CAN ID collisions.
In order to comply with this requirement,
UAVCAN requires all nodes to immediately abort transmission of all anonymous transfers once an error on
the CAN bus is detected\footnote{I.e., if the CAN controller uses automatic retransmission,
it should be disabled for anonymous frames.}.
This measure allows the protocol to prevent the bus deadlock that may occur if the automatic
retransmission on bus error is not suppressed.
Some method of media access control\footnote{%
    E.g., CSMA/CD (carrier-sense multiple access with collision detection).
    Additional bus access control logic is needed at the application level because
    the possibility of identifier collisions in anonymous frames undermines the access control logic implemented
    in CAN bus controller hardware.
}
should be used at the application level for further conflict resolution\footnote{%
    The described principles make anonymous messages highly non-deterministic and inefficient.
    This is considered acceptable because the scope of anonymous messages is limited to a very narrow set of use
    cases which tolerate their downsides. The UAVCAN specification employs anonymous messages only for the
    plug-and-play feature defined in section \ref{sec:application_level_functions}.
    Deterministic applications are advised to avoid reliance on anonymous messages completely.
}.

Nodes that receive anonymous transfers must ignore their source node identifiers.

\subsection{CAN frame data}

The CAN frame data field may contain the following segments, in the listed order:
\begin{samepage}
\begin{enumerate}
    \item The useful payload (serialized data structure). This segment may be empty.
    \item Possible padding bytes.
          Padding bytes may be necessary if the transport layer does not provide byte-level
          granularity of the data field length (e.g., CAN FD).
    \item The last frame of multi-frame transfers always contains the transfer CRC (section \ref{sec:transfer_crc}).
    \item The last byte of the data field always contains the \emph{tail byte}.
\end{enumerate}
\end{samepage}
The segments are documented below in this section.

\subsubsection{Tail byte}

UAVCAN adds one byte of overhead to every CAN frame irrespective of the type of the transfer.
The extra byte contains certain metadata for the needs of the transport layer.
It is named the \emph{tail byte}, and as the name suggests, it is always situated
at the very last byte of the data field of every CAN frame.
The tail byte contains four fields: \emph{start of transfer}, \emph{end of transfer},
\emph{toggle bit}, and the transfer ID (described earlier in the section \ref{sec:transfer_id}).
The placement of the fields and their usage for single-frame and multi-frame transfers
are documented in the table \ref{table:can_tail_byte}.

% Please do not remove the hard placement specifier [H], it is needed to keep tables ordered.
\begin{table}[H]\caption{Tail byte structure}\label{table:can_tail_byte}
\begin{tabu}{|c|l|X[2]|X[3]|}
    \hline
    \rowfont{\bfseries}
    Bit & Field & Single-frame transfers & Multi-frame transfers \\
    \hline

    7   & Start of transfer & Always 1  & First frame: 1, otherwise 0. \\\hline
    6   & End of transfer   & Always 1  & Last frame: 1, otherwise 0. \\\hline
    5   & Toggle bit        & Always 1  & First frame: 1, then alternates; section \ref{sec:toggle_bit}. \\\hline

    4   &               & \multicolumn{2}{c|}{} \\
    3   &               & \multicolumn{2}{c|}{Modulo 32 (range [0, 31])} \\
    2   & Transfer ID   & \multicolumn{2}{c|}{section \ref{sec:transfer_id}} \\
    1   &               & \multicolumn{2}{c|}{} \\
    0   &               & \multicolumn{2}{c|}{\footnotesize{(least significant bit)}} \\
    \hline
\end{tabu}
\end{table}

The transfer ID field is populated according to the specification provided in the section \ref{sec:transfer_id}.
The usage of this field is independent of the type of the transfer.

For single-frame transfers, the fields start-of-transfer, end-of-transfer, and the toggle bit
are all set to~1.

For multi-frame transfers, the fields start-of-transfer and end-of-transfer
are used to state the boundaries of the current transfer as described in the table.
The transfer ID value is identical for all frames of a multi-frame transfer.

The toggle bit, as described in the section \ref{sec:toggle_bit}, serves
two main purposes: CAN frame deduplication and protocol version detection.

\subsubsection{Padding bytes}

Certain transports (such as CAN FD) may not provide byte-level granularity of the CAN data field length.
In that case, the useful payload is to be padded with the minimal number of padding bytes required
to bring the total length of the CAN data field to a value that can satisfy the length granularity constraints.

When transmitting, each padding byte must be set to $\text{85} = \text{55}_\text{hex} = \text{0101\,0101}_\text{bin}$.
This specific padding value is chosen to avoid stuff bits and to facilitate CAN controller synchronization.

When receiving, the values of the padding bytes must be ignored.
In other words, receiving nodes must not make any assumptions about the values of the
padding bytes.

Usage of padding bytes implies that when a serialized message is being deserialized by a receiving node,
the byte sequence used for deserialization may be longer than the actual byte sequence generated by the
emitting node during serialization.
Therefore, nodes must ignore the trailing unused data bytes at the end of serialized byte sequences;
a length mismatch is only to be considered an error if the received byte sequence is shorter
than expected by the deserialization routine.

\subsubsection{Single-frame transfers}

For single-frame transfers, the data field of the CAN frame contains two or three segments:
the useful payload (which is the serialized data structure, may be empty), possible padding bytes,
and the tail byte (the last byte of the data field).

The resulting data field segmentation is shown in the table \ref{table:can_data_segments_single_frame}.

\begin{table}[H]\caption{CAN frame data segments for single-frame transfers}
\label{table:can_data_segments_single_frame}
\begin{tabu}{|l l X|}
    \hline
    \rowfont{\bfseries}
    Offset                  & Length                     & Segment \\\hline
    $0$                     & $L_\text{payload}\geq{}0$  & Useful payload (serialized data structure). \\\hline
    $L_\text{payload}$      & $L_\text{padding}\geq{}0$  & Padding bytes (if necessary). \\\hline
    $L_\text{payload} + L_\text{padding}$ & $1$          & Tail byte. \\\hline
\end{tabu}
\end{table}

\subsubsection{Multi-frame transfers}

For multi-frame transfers, all frames except the last one contain only a fragment
of the useful payload and the tail byte.
Notice that the padding bytes are not used in multi-frame transfers, excepting the last frame;
instead of padding, every frame except the last one must use the number of payload bytes that
satisfies the length granularity constraints.

The useful payload is fragmented in the forward order: the first CAN frame of a multi-frame transfer
contains the beginning of the payload (the first fragment),
the following frames contain the subsequent fragments of the useful payload.
The last CAN frame of a multi-frame transfer contains the last fragment, unless
the last fragment was fully accommodated by the second-to-last CAN frame of the transfer.
In the latter case, the last CAN frame will contain only the metadata,
as specified below in this section.

Each CAN frame of a multi-frame transfer except the last one
should use the maximum CAN data length permitted by the transport.
Observe that this is not a hard requirement;
some systems that utilize CAN FD may opt for shorter CAN frames in order to reduce the worst case
preemption latency, as explained in section \ref{sec:transport_multi_frame_transfers}.
Therefore, UAVCAN implementations must be able to correctly process incoming multi-frame transfers with
arbitrary CAN frame data lengths.

The resulting data field segmentation for all frames of a multi-frame transfer except the last one is
shown in the table \ref{table:can_data_segments_multi_frame_not_last}.

\begin{table}[H]\caption{CAN frame data segments for multi-frame transfers (except the last CAN frame of the transfer)}
\label{table:can_data_segments_multi_frame_not_last}
\begin{tabu}{|l l X|}
    \hline
    \rowfont{\bfseries}
    Offset                  & Length                & Segment \\\hline
    $0$                     & $L_\text{payload}>0$  & A fragment of the useful payload (serialized data structure).
                                                      This segment occupies the entirety of the CAN data field
                                                      except the last byte, which is used by the tail byte.
                                                      No padding is allowed. \\\hline
    $L_\text{payload}$      & $1$                   & Tail byte. \\\hline
\end{tabu}
\end{table}

The last CAN frame of a multi-frame transfer contains one or two additional segments:
the padding bytes (if necessary) and the transfer CRC.
The padding rules are identical to those of single-frame transfers.
The transfer CRC is to be allocated in the big-endian byte order\footnote{Most significant byte first.
This byte order is used to allow faster CRC residue checks; more info in section \ref{sec:transfer_crc}.}
immediately before the tail byte.
The resulting data field segmentation is shown in the table \ref{table:can_data_segments_multi_frame_last}.

\begin{table}[H]\caption{CAN frame data segments for multi-frame transfers (the last CAN frame of the transfer)}
\label{table:can_data_segments_multi_frame_last}
\begin{tabu}{|l l X|}
    \hline
    \rowfont{\bfseries}
    Offset                  & Length                     & Segment \\\hline
    $0$                     & $L_\text{payload}\geq{}0$  & The last fragment of the useful payload
                                                           (serialized data structure). \\\hline
    $L_\text{payload}$      & $L_\text{padding}\geq{}0$  & Padding bytes (if necessary). \\\hline

    \multirow{2}{*}{$L_\text{payload} + L_\text{padding}$} & \multirow{2}{*}{$2$} &
                                                           Transfer CRC, high byte.\\
                            &                            & Transfer CRC, low byte.\\\hline

    $L_\text{payload} + L_\text{padding} + 2$ & $1$        & Tail byte. \\\hline
\end{tabu}
\end{table}

\subsection{Software design considerations}

\subsubsection{Ordered transmission}

Multi-frame transfers use identical CAN ID for all frames of the transfer,
and UAVCAN requires that all frames of a multi-frame transfer should be transmitted in the correct order.
Therefore, the CAN controller driver software must ensure that CAN frames with identical CAN ID values
must be transmitted in their order of appearance in the transmission queue.
Some CAN controllers will not meet this requirement by default,
so the designer must take special care to ensure the correct behavior, and apply workarounds if necessary.

\subsubsection{Transmission timestamping}

Certain advanced features of UAVCAN may require the driver to timestamp outgoing transport frames, e.g.,
the time synchronization feature.

\begin{remark}[breakable]
    A sensible approach to transmission timestamping is built around the concept of \emph{loop-back frames},
    which is described here.

    If the application needs to timestamp an outgoing frame, it sets a special flag -- the \emph{loop-back flag} --
    on the frame before sending it to the driver.
    The driver would then automatically re-enqueue this frame back into the reception queue once it is transmitted
    (keeping the loop-back flag set so that the application is able to distinguish the loop-back
    frame from regular received traffic).
    The timestamp of the loop-backed frame would be of the moment when it was delivered to the bus.

    The advantage of the loop-back based approach is that it relies on the same interface between
    the application and the driver that is used for regular communications.
    No complex and dangerous callbacks or write-backs from interrupt handlers are involved.
\end{remark}

\subsubsection{Inner priority inversion}

Implementations should take necessary precautions against the problem of inner priority inversion.
The following non-normative section provides an overview of the inner priority inversion problem and
suggests a possible solution.

\begin{remark}[breakable]
    Suppose the application needs to emit a frame with the CAN ID $X$.
    The frame is submitted to the CAN controller's registers and the transmission is started.
    Suppose that afterwards it turned out that there is a new frame with the CAN ID $(X-1)$ that needs to be sent,
    too, but the previous frame $X$ is in the way, and it is blocking the transmission of the new frame.
    This may turn into a problem if the lower-priority frame is losing arbitration on the bus due
    to the traffic on the bus having higher priority than the current frame,
    but lower priority than the next frame that is waiting in the queue.

    A naive solution to this is to continuously check whether the priority of the frame that is currently being
    transmitted by the CAN controller is lower than the priority of the next frame in the queue, and if it is,
    abort transmission of the current frame, move it back to the transmission queue,
    and begin transmission of the new one instead.
    This approach, however, has a hidden race condition:
    the old frame may be aborted at the moment when it has already been received by remote nodes,
    which means that the next time it is re-transmitted, the remote nodes will see it duplicated.
    Additionally, this approach increases the complexity of the driver and can possibly affect
    its throughput and latency.

    Most CAN controllers offer a proper solution to the problem:
    they have multiple transmission mailboxes (usually at least 3),
    and the controller always chooses for transmission the mailbox which contains the highest priority frame.
    This provides the application with a possibility to avoid the inner priority inversion problem:
    whenever a new transmission is initiated, the application should check whether the priority of the next frame
    is higher than any of the other frames that are already awaiting transmission.
    If there is at least one higher-priority frame pending,
    the application doesn't move the new one to the controller's transmission mailboxes,
    it remains in the queue.
    Otherwise, if the new frame has a higher priority level than all of the pending frames,
    it is pushed to the controller's transmission mailboxes and removed from the queue.
    In the latter case, if a lower-priority frame loses arbitration,
    the controller would postpone its transmission and try transmitting the higher-priority one instead.
    That resolves the problem.

    There is an interesting extreme case, however.
    Imagine a controller equipped with $N$ transmission mailboxes.
    Suppose the application needs to emit $N$ frames in the increasing order of priority,
    which leads to all of the transmission mailboxes of the controller being occupied.
    Now, if all of the conditions below are satisfied, the system ends up with a priority inversion condition
    nevertheless, despite the measures described above:

    \begin{itemize}
        \item The highest-priority pending CAN frame cannot be transmitted due to the bus being saturated
        with a higher-priority traffic.
        \item The application needs to emit a new frame which has a higher priority than that which saturates the bus.
    \end{itemize}

    If both hold, a priority inversion is afoot because there is no free transmission mailbox to
    inject the new higher-priority frame into.
    The scenario is extremely unlikely, however;
    it is also possible to construct the application in a way that would preclude the problem,
    e.g., by limiting the number of simultaneously used distinct CAN ID values.

    The following pseudocode demonstrates the principles explained above:

    \begin{samepage}
    \begin{minted}{cpp}
    // Returns the index of the TX mailbox that can be used for the transmission of the newFrame
    // If none are available, returns -1.
    getFreeMailboxIndex(newFrame)
    {
        chosen_mailbox = -1     // By default, assume that no mailboxes are available

        for i = 0...NumberOfTxMailboxes
        {
            if isTxMailboxFree(i)
            {
                chosen_mailbox = i
                // Note: cannot break here, must check all other mailboxes as well.
            }
            else
            {
                if not isFramePriorityHigher(newFrame, getFrameFromTxMailbox(i))
                {
                    chosen_mailbox = -1
                    break   // Denied - must wait until this mailbox has finished transmitting
                }
            }
        }

        return chosen_mailbox
    }
    \end{minted}
    \end{samepage}
\end{remark}

\subsubsection{Automatic hardware acceptance filter configuration}

Most CAN controllers are equipped with hardware acceptance filters.
Hardware acceptance filters reduce the application workload by ignoring irrelevant CAN frames on the bus
by comparing their ID values against the set of relevant ID values configured by the application.

\begin{remark}[breakable]
    There exist two common approaches to CAN hardware filtering:
    list-based and mask-based.
    In the case of the list-based approach, every CAN frame detected on the bus is compared
    against the set of reference CAN ID values provided by the application;
    only those frames that are found in the reference set are accepted.
    Due to the complex structure of the CAN ID field used by UAVCAN,
    usage of the list-based filtering method with this protocol is impractical.

    Most CAN controller vendors implement mask-based filters,
    where the behavior of each filter is defined by two parameters: the mask $M$ and the reference ID $R$.
    Then, such filter accepts only those CAN frames for which the following bitwise logical condition holds
    true\footnote{Notation: $\land$ -- bitwise logical AND, $\oplus$ -- bitwise logical XOR,
    $\neg$ -- bitwise logical NOT.}:
    $$((X \land M) \oplus R) \leftrightarrow 0$$
    where $X$ is the CAN ID value of the evaluated frame.

    Complex UAVCAN applications are often required to operate with more distinct transfers than there are
    acceptance filters available in the hardware.
    That creates the challenge of finding the optimal configuration of the available filters that meets the
    following criteria:
    \begin{itemize}
        \item All CAN frames needed by the application are accepted.
        \item The number of irrelevant frames (i.e., not used by the application) accepted from the bus is minimized.
    \end{itemize}

    The optimal configuration is a function of the number of available hardware filters,
    the set of distinct transfers needed by the application,
    and the expected frequency of occurrence of all possible distinct transfers on the bus.
    The latter is important because if there are to be irrelevant transfers,
    it makes sense to optimize the configuration so that the acceptance of less common irrelevant transfers
    is preferred over the more common irrelevant transfers, as that reduces the processing load on the application.

    The optimal configuration depends on the properties of the network the node is connected to.
    In the absence of the information about the network,
    or if the properties of the network are expected to change frequently,
    it is possible to resort to a quasi-optimal configuration which assumes that
    the occurrence of all possible irrelevant transfers is equally probable.
    As such, the quasi-optimal configuration is a function of only the number of available hardware filters
    and the set of distinct transfers needed by the application.

    The quasi-optimal configuration can be easily found automatically.
    Certain implementations of the UAVCAN protocol stack include this functionality,
    allowing the application to easily adjust the configuration of the hardware acceptance filters
    using a very simple API.

    The quasi-optimal hardware acceptance filter configuration algorithm is defined below.

    First, the bitwise \emph{filter merge} operation is defined on filter configurations $A$ and $B$.
    The set of CAN frames accepted by the merged filter configuration is a superset of
    those accepted by $A$ and $B$.
    The definition is as follows:
    \begin{equation*}
    \begin{split}
        m_M(R_A, R_B, M_A, M_B) & = M_A \land M_B \land \neg (R_A \oplus R_B) \\
        m_R(R_A, R_B, M_A, M_B) & = R_A \land m_M(R_A, R_B, M_A, M_B)
    \end{split}
    \end{equation*}

    The \emph{filter rank} is a function of the mask of the filter.
    The rank of a filter is a unitless quantity that defines in relative terms how selective the filter
    configuration is.
    The rank of a filter is proportional to the likelihood that the filter will reject a random CAN ID.
    In the context of hardware filtering, this quantity is conveniently representable via the number of bits set in
    the filter mask parameter:
    \begin{equation*}
    r(M) =
    \begin{cases}
        0                                   & \quad M < 1 \\
        r(\lfloor\frac{M}{2}\rfloor)        & \quad M \bmod 2 = 0 \\
        r(\lfloor\frac{M}{2}\rfloor) + 1    & \quad M \bmod 2 \neq 0 \\
    \end{cases}
    \end{equation*}

    Having the low-level operations defined, we can proceed to define the whole algorithm.
    First, construct the initial set of CAN acceptance filter configurations
    according to the requirements of the application.
    Then, as long as the number of configurations in the set exceeds the number of available
    hardware acceptance filters, repeat the following:
    \begin{enumerate}
        \item Find the pair $A$, $B$ of configurations in the set for which $r(m_M(R_A, R_B, M_A, M_B))$ is maximized.
        \item Remove $A$ and $B$ from the set of configurations.
        \item Add a new configuration $X$ to the set of configurations, where
        $X_M = m_M(R_A, R_B, M_A, M_B)$, and $X_R = m_R(R_A, R_B, M_A, M_B)$.
    \end{enumerate}

    The algorithm reduces the number of filter configurations by one at each iteration,
    until the number of available hardware filters is sufficient to accommodate the whole set of configurations.
\end{remark}
