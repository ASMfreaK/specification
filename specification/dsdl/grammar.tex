\section{Grammar}

\subsection{File structure}

DSDL uses a simple regular grammar which can be parsed using standard POSIX regular
expressions\footnote{Regular grammars are easier to parse than context-free grammars,
which are commonly used in other machine languages. The set of regular grammars is a subset of context-free grammars.}.
The following grammar and syntax specification rely on the standard IEEE POSIX ERE\footnote{IEEE Std 1003.1.}
regular expression notation.

A DSDL definition file is modeled as a sequence of \emph{lines},
where each line is terminated with any of the following ASCII character sequences:

\begin{itemize}
    \item line feed ``\texttt{\textbackslash{}n}'' (ASCII code 10);
    \item carriage return ``\texttt{\textbackslash{}r}'' (ASCII code 13);
    \item line feed, then carriage return ``\texttt{\textbackslash{}r\textbackslash{}n}''
    (ASCII codes 13 and 10, respectively).
\end{itemize}

Lines may contain \emph{horizontal whitespace characters}, which are used for token separation purposes.
The set of horizontal whitespace characters contains the following ASCII code points:

\begin{itemize}
    \item space ``\texttt{ }'' (ASCII code 32);
    \item horizontal tabulation ``\texttt{\textbackslash{}t}'' (ASCII code 9).
\end{itemize}

DSDL lines may contain comments for the benefit of human readers.
A comment is delimited by a number sign ``\verb|#|'' (ASCII code 35) on the left side
and the end of its line on the right side.
Comments are ignored by DSDL processors, and as such,
they are allowed to contain any Unicode characters except for the line termination characters defined earlier.

Each line may contain at most one \emph{statement} situated immediately at its beginning (leftmost side),
which may be followed by an arbitrary number of horizontal whitespace characters,
which in turn may be followed by a comment.
Lines that do not contain a statement are called \emph{empty lines},
regardless of whether they contain any horizontal whitespace characters or a comment.

There are three kinds of statements:

\begin{description}
    \item[Attribute declaration] --- used to declare an \emph{attribute}; section \ref{sec:dsdl_attributes}.

    \item[Directive] --- used to specify a \emph{directive}; section \ref{sec:dsdl_directives}.

    \item[Service response marker] --- used to denote that the current definition is of a service type,
    and also to separate the request data structure from the response data structure;
    section \ref{sec:dsdl_service_response_marker}.
\end{description}

\subsection{Attributes}\label{sec:dsdl_attributes}

An attribute specifies a high-level data entity pertaining to its data type.
All possible kinds of attributes are defined in this section.

\subsubsection{Fields}

A \emph{field} defines a data entity the value of which is set by an application at runtime.
A collection of fields forms the body of the data structure which is then exchanged over the network in its
serialized form.

\subsubsection{Constants}

A \emph{constant} defines a data entity the value of which is set by the data type designer
when the data type is defined.
Constants are never exchanged over the network.
The main purpose of the constant is to define auxiliary data items for the benefit of the application.

\subsubsection{Padding fields}

A \emph{padding field} defines a data entity which is ignored by the application.
Padding fields are used to manually align other data fields or to reserve space for use in future versions of the
data type.

\subsection{Directives}\label{sec:dsdl_directives}

\subsubsection{Union specifier}

\subsubsection{Deprecation specifier}

\subsubsection{Assertion check directive}

\subsubsection{Diagnostic output directive}

\subsection{Service response marker}\label{sec:dsdl_service_response_marker}

\subsection{Literals}\label{sec:dsdl_literals}
