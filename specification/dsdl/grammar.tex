\section{Grammar}

\subsection{Formal definition}

This section contains the formal definition of the DSDL grammar.
Its notation is introduced beforehand.

\subsubsection{Notation}

The notation used in the following definition is a modified ISO/IEC 14977 extended Backus--Naur
form\footnote{This notation is a subset of the notation defined in a Python parsing library titled Parsimonious.
Parsimonious is an MIT-licensed software product authored by Erik Rose;
its sources are available at \url{https://github.com/erikrose/parsimonious}.}.
The rule definition patterns are specified in the table \ref{table:dsdl_grammar_definition_notation}.

\begin{table}[hbtp]
    \begin{UAVCANSimpleTable}{Notation used in the formal grammar definition}{|l X|}
        \label{table:dsdl_grammar_definition_notation}
        Pattern & Description \\

        \texttt{"korovan"} &
        Denotes a terminal string of ASCII characters.
        The string is case-sensitive. \\

        \emph{(space)} &
        Concatenation.
        E.g., \texttt{korovan paukan excavator} matches a sequence where the specified tokens
        appear in the defined order. \\

        \texttt{korovan / paukan / excavator} &
        Alternatives.
        The leftmost matching alternative is accepted. \\

        \texttt{korovan?} &
        Optional greedy match. \\

        \texttt{paukan*} &
        Zero or more expressions, greedy match. \\

        \texttt{excavator+} &
        One or more expressions, greedy match. \\

        \texttt{\textasciitilde{}r"regex-pattern"} &
        An IEEE POSIX Extended Regular Expression pattern defined between the double quotes.
        The expression operates on the ASCII character set and is always case-sensitive.
        ASCII escape sequences ``\texttt{\textbackslash{}r}'', ``\texttt{\textbackslash{}n}'', and
        ``\texttt{\textbackslash{}t}'' are used to denote ASCII carriage return (code 13),
        line feed (code 10), and tabulation (code 9) characters, respectively. \\

        \texttt{\textasciitilde{}r'regex-pattern'} &
        As above, with single quotes instead of double quotes. \\

        \texttt{(korovan paukan)} &
        Parentheses are used for grouping. \\
    \end{UAVCANSimpleTable}
\end{table}

\subsubsection{Definition}

From the top level down to the expression rule, the grammar is a valid regular grammar,
meaning that it can be parsed using standard regular expressions.
The expression syntax is modeled after that of the Python programming language.

\clearpage\inputminted[fontsize=\scriptsize]{python}{dsdl/grammar.parsimonious}
