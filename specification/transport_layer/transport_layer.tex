\chapter{Transport layer}\label{sec:transport_layer}

This chapter defines the transport layer of UAVCAN.
First, general implementation-agnostic concepts are introduced.
Afterwards, they are further defined for each supported transport medium, e.g., CAN FD.

As the specification is extended to add support for new transport protocols,
some of the generic aspects may be pushed to lower-level transport-specific sections
if they are found to map poorly on the newly added transports.
Such changes are guaranteed to preserve full backward compatibility of the existing transport protocols.

\section{Core concepts}

\subsection{Transfer}

A \emph{transfer} is an act of data transmission between nodes.

\subsubsection{Broadcast and unicast transfers}

A transfer that is addressed to any interested node except the source node is a \emph{broadcast transfer}.
A transfer that is addressed to one particular node is a \emph{unicast transfer}.

In the case of broadcast transfers, the sending node makes the data widely available on the bus,
allowing any interested node to freely opt-in and process
it\footnote{The word ``broadcast'' should not lead one to believe that every node is required to
process such transfers. The opt-in logic is facilitated by automatic acceptance filtering features
implemented on the transport layer.}.
The decision of whether to process any given transfer or not is made by receiving nodes.

In the case of unicast transfers, the addressing logic is inverted:
the sending node decides which particular remote node should receive the transfer.
All other nodes remain unaffected by such transmission and take no part in the addressing process.

\subsubsection{Message and service transfers}

A \emph{message transfer} is a broadcast transfer that contains a serialized message and its
metadata\footnote{Such as the subject-ID and the source node-ID.}.

A \emph{service transfer} is a unicast transfer that contains either a service request or a service response
with related metadata.

\subsubsection{Single-frame and multi-frame transfers}

Both message and service transfers can be further distinguished between single-frame and multi-frame transfers.

A \emph{single-frame transfer} is a transfer that is entirely contained in a single transport frame.
The amount of data that can be exchanged using single-frame transfers is dependent on the transport protocol in use.

A \emph{multi-frame transfer} is a transfer that has its payload distributed over multiple transport frames.
The UAVCAN protocol stack handles transfer decomposition and reassembly automatically.

The choice between single-frame and multi-frame transfers is made by the UAVCAN protocol logic on
the transmitting node based on the amount of payload data to be transferred.
The application does not have any control over the type of transfer that will be used
except limiting the amount of payload data.
UAVCAN protocol implementations must always choose single-frame transfers if possible;
multi-frame transfers can be used only if all of the requested payload cannot be allocated in one transport frame.

\subsubsection{Common properties}

The properties listed in the table \ref{table:common_transfer_properties} are common to all types of transfers.

\begin{UAVCANSimpleTable}{Common transfer properties}{|l X|}\label{table:common_transfer_properties}
    Property        & Description \\
    Payload         & The serialized object. \\
    Port-ID         & A numerical identifier that indicates how the data should be processed.
                      This is the subject-ID for message transfers and service-ID for service transfers. \\
    Source node-ID  & The node-ID of the transmitting node (excepting anonymous message transfers). \\
    Priority        & A non-negative integer value that defines the transfer urgency.
                      Higher priority transfers can preempt lower priority transfers. \\
    Transfer-ID     & A small overflowing integer that increments with every transfer
                      of this data type from a given node. \\
\end{UAVCANSimpleTable}

\subsection{Message publication}

Message publication is the main method of communication between UAVCAN nodes.

A published message is carried by a single message transfer that contains the serialized message object.
A published message does not contain any additional fields besides those listed in the table
\ref{table:common_transfer_properties}.

In order to publish a message, the publishing node must have a node-ID that is unique within the network.
An exception applies to \emph{anonymous message publications}.

\subsubsection{Anonymous message publication}\label{sec:transport_anonymous_message_publication}

An anonymous message transfer is a transfer that can be sent from a node that does not have a node-ID.
This kind of message transfer is especially useful for facilitation of \emph{plug-and-play nodes}
(a high-level concept that is reviewed in detail in chapter \ref{sec:application_layer}).

A node that does not have a node-ID is said to be in \emph{passive mode}.
Passive nodes are unable to initiate regular data exchanges,
but they can listen to the transfers exchanged over the bus,
and they can emit anonymous message transfers.

An anonymous message has the same properties as a regular message, except for the source node-ID.

An anonymous transfer can only be a single-frame transfer. Multi-frame anonymous message transfers are not allowed.
This restriction must be kept in mind when designing message data types
intended for use with anonymous message transfers:
when used with anonymous transfers, the whole message must fit into a single transport frame;
however, the same data type can be used with multi-frame regular (non-anonymous) transfers, if desired.

Anonymous messages may require special handling logic depending on the transport layer in use.

\subsubsection{Message timing requirements}

Generally, a message transmission should be aborted if it cannot be completed in 1 second.
Applications are allowed to deviate from this recommendation,
provided that every such deviation is explicitly documented.
It is expected that high-frequency high-priority messages may opt for lower timeout values,
whereas low-priority delayable data may opt for higher timeout values to account for network congestion.

\subsection{Service invocation}

A service invocation sequence consists of two related service transfers:
\emph{service request transfer} and \emph{service response transfer}.

A service request transfer is sent from the invoking node -- \emph{client node} -- to the node
that provides the service -- \emph{server node}.
Upon handling the request, the server node responds to the client node with a service response transfer.
The client will match the response with the corresponding request by comparing the following values:
server node-ID, service-ID, and the transfer-ID.

The tables \ref{table:service_request_transfer_properties} and \ref{table:service_response_transfer_properties}
describe the properties of service request and service response transfers, respectively.

Both the client and the server must have node-ID values that are unique within the network;
service invocation is not available to passive nodes.
The client and the server must be two distinct nodes.

\begin{UAVCANSimpleTable}{Service request transfer properties}{|l X|}\label{table:service_request_transfer_properties}
    Property                        & Description \\
    Payload                         & The serialized service request object. \\
    Service-ID                      & See the table \ref{table:common_transfer_properties}. \\
    Source node-ID                  & The node-ID of the client (the invoking node). \\
    Destination node-ID             & The node-ID of the server (the invoked node). \\
    Priority                        & See the table \ref{table:common_transfer_properties}. \\
    Transfer-ID                     & An integer value that:
        \begin{enumerate}
            \item allows the server to distinguish the request from other requests from the same client;
            \item allows the client to match the response with its request.
        \end{enumerate} \\
\end{UAVCANSimpleTable}

\begin{UAVCANSimpleTable}{Service response transfer properties}{|l X|}\label{table:service_response_transfer_properties}
    Property                        & Description \\
    Payload                         & The serialized service response object. \\
    Service-ID                      & Same value as in the request transfer. \\
    Source node-ID                  & The node-ID of the server (the invoked node). \\
    Destination node-ID             & The node-ID of the client (the invoking node). \\
    Priority                        & Same value as in the request transfer. \\
    Transfer-ID                     & Same value as in the request transfer. \\
\end{UAVCANSimpleTable}

\subsubsection{Service timing requirements}

Applications are recommended to follow the service invocation timing recommendations specified below.
Applications are allowed to deviate from these recommendations,
provided that every such deviation is explicitly documented.

\begin{itemize}
    \item Service transfer transmission should be aborted if does not complete in 1 second.
    \item The client should stop waiting for a response from the server if one has not arrived within 1 second.
\end{itemize}

If the server uses a significant part of the timeout period to process the request,
the client might drop the request before receiving the response.
It is recommended to ensure that the server will be able to process any request in less than 0.5 seconds.

\subsection{Transfer priority}\label{sec:transfer_prioritization}

UAVCAN transfers are prioritized by means of the transfer priority property,
which allows at least 8 (eight) different priority levels for all types of transfers
(some transports may support more than eight priority levels).
Transfers with higher priority levels preempt transfers with lower priority levels,
delaying their transmission until there are no more higher priority transfers to exchange.

\begin{remark}[breakable]
    The priority level mnemonics and their usage recommendations are specified in the following list.
    The mapping between the mnemonics and actual numeric identifiers is transport-dependent.

    % https://forum.uavcan.org/t/transfer-priority-level-mnemonics/218/6?u=pavel.kirienko
    \begin{description}
        \item[Exceptional] -- The bus designer can ignore these messages when calculating bus load since they
        should only be sent when a total system failure has occurred.
        For example, a self-destruct message on a rocket would use this priority.
        Another analogy is an NMI on a microcontroller.

        \item[Immediate] -- Immediate is a ``high priority message'' but with additional latency constraints.
        Since exceptional messages are not considered when designing a bus, the latency of immediate messages
        can be determined by considering only immediate messages.

        \item[Fast] -- Fast and immediate are both ``high priority messages'' but with additional latency constraints.
        Since exceptional messages are not considered when designing a bus,
        the latency of fast messages can be determined by considering only immediate and fast messages.

        \item[High] -- High priority messages are more important than nominal messages but have looser
        latency requirements than fast messages. This priority is used so that,
        in the presence of rogue nominal messages, important commands can be received.
        For example, one might envision a failure mode where a temperature sensor starts to
        load a vehicle bus with nominal messages.
        The vehicle remains operational (for a time) because the controller is exchanging fast and
        immediate messages with sensors and actuators.
        A system safety monitor is able to detect the distressed bus and command the vehicle to a
        safe state by sending high priority messages to the controller.

        \item[Nominal] -- This is what all messages should use by default.
        Specifically the heartbeat messages should use this priority.

        \item[Low] -- Low priority messages are expected to be sent on a bus under all conditions but cannot
        prevent the delivery of nominal messages.
        They are allowed to be delayed but latency should be constrained by the bus designer.

        \item[Slow] -- Slow messages are low priority messages that have no time sensitivity at all.
        The bus designer need only ensure that, for all possible system states,
        these messages will eventually be sent.

        \item[Optional] -- These messages might never be sent (theoretically) for some possible system states.
        The system must tolerate never exchanging optional messages in every possible state.
        The bus designer can ignore these messages when calculating bus load.
        This should be the priority used for diagnostic or debug messages that are not required on an
        operational system.
    \end{description}
\end{remark}

\subsection{Transfer descriptor}\label{sec:transfer_descriptor}

Transfer emission and reception processes rely on the concept of \emph{transfer descriptor}.

A transfer descriptor is a set of properties that identify a particular set of transfers that originate
from the same source node, share the same port-ID, same kind (message or service), and are addressed to the same
destination node (the latter applies only to unicast transfers).

The properties that constitute a transfer descriptor are listed below:

\begin{itemize}
    \item Transfer kind (message or service).
    \item Port-ID (subject-ID for message transfers, service-ID for service transfers).
    \item Source node-ID.
    \item Destination node-ID (only for service transfers).
\end{itemize}

For convenience, two derived definitions are introduced.
Their objective is to simplify the description of transfer reception and emission logic that appears later in this
specification.
\begin{description}
    \item[Emitted transfer descriptor] -- a transfer descriptor where the source node-ID equals the local node's ID.
    \item[Received transfer descriptor] -- a transfer descriptor where the destination node-ID equals
    the local node's ID (for service transfers) or is not defined (for message transfers).
\end{description}

\subsubsection{Hard real-time considerations}

Hard real-time applications require a predictable and deterministic data processing time.
The concept of transfer descriptor plays an important role in communication;
hence, its contribution to the worst case data processing load should be carefully analyzed.

\begin{remark}
    From the above definition of transfer descriptor it is easy to derive that for any
    message subject-ID or any service subject-ID the maximum number of transfer descriptors
    that can be observed by the local node will never exceed the number of nodes on the bus minus
    one\footnote{The local node cannot exchange data with itself, hence minus one.}.
    If the number of nodes on the bus cannot be known in advance, it can be considered to equal the maximum
    number of nodes permitted by the used transport layer\footnote{E.g., 128 nodes for the CAN bus transport.}.

    The total number of distinct transfer descriptors that can be observed by a node on any valid UAVCAN bus
    is a product of the number of distinct port-ID values utilized by the node and the number of other nodes on the bus.

    The transport emission and reception logic defined later in this specification relies on data structures
    indexed by transfer descriptor values.
    Elements of such structures can be easily accessed via constant-complexity static look-up tables
    because the worst case number of elements is always statically known.
\end{remark}

\section{Transfer emission}

\subsection{Transfer-ID computation}\label{sec:transfer_id}

The \emph{transfer-ID} is a small unsigned integer value in the range from 0 to 31, inclusive,
that is provided for every transfer.
This value is crucial for many aspects of UAVCAN communication\footnote{One might be tempted to use the transfer-ID
value for temporal synchronization of parallel message streams originating from the same node,
where messages bearing the same transfer-ID value are supposed to correspond to the same moment in time.
Such use is strongly discouraged because it is impossible to detect if one node is more than
32 messages behind another.
If temporal synchronization is necessary, explicit time stamping should be used instead.};
specifically:
\begin{description}
    \item[Message sequence monitoring] - the continuously increasing transfer-ID allows receiving nodes to
    detect lost messages and detect when a message stream from any remote node is interrupted.

    \item[Service response matching] - when a server responds to a request, it uses the same transfer-ID for the
    response as in the request,
    allowing any node to emit concurrent requests to the same server while being able to
    match each response with the corresponding request.

    \item[Transport frame deduplication] - for single-frame transfers,
    the transfer-ID allows receiving nodes to work around the transport
    frame duplication problem\footnote{This is a well-known issue that can be observed with certain
    transports such as CAN bus -- a frame that appears valid to the receiver may under certain
    (rare) conditions appear invalid to the transmitter, triggering the latter to retransmit the frame,
    in which case it will be duplicated on the side of the receiver.
    Sequence counting mechanisms such as the transfer-ID or the toggle bit (both of which are used in UAVCAN)
    allow applications to circumvent this problem.} (multi-frame transfers combat the frame duplication
    problem using the toggle bit, which is introduced later).

    \item[Multi-frame transfer reassembly] - more info is provided in section \ref{sec:transfer_reception}.

    \item[Automatic management of redundant interfaces] - the transfer-ID parameter allows the UAVCAN protocol
    stack to perform automatic switchover to a back-up interface shall the primary interface fail.
    The switchover logic can be completely transparent to the application, joining several independent
    redundant physical transports into a highly reliable single virtual communication channel.
\end{description}

For message transfers and service request transfers the ID value should be computed as described below.
For service response transfers this value must be directly copied from the corresponding service request transfer.

Every node that is interested in emitting transfers must maintain a mapping
(or a similar functionally equivalent static structure\footnote{For example, simple static variables.})
from emitted transfer descriptors (section \ref{sec:transfer_descriptor}) to transfer-ID counters.
This mapping is referred to as the \emph{emitted transfer-ID map}.

Whenever a node needs to emit a transfer, it will query its transfer-ID map for the appropriate transfer descriptor.
If the map does not contain such entry, a new entry will be created with the transfer-ID counter initialized to zero.
The node will use the current value of the transfer-ID from the map for the transfer,
and then the value stored in the map will be incremented by one.
When the stored transfer-ID exceeds its maximum value, it will roll over to zero.

It is expected that some nodes will need to emit certain transfers aperiodically or on an ad-hoc basis,
thereby creating unused entries in the emitted transfer-ID map.
If such aperiodic or ad-hoc transfers are of interest,
the worst case number of unused entries can be determined statically as a function of the number of
port identifiers used and the number of addressed nodes on the bus (the latter applies to services only).
Nodes are not allowed to remove any entries from the transfer-ID map as long as they are running.

\subsection{Single frame transfers}

If the size of the entire transfer payload does not exceed the space available for payload in a single transport frame,
the whole transfer will be contained in one transport frame.
Such transfer is called a \emph{single-frame transfer}.

Single frame transfers are more efficient than multi-frame transfers in terms of throughput, latency,
and data overhead.

\subsection{Multi-frame transfers}\label{sec:transport_multi_frame_transfers}

\emph{Multi-frame transfers} are used when the size of the transfer payload exceeds the space available
for payload in a single transport frame.

Two new concepts are introduced in the context of multi-frame transfers, both of which are reviewed below in detail:
\begin{samepage}
\begin{itemize}
    \item Transfer CRC\footnote{CRC stands for ``cyclic redundancy check'', an error-detecting code
    added to data transmissions to reduce the likelihood of undetected data corruption.}.
    \item Toggle bit.
\end{itemize}
\end{samepage}

In order to emit a multi-frame transfer, the node must first compute the CRC for the entirety of the transfer payload.
The node appends the resulting CRC value at the end of the transfer payload in the big-endian byte order,
and then emits the resulting byte set in chunks as an ordered sequence of transport frames,
where the first transport frame contains the beginning of the payload bytes,
and the last transport frame contains the last bytes of the payload (possibly none) plus the transfer CRC.

The data field of all transport frames of a multi-frame transfer, except the last one, should be fully utilized.
Applications are allowed to limit the maximum amount of data transferred per transport frame in order to
improve the preemption granularity, thus reducing the worst case latency of higher priority
transfers\footnote{For example, some CAN FD applications may choose to restrict the maximum payload size to 32 bytes
rather than the protocol limit of 64 bytes, as that provides more opportunities for higher-priority frames to
take over the bus. The trade-off is that smaller frames lead to higher transfer fragmentation, increase the bus load,
and increase the overall average latency.}.
Receiving nodes must be prepared to reconstruct multi-frame transfers that utilize the
available payload space partially.

All frames of a multi-frame transfer should be pushed to the transmission queue at once,
in the proper order from the first frame to the last frame.
Explicit gap time between transport frames belonging to the same transfer should not be introduced;
rather, implementations always should strive to minimize it.
Re-ordering of frames belonging to the same multi-frame transfer is prohibited.

\subsubsection{Transfer CRC}\label{sec:transfer_crc}

Transfer CRC allows receiving nodes to ensure that a received multi-frame transfer has been reassembled correctly.

It should be understood that the transfer CRC is not intended for bit-level data integrity checks,
as that must be managed by the transport layer implementation on a per-frame
basis\footnote{Bit-level errors at the transport frame level may compromise the error-detecting
properties of the transfer CRC.}.
As such, the transfer CRC allows receiving nodes to ensure that all of the frames of a multi-frame
transfer were received, all of the received frames were reassembled in the correct order,
and that all of the received frames belong to the same multi-frame transfer.

The transfer CRC is computed over the entire payload of the transfer.
Certain transport implementations\footnote{Such as CAN FD.} may require a short sequence of padding bytes
to be added at the end of the transfer payload due to the low granularity of the frame payload length property;
in that case, the padding bytes must be included in the CRC computation as well,
as if they were part of the useful payload.

The resulting CRC value is appended to the transfer in the \emph{big-endian byte order}
(most significant byte first),
in order to take advantage of the CRC residue check intrinsic to the used algorithm.

The transfer CRC algorithm specification is provided in the table \ref{table:transfer_crc_params}.

\begin{minipage}{0.7\textwidth}
\begin{UAVCANSimpleTable}{Transfer CRC algorithm parameters}{|ll|}\label{table:transfer_crc_params}
    Property        & Value \\
    Name            & CRC-16/CCITT-FALSE \\
    Initial value   & $\mathrm{FFFF}_{16}$ \\
    Polynomial      & $\mathrm{1021}_{16}$ \\
    Reverse         & No \\
    Output XOR      & $0$ \\
    Residue         & $0$ \\
    Check           & $\left(49, 50, \ldots, 56, 57\right) \rightarrow \mathrm{29B1}_{16}$ \\
\end{UAVCANSimpleTable}
\end{minipage}

The following code snippet provides a basic implementation of the transfer CRC algorithm in C++.

\begin{minipage}{0.9\textwidth}
\begin{minted}{cpp}
// UAVCAN transfer CRC algorithm implementation in C++.
// License: CC0, no copyright reserved.

#include <iostream>
#include <cstdint>
#include <cstddef>

class TransferCRC
{
    std::uint16_t value_ = 0xFFFFU;

public:
    void add(std::uint8_t byte)
    {
        value_ ^= static_cast<std::uint16_t>(byte) << 8U;
        for (std::uint8_t bit = 8; bit > 0; --bit)
        {
            if ((value_ & 0x8000U) != 0)
            {
                value_ = (value_ << 1U) ^ 0x1021U;
            }
            else
            {
                value_ = value_ << 1U;
            }
        }
    }

    void add(const std::uint8_t* bytes, std::size_t length)
    {
        while (length-- > 0)
        {
            add(*bytes++);
        }
    }

    [[nodiscard]] std::uint16_t get() const { return value_; }
};

int main()
{
    TransferCRC crc;
    crc.add(reinterpret_cast<const std::uint8_t*>("123456789"), 9);
    std::cout << std::hex << "0x" << crc.get() << std::endl;  // Outputs 0x29B1
    return 0;
}
\end{minted}
\end{minipage}

\subsubsection{Toggle bit}\label{sec:toggle_bit}

The toggle bit is a property defined at the transport frame level.
Its purpose is to detect and avoid transport frame duplication errors in multi-frame
transfers\footnote{In single-frame transfers, transport frame deduplication is based on the transfer-ID counter.}.

The toggle bit of the first transport frame of a multi-frame transfer must be set to one.
The toggle bits of the following transport frames of the transfer must alternate,
i.e., the toggle bit of the second transport frame must be zero,
the toggle bit of the third transport frame must be one, and so on.

For single-frame transfers, the toggle bit must be set to one or removed completely,
whichever option works best for the particular transport.

Transfers where the initial value of the toggle bit is zero must be ignored.
The initial state of the toggle bit may be inverted in the future revisions of the protocol
to facilitate automatic protocol version detection.

\subsection{Redundant interface support}

In configurations with redundant bus interfaces,
nodes are required to submit every outgoing transfer to the transmission queues of
all available redundant interfaces simultaneously.
It is recognized that perfectly simultaneous transmission may not be possible due to different
utilization rates of the redundant interfaces and different phasing of their traffic;
however, that is not an issue for UAVCAN.
If perfectly simultaneous frame submission is not possible, interfaces with lower numerical index
should be handled in the first order.

An exception to the above rule applies if the payload of the transfer depends on some properties
of the interface through which the transfer is emitted.
An example of such a special case is the time synchronization algorithm leveraged by UAVCAN
(documented in chapter \ref{sec:application_layer} of the specification).

Redundant interfaces are used for increased fault tolerance, not for load sharing reasons.
Whenever a node is connected to an interface the likelihood of the interface failing is increased.
This suggests that backup interfaces may only interconnect with mission-critical equipment,
unless a homogeneous network architecture is desired\footnote{Heterogeneous transport configuration
complicates the analysis of the network, which might make it impractical in safety-critical deployments.
In that case, a simpler configuration where each available redundant bus is connected to every node may be
preferred.}.
See section \ref{sec:phy_non_uniform_transport_redundancy}.

\section{Transfer reception}\label{sec:transfer_reception}

\subsection{Transfer-ID comparison}\label{sec:transfer_id_forward_distance}

The following explanation relies on the concept of the \emph{transfer-ID forward distance}.
Transfer-ID forward distance $F$ is a function of two transfer-ID values,
$A$ and $B$, that defines the number of increment operations that need to be applied to
$A$ so that $A^\prime{} = B$, assuming modulo 32 arithmetic\footnote{%
    For example:
    $A=0, B=0, F\rightarrow0$;
    $A=0, B=5, F\rightarrow5$;
    $A=5, B=0, F\rightarrow27$;
    $A=31, B=30, F\rightarrow31$;
    $A=31, B=0, F\rightarrow1$.
}:
$$A + F = B \quad (\bmod{}\ 32)$$
The \emph{half range} of transfer-ID is 16.

The following code sample provides an example implementation of the transfer-ID comparison algorithm in C++.

\begin{minipage}{0.9\textwidth}  % Mini page is needed to prevent page breaks within the snippet
\begin{minted}{cpp}
// UAVCAN transfer-ID forward distance computation algorithm implemented in C++.
// License: CC0, no copyright reserved.

#include <cstdint>
#include <iostream>
#include <cassert>

constexpr std::uint8_t TransferIDBitLength = 5;  // Defined by the specification

[[nodiscard]]
constexpr std::uint8_t computeForwardDistance(std::uint8_t a, std::uint8_t b)
{
    constexpr std::uint8_t MaxValue = (1U << TransferIDBitLength) - 1U;
    assert((a <= MaxValue) && (b <= MaxValue));

    std::int16_t d = static_cast<std::int16_t>(b) - static_cast<std::int16_t>(a);
    if (d < 0)
    {
        d += 1U << TransferIDBitLength;
    }

    assert(d >= 0);
    assert(d <= MaxValue);
    assert(((a + d) & MaxValue) == b);
    return static_cast<std::uint8_t>(d);
}

int main()
{
    assert(0  == computeForwardDistance(0, 0));
    assert(1  == computeForwardDistance(0, 1));
    assert(7  == computeForwardDistance(0, 7));
    assert(0  == computeForwardDistance(7, 7));
    assert(31 == computeForwardDistance(31, 30)); // overflow
    assert(1  == computeForwardDistance(31, 0));  // overflow
    return 0;
}
\end{minted}
\end{minipage}

\subsection{Payload truncation}

The deterministic nature of UAVCAN in general and DSDL in particular allows implementations to statically determine the
maximum amount of memory that is required to contain a data object of a particular type.
Consequently, an implementation that is interested in receiving data objects of a particular type\footnote{%
    Messages, service requests, or service responses.
}
can statically determine the maximum length of the transfer payload.

Implementations shall be able to handle incoming transfers containing a larger amount of payload data than expected.
In the event of such extra payload being received, a compliant implementation should silently\footnote{%
    Such occurrence is not indicative of a problem so it shall not be reported as such.
}
discard the excessive (unexpected) data at the end of the received payload.
The transfer CRC, if applicable, shall be validated regardless of the presence of the extra payload in the transfer.

The requirement to silently discard the excessive payload data at the end of the transfer is motivated by
the necessity to allow extensibility of data type definitions, as described in chapter \ref{sec:dsdl}.
Additionally, excessive payload data may contain padding bytes if required by the particular transport layer.

\begin{figure}[H]
    $$
    \raisebox{1em}{\footnotesize{\text{first byte}}}
    \overbrace{%
        \underbrace{\huge{%
            \blacksquare\blacksquare\blacksquare\blacksquare\blacksquare\blacksquare%
            \blacksquare\blacksquare\blacksquare\blacksquare\blacksquare\blacksquare%
        }}_{\substack{\text{Expected, accepted} \\ \text{payload}}}%
        \underbrace{\huge{%
            \boxtimes\boxtimes\boxtimes\boxtimes\boxtimes\boxtimes\boxtimes\boxtimes%
        }}_{\substack{\text{Excessive, discarded} \\ \text{payload}}}%
    }^{\substack{%
        \text{Transfer CRC is validated} \\
        \text{for the entire transfer payload} \\
        \text{before the truncation}}
    }
    \raisebox{1em}{\footnotesize{\text{last byte}}}
    $$
    \caption{Transfer payload truncation.\label{fig:transport_payload_truncation}}
\end{figure}

\begin{remark}
    Let node $A$ publish an object of the following type over the subject $X$:

    \begin{minted}{python}
        float32 parameter
        float32 variance
    \end{minted}

    Let node $B$ subscribe to the subject $X$ expecting an object of the following type:

    \begin{minted}{python}
        float32 parameter
    \end{minted}

    The payload truncation requirement guarantees that the two nodes will be able to interoperate despite
    relying on incompatible data type definitions.
    Under this example, the duty of ensuring the semantic compatibility lies on the system integrator.
\end{remark}

Implementations shall not enforce the minimum payload size on received transfers.
In other words, implementations shall not discard a transfer even if it is determined that it contains less payload
data than expected.
Serialization validity constraints are to be enforced by the object deserialization routines
instead of the transport layer.

\subsection{State variables}

\subsubsection{Main principles}

Nodes that receive transfers must keep a certain set of state variables for each
received transfer descriptor (section \ref{sec:transfer_descriptor}).

The set of state variables as documented in the table \ref{table:transfer_receiver_state_variables}
will be referred to as the \emph{receiver state}.
For the purposes of this specification, it is assumed that the node will maintain a
mapping from transfer descriptors to receiver states, which will be referred to as the \emph{receiver map}.
It is understood that implementations might prefer different architectures, which is permitted as
long as the resulting behavior of the node observable at the protocol level is functionally equivalent.

Whenever a node receives a transfer, it will query its receiver map for the matching received transfer descriptor.
If the matching state does not exist, the node will add a new receiver state to the map
and initialize it as defined in section \ref{sec:transfer_reception_initial_state}.
The node then will proceed with the procedure of \emph{receiver state update},
which is defined in section \ref{sec:transfer_reception_state_update_redundant} for redundant transports
and section \ref{sec:transfer_reception_state_update_non_redundant} for non-redundant transports.

It is expected that some transfers will be aperiodic or ad-hoc,
which implies that the receiver map may over time accumulate receiver states that are no longer used.
Therefore, nodes are allowed, but not required, to remove any receiver state from the receiver map
as soon as the state reaches the \emph{transfer-ID timeout condition}\footnote{Such behavior is
not recommended for hard real-time applications, where deterministic static look-up tables
should be preferred instead.},
as defined in section \ref{sec:transfer_id_timeout_condition}.

Receiver state can only be modified when a new transport frame of a matching transfer is received.
This guarantee simplifies implementation, as it implies that the receiver states will not
require any periodic background maintenance activities.

\begin{UAVCANSimpleTable}{Transfer reception state variables}{|l X|}
    State               & Description \label{table:transfer_receiver_state_variables} \\
    Transfer payload    & Useful payload byte sequence; extended upon reception of new matching transport frames. \\
    Transfer-ID         & The transfer-ID value of the next expected transport frame. Section \ref{sec:transfer_id}. \\
    Next toggle bit     & Expected value of the toggle bit in the next transport frame.
                          Section \ref{sec:toggle_bit}. \\
    Transfer timestamp  & The local monotonic timestamp sampled when the first frame of the transfer arrived.
                          Here, ``monotonic'' means that the reference clock does not change its rate or make leaps. \\
    Interface index     & Only in the case of redundant transport interfaces. \\
\end{UAVCANSimpleTable}

\subsubsection{Initial state}\label{sec:transfer_reception_initial_state}

The initial state is reached when a new entry of the receiver map is created or an existing entry is reset.
Like any other state update, an entry can be created or reset only synchronously with
the reception of a matching transport frame.

Upon reset, the receiver state will meet the following conditions:

\begin{itemize}
    \item The transfer payload buffer is empty.
    \item The transfer-ID state matches the actual transfer-ID value from the newly received transfer,
    unless this is a non-first frame of a multi-frame transfer.
    In the latter case, the transfer-ID state will match the received transfer-ID value incremented by one.
    \item The toggle bit is set to its initial state (section \ref{sec:toggle_bit}).
    \item The transfer timestamp matches the reception timestamp from the transport frame.
    \item The interface index matches the index of the interface that the new frame was received from
    (for nodes with redundant interfaces only).
\end{itemize}

A receiver state must be reset when any of the following conditions are met:

\begin{itemize}
    \item A new receiver state instance is created.

    \item A transfer-ID timeout condition is reached (section \ref{sec:transfer_id_timeout_condition}).

    \item A first frame of a transfer (either a multi-frame or a single-frame; in the latter case, the same frame
    would also be the last frame of the transfer) is received from the same interface as the previous frame
    (does not apply to non-redundantly interfaced nodes),
    and the transfer-ID forward distance (section \ref{sec:transfer_id_forward_distance}) from the received
    transfer-ID to the stored transfer-ID is greater than one.

    \item Only for redundantly interfaced nodes: A first frame of a transfer is received,
    an interface switchover condition is reached (section \ref{sec:transfer_interface_switchover_condition}),
    and the transfer-ID forward distance from the stored transfer-ID to the received transfer-ID is
    less than the transfer-ID half range (section \ref{sec:transfer_id_forward_distance}).
\end{itemize}

\subsubsection{Transfer-ID timeout condition}\label{sec:transfer_id_timeout_condition}

A state is said to have reached the transfer-ID timeout condition
if the last matching transfer was seen more than 2 (two) seconds ago.
When this condition is reached, the receiver must accept the next transfer disregarding its transfer-ID value.

Nodes are allowed to use different timeout values, if that is believed to benefit the application.
If a different timeout value is used, it must be explicitly documented.

Low timeout values increase the risk of undetected transfer duplication when such transfers are significantly
delayed due to bus congestion, which is possible with very low-priority transfers when the bus utilization is high.

High timeout values increase the risk of an undetected transfer loss when a remote node suffers an emitted transfer-ID
map state loss (e.g., due to the whole node being restarted).
However, the effects of such a transfer loss caused by a loss of state on a remote node
are always confined to the first transfer only.

\subsubsection{Interface switchover condition}\label{sec:transfer_interface_switchover_condition}

This condition is only applicable for configurations with redundant transport interfaces, which means
the node is allowed to receive the next transfer from an interface that is not the same
the previous transfer was received from.

The condition is reached when the last matching transfer was successfully received more than
$T_\text{switch}$ seconds ago. The value of $T_\text{switch}$ should not exceed the reception transfer
ID timeout, as defined in section \ref{sec:transfer_id_timeout_condition},
because if $T_\text{switch}$ were to exceed the transfer-ID timeout, an interface switchover would be
performed by the normal receiver state reset procedure, rendering $T_\text{switch}$ useless.

The actual value of $T_\text{switch}$ can be either a constant chosen by the designer according
to the application requirements (e.g., the maximum recovery time in the event of an interface failure),
or the protocol stack can estimate this value automatically by analyzing the transfer intervals.

Nodes are required to let the first interface time out before using the next one because the
transfer-ID field is expected to wrap around frequently (every 32 transfers).
Different interfaces are expected to exhibit different latencies even in a properly functioning system,
especially if the system contains both redundantly-interfaced and non-redundantly-interfaced nodes.
If the latency of a backup interface relative to the primary interface exceeds 32 transfer intervals,
and receiving nodes were to be allowed to switch between interfaces freely disregarding the timeout,
the receiving node would skip the whole period of transfer-IDs (32 transfers will be lost).
The problem would primarily affect low-priority transfers where large latencies are more likely.

\subsection{State update in a redundant interface configuration}
\label{sec:transfer_reception_state_update_redundant}

The following pseudocode demonstrates the transfer reception process
for a configuration with redundant transport interfaces.
Implementations are allowed to implement the reception logic differently as long as the resulting
behavior is equivalent.

\clearpage
\begin{minted}{cpp}
// Constants:
tid_timeout := 2 seconds;
tid_half_range := 16;
iface_switch_delay := UserDefinedConstant; // Or autodetect

// State variables:
initialized := 0;
payload;
this_transfer_timestamp;
current_transfer_id;
iface_index;
toggle;

function receiveFrame(frame)
{
    // Resolving the state flags:
    tid_timed_out := (frame.timestamp - this_transfer_timestamp) > tid_timeout;
    same_iface := frame.iface_index == iface_index;
    start_of_transfer := frame.start_of_transfer;
    non_wrapped_tid := computeForwardDistance(current_transfer_id, frame.transfer_id) < tid_half_range;
    not_previous_tid := computeForwardDistance(frame.transfer_id, current_transfer_id) > 1;
    iface_switch_allowed := (frame.timestamp - this_transfer_timestamp) > iface_switch_delay;
    // Using the state flags from above, deciding whether we need to reset:
    need_restart :=
        (!initialized) or
        (tid_timed_out) or
        (same_iface and start_of_transfer and not_previous_tid) or
        (iface_switch_allowed and start_of_transfer and non_wrapped_tid);

    if (need_restart)
    {
        initialized := 1;
        iface_index := frame.iface_index;
        current_transfer_id := frame.transfer_id;
        payload.clear();
        toggle := frame.toggle;
        if (!start_of_transfer)
        {
            current_transfer_id.increment();
            return;         // Ignore this frame, since the start of the transfer has already been missed
        }
    }

    if (frame.iface_index != iface_index)
    {
        return;  // Wrong interface, ignore
    }

    if (frame.toggle != toggle)
    {
        return;  // Unexpected toggle bit, ignore
    }

    if (frame.transfer_id != current_transfer_id)
    {
        return;  // Unexpected transfer-ID, ignore
    }

    if (start_of_transfer)
    {
        this_transfer_timestamp := frame.timestamp;
    }

    toggle := !toggle;
    payload.append(frame.data);

    if (frame.end_of_transfer)
    {
        // CRC validation for multi-frame transfers is intentionally omitted for brevity
        processTransfer(payload, ...);
        current_transfer_id.increment();
        toggle := 1;
        payload.clear();
    }
}
\end{minted}

\clearpage
\subsection{State update in a non-redundant interface configuration}
\label{sec:transfer_reception_state_update_non_redundant}

The following pseudocode demonstrates the transfer reception process for a configuration
with a non-redundant transport interface.
This is a specialization of the more general algorithm defined for redundant transport.
Implementations are allowed to implement the reception logic differently as long as the resulting
behavior is equivalent.

\begin{minted}{cpp}
// Constants:
tid_timeout := 2 seconds;

// State variables:
initialized := 0;
payload;
this_transfer_timestamp;
current_transfer_id;
toggle;

function receiveFrame(frame)
{
    // Resolving the state flags:
    tid_timed_out := (frame.timestamp - this_transfer_timestamp) > tid_timeout;
    start_of_transfer := frame.start_of_transfer;
    not_previous_tid := computeForwardDistance(frame.transfer_id, current_transfer_id) > 1;
    // Using the state flags from above, deciding whether we need to reset:
    need_restart :=
        (!initialized) or
        (tid_timed_out) or
        (start_of_transfer and not_previous_tid);

    if (need_restart)
    {
        initialized := 1;
        current_transfer_id := frame.transfer_id;
        payload.clear();
        toggle := frame.toggle;
        if (!start_of_transfer)
        {
            current_transfer_id.increment();
            return; // Ignore this frame, since the start of the transfer has already been missed
        }
    }

    if (frame.toggle != toggle)
    {
        return;  // Unexpected toggle bit, ignore
    }

    if (frame.transfer_id != current_transfer_id)
    {
        return;  // Unexpected transfer-ID, ignore
    }

    if (start_of_transfer)
    {
        this_transfer_timestamp := frame.timestamp;
    }

    toggle := !toggle;
    payload.append(frame.data);

    if (frame.end_of_transfer)
    {
        // CRC validation for multi-frame transfers is intentionally omitted for brevity
        processTransfer(payload, ...);
        current_transfer_id.increment();
        toggle := 1;
        payload.clear();
    }
}
\end{minted}

% Please keep \clearpage in front of every transport-specific specification to enforce clear separation!
\clearpage\section{UAVCAN/CAN}\label{sec:transport_can}

\hyphenation{UAVCAN/CAN}  % Disable hyphenation.

This section specifies a concrete transport based on CAN 2.0B (ISO 11898).
Throughout this section, ``CAN'' implies both Classic CAN 2.0 and CAN FD, unless specifically noted otherwise.
CAN FD should be considered the primary transport protocol.

\begin{UAVCANSimpleTable}{UAVCAN/CAN transport capabilities}{|l X l|}
    \label{table:transport_can_capabilities}
    Parameter & Value & References \\

    Maximum node-ID value &
    127 (7 bits wide). &
    \ref{sec:basic} \\

    Transfer-ID mode &
    Cyclic, modulo 32. &
    \ref{sec:transport_transfer_id} \\

    Number of transfer priority levels &
    8 (no additional levels). &
    \ref{sec:transport_transfer_priority} \\

    Largest single-frame transfer payload &
    Classic CAN -- 7~bytes, CAN FD -- up to 63~bytes. &
    \ref{sec:transport_transfer_payload} \\

    Anonymous transfers &
    Supported with non-deterministic collision resolution policy. &
    \ref{sec:transport_route_specifier} \\
\end{UAVCANSimpleTable}

\subsection{CAN ID field}

UAVCAN/CAN transport frames are CAN 2.0B frames.
The 29-bit CAN ID encodes the session specifier\footnote{Section \ref{sec:transport_session_specifier}.}
of the transfer it belongs to along with its priority.
The CAN data field of every frame contains the transfer payload
(or, in the case of multi-frame transfers, a fraction thereof), the transfer-ID, and other metadata.

UAVCAN/CAN can share the same bus with other high-level CAN bus protocols provided that they
do not make use of CAN 2.0B frames\footnote{For example, CANOpen or CANaerospace.}.
However, future revisions of UAVCAN/CAN may utilize CAN 2.0A as well,
so backward compatibility with other high-level CAN bus protocols is not guaranteed.

UAVCAN/CAN can share the same bus with UAVCAN/CAN v0 -- the earlier experimental revision of the protocol
(not recommended for new designs).
The protocol version can be determined at runtime on a per-frame basis as described
in section~\ref{sec:transport_can_toggle_bit}.

UAVCAN/CAN utilizes two different CAN ID bit layouts for message transfers and service transfers.
The bit layouts are summarized on figure~\ref{fig:transport_can_id_structure}.
Tables \ref{table:transport_can_id_fields_message_transfer} and \ref{table:transport_can_id_fields_service_transfer}
summarize the purpose of each field and their permitted values
for message transfers and service transfers, respectively.

% Please do not remove the hard placement specifier [H], it is needed to keep elements ordered.
\begin{figure}[H]
    \centering
    \resizebox{\textwidth}{!}{
        \footnotesize
        \begin{tabular}{|l|c|c|c|c|c|c|c|c|c|c|c|c|c|c|c|c|c|c|c|c|c|c|c|c|c|c|c|c|c|} \hline
            %
            % Message transfer
            %
            \multirow{2}{*}{\textbf{Message}} &
            \multicolumn{4}{c|}{Service, not message} &
            \multicolumn{3}{c|}{Anonymous} &
            \multicolumn{14}{c|}{\multirow{2}{*}{Subject-ID}} &
            \multicolumn{1}{c|}{\multirow{2}{*}{R}} &
            \multicolumn{7}{c|}{\multirow{2}{*}{Source node-ID}}
            \\\cline{2-4} \cline{7-8}

            &
            \multicolumn{3}{c|}{Priority}
            &
            &
            &
            R &
            \multicolumn{15}{c|}{} &
            &
            \multicolumn{7}{c|}{}
            \\

            \textbf{Values} &
            \multicolumn{3}{c|}{$[0, 7]$} &
            $0$ &
            $\mathbb{B}$ &
            $0$ &
            \multicolumn{1}{c}{} &
            \multicolumn{14}{c|}{$[0, 32767]$} &
            $0$ &
            \multicolumn{7}{c|}{$[0, 127]$}
            \\\hline

            \textbf{CAN ID bit} &
            28 & 27 & 26 & 25 & 24 & 23 & 22 & 21 & 20 & 19 & 18 & 17 & 16 & 15 &
            14 & 13 & 12 & 11 & 10 &  9 &  8 &  7 &  6 &  5 &  4 &  3 &  2 &  1 &  0
            \\\hline

            \textbf{CAN ID byte} &
            \multicolumn{5}{c|}{3} & \multicolumn{8}{c|}{2} & \multicolumn{8}{c|}{1} & \multicolumn{8}{c|}{0}
            \\\hline

            \multicolumn{30}{c}{} \\ \hline % Table separator

            %
            % Service transfer
            %
            \multirow{2}{*}{\textbf{Service}} &
            \multicolumn{4}{c|}{Service, not message} &
            \multicolumn{5}{c|}{Request, not response} &
            \multicolumn{6}{c|}{} &
            \multicolumn{7}{c|}{\multirow{2}{*}{Destination node-ID}} &
            \multicolumn{7}{c|}{\multirow{2}{*}{Source node-ID}}
            \\\cline{2-4} \cline{7-10}

            &
            \multicolumn{3}{c|}{Priority} &
            &
            &
            R &
            \multicolumn{9}{c|}{Service-ID} &
            \multicolumn{7}{c|}{} &
            \multicolumn{7}{c|}{}
            \\

            \textbf{Values} &
            \multicolumn{3}{c|}{$[0, 7]$} &
            $1$ &
            $\mathbb{B}$ &
            $0$ &
            \multicolumn{9}{c|}{$[0, 511]$} &
            \multicolumn{7}{c|}{$[0, 127]$} &
            \multicolumn{7}{c|}{$[0, 127]$}
            \\\hline

            \textbf{CAN ID bit} &
            28 & 27 & 26 & 25 & 24 & 23 & 22 & 21 & 20 & 19 & 18 & 17 & 16 & 15 &
            14 & 13 & 12 & 11 & 10 &  9 &  8 &  7 &  6 &  5 &  4 &  3 &  2 &  1 &  0
            \\\hline

            \textbf{CAN ID byte} &
            \multicolumn{5}{c|}{3} & \multicolumn{8}{c|}{2} & \multicolumn{8}{c|}{1} & \multicolumn{8}{c|}{0}
            \\\hline
        \end{tabular}
    }
    \caption{CAN ID bit layout}\label{fig:transport_can_id_structure}
\end{figure}

\begin{UAVCANSimpleTable}{CAN ID bit fields for message transfers}{|l l l X|}
    \label{table:transport_can_id_fields_message_transfer}
    Field               & Width & Valid values  & Description \\

    Transfer priority   & 3     & $[0, 7]$ (any)    & Section \ref{sec:transport_transfer_priority}. \\

    Service not message & 1     & $0$               & Always zero for message transfers. \\

    Anonymous           & 1     & $\{0, 1\}$ (any)  & Zero for regular message transfers,
                                                      one for anonymous transfers. \\

    Reserved bit 23     & 1     & $0$               & Discard frame if this field has a different value. \\

    Subject-ID          & 15    & $[0, 32767]$ (any) & Subject-ID of the current message transfer. \\

    Reserved bit 7      & 1     & $0$               & Discard frame if this field has a different value. \\

    Source node-ID      & 7     & $[0, 127]$ (any)  & Node-ID of the origin.
                                                      For anonymous transfers, this field contains a pseudo-ID instead,
                                                      as described in section
                                                      \ref{sec:transport_can_source_node_pseudo_id}. \\
\end{UAVCANSimpleTable}

\begin{UAVCANSimpleTable}{CAN ID bit fields for service transfers}{|l l l X|}
    \label{table:transport_can_id_fields_service_transfer}
    Field               & Width & Valid values  & Description \\

    Transfer priority   & 3     & $[0, 7]$ (any)    & Section \ref{sec:transport_transfer_priority}. \\

    Service not message & 1     & $1$               & Always one for service transfers. \\

    Request not response& 1     & $\{0, 1\}$ (any)  & One for service request, zero for service response. \\

    Reserved bit 23     & 1     & $0$               & Discard frame if this field has a different value. \\

    Service-ID          & 9     & $[0, 511]$ (any)  & Service-ID of the encoded service object
                                                      (request or response). \\

    Destination node-ID & 7     & $[0, 127]$ (any)  & Node-ID of the destination:
                                                      server if request, client if response. \\

    Source node-ID      & 7     & $[0, 127]$ (any)  & Node-ID of the origin:
                                                      client if request, server if response. \\
\end{UAVCANSimpleTable}

\subsubsection{Transfer priority}

Valid values for transfer priority range from 0 to 7, inclusively,
where 0 corresponds to the highest priority, and 7 corresponds to the lowest priority
(according to the CAN bus arbitration policy).

In multi-frame transfers, the value of the priority field shall be identical for all frames of the transfer.

\begin{remark}[breakable]
    When multiple transfers of different types with the same priority contest for bus access,
    the following precedence is ensured (from higher priority to lower priority):

    \begin{enumerate}
        \item Message transfers (the primary method of data exchange in UAVCAN networks).
        \item Anonymous (message) transfers.
        \item Service response transfers (preempt requests).
        \item Service request transfers (responses take precedence over requests to make service calls more atomic
              and reduce the number of pending states in the system).
    \end{enumerate}

    Mnemonics for transfer priority levels are provided in section \ref{sec:transport_transfer_priority},
    and their mapping to the UAVCAN/CAN priority field is as follows:

    \begin{UAVCANCompactTable}{|l X|}
        Priority field value    & Mnemonic name \\
        0                       & Exceptional   \\
        1                       & Immediate     \\
        2                       & Fast          \\
        3                       & High          \\
        4                       & Nominal       \\
        5                       & Low           \\
        6                       & Slow          \\
        7                       & Optional      \\
    \end{UAVCANCompactTable}

    Since the value of transfer priority is required to be the same for all frames in a transfer,
    it follows that the value of the CAN ID is guaranteed to be the same for all CAN frames of the transfer.
    Given a constant transfer priority value, all CAN frames under a given session specifier will be equal.
\end{remark}

\subsubsection{Source node-ID field in anonymous transfers}\label{sec:transport_can_source_node_pseudo_id}

The source node-ID field of anonymous transfers shall be initialized with a pseudorandom \emph{pseudo-ID} value.
The source of the pseudorandom data used for the pseudo-ID shall aim to produce different values
for different CAN frame data field values.

A node transmitting an anonymous transfer shall abort its transmission and discard it upon detection of a bus error.
Some method of media access control should be used at the application level for further conflict resolution.

\begin{remark}[breakable]
    CAN bus does not allow different nodes to transmit CAN frames with different data under the same CAN ID value.
    Owing to the fact that the CAN ID includes the node-ID of the transmitting node,
    this restriction does not affect non-anonymous transfers.
    However, anonymous transfers would violate this restriction because their source node-ID is not defined,
    hence the additional measures described in this section.

    A possible way of initializing the source node pseudo-ID value is to compute the arithmetic sum
    of all bytes of the transfer payload, taking the least significant bits of the result as the pseudo-ID
    (usage of stronger hashes is encouraged).
    Implementations that adopt this approach will be using the same pseudo-ID value for identical transfer payloads,
    which is acceptable since this will not trigger an error on the bus.

    Because the set of possible pseudo-ID values is small,
    a collision where multiple nodes emit CAN frames with different data but the same CAN ID is likely to happen
    despite the randomization measures described here.
    Therefore, if anonymous transfers are used,
    implementations shall account for possible errors on the CAN bus triggered by CAN ID collisions.

    Automatic retransmission should be disabled for anonymous transfers (like in TTCAN).
    This measure allows the protocol to prevent temporary disruptions that may occur if the automatic
    retransmission on bus error is not suppressed.

    Additional bus access control logic is needed at the application level because
    the possibility of identifier collisions in anonymous frames undermines the access control logic implemented
    in CAN bus controller hardware.

    The described principles make anonymous transfers highly non-deterministic and inefficient.
    This is considered acceptable because the scope of anonymous transfers is limited to a very narrow set of use
    cases which tolerate their downsides. The UAVCAN specification employs anonymous transfers only for the
    plug-and-play feature defined in section \ref{sec:application_functions}.
    Deterministic applications are advised to avoid reliance on anonymous transfers completely.

    None of the above considerations affect nodes that do not transmit anonymous transfers.
\end{remark}

\subsection{CAN data field}

\subsubsection{Layout}

UAVCAN/CAN utilizes a fixed layout of the CAN data field:
the last byte of the CAN data field contains the metadata, it is referred to as the \emph{tail byte}.
The preceding bytes of the data field contain the transfer payload,
which may be extended with padding bytes and transfer CRC.

A CAN frame whose data field contains less than one byte is not a valid UAVCAN/CAN frame.

The bit layout of the tail byte is shown in table \ref{table:transport_can_tail_byte}.

% Please do not remove the hard placement specifier [H], it is needed to keep tables ordered.
\begin{table}[H]\caption{Tail byte structure}\label{table:transport_can_tail_byte}
    \begin{tabu}{| c l | X[c2] X[c3] |}
        \hline
        \rowfont{\bfseries}
        Bit & Field & Single-frame transfers & Multi-frame transfers \\
        \hline
        7   & \textbf{Start of transfer}& Always 1  & First frame: 1, otherwise 0. \\\hline
        6   & \textbf{End of transfer}  & Always 1  & Last frame: 1, otherwise 0. \\\hline
        5   & \textbf{Toggle bit}       & Always 1  & First frame: 1, then alternates;
                                                      section \ref{sec:transport_can_toggle_bit}. \\\hline
        4   &                           & \multicolumn{2}{c|}{} \\
        3   &                           & \multicolumn{2}{c|}{Modulo 32 (range [0, 31])} \\
        2   & \textbf{Transfer-ID}      & \multicolumn{2}{c|}{section \ref{sec:transport_transfer_id}} \\
        1   &                           & \multicolumn{2}{c|}{} \\
        0   &                           & \multicolumn{2}{c|}{\footnotesize{(least significant bit)}} \\
        \hline
    \end{tabu}
\end{table}

\subsubsection{Toggle bit}\label{sec:transport_can_toggle_bit}

Transport frames that form a multi-frame transfer are equipped with a \emph{toggle bit}
which alternates its state every frame within the transfer for frame deduplication purposes\footnote{%
    A frame that appears valid to the receiving node may under certain conditions appear invalid to the transmitter,
    triggering the latter to retransmit the frame, in which case it will be duplicated on the side of the receiver.
}.

\begin{remark}[breakable]
    The toggle bit can be used to facilitate operation of heterogeneous deployments where the experimental
    UAVCAN/CAN v0 shares the same CAN bus with the current version of the standard.

    Whenever a new transfer is initiated, the original state of the toggle bit reflects the protocol version.
    Implementations that need to support simultaneous operation of two versions of the protocol can record
    the state of the toggle bit when the ``start of transfer'' bit is set, and keep this information
    indexed by the value of the CAN ID field (all frames of a transfer are guaranteed to share the same CAN ID).
    The resulting mapping from CAN ID to the protocol version can be used to route incoming frames to the
    implementation of the appropriate version of the protocol.

    \begin{UAVCANSimpleTable}{Protocol version detection based on the toggle bit}{|l l X|}
        Start of transfer   & Toggle bit    & Protocol version \\
        1                   & 0             & UAVCAN v0 (experimental version). \\
        1                   & 1             & UAVCAN v1 (this version). \\
        0                   & x             & Keep the state of the toggle bit from the first frame of the transfer
                                              to detect protocol version in multi-frame transfers. \\
    \end{UAVCANSimpleTable}
\end{remark}

\subsubsection{Transfer payload decomposition}

The transport-layer MTU of Classic CAN-based implementations shall be 8 bytes (the maximum).
The transport-layer MTU of CAN FD-based implementations should be 64 bytes (the maximum).

CAN FD does not guarantee byte-level granularity of the CAN data field length.
If the desired length of the CAN data field cannot be represented due to the granularity constraints,
zero padding bytes are used.

In single-frame transfers, padding bytes are inserted between the end of the payload and the tail byte.

In multi-frame transfers, the transfer payload is appended with trailing zero padding bytes
followed by the transfer CRC (section \ref{sec:transport_can_transfer_crc}).
All transport frames of a multi-frame transfer except the last one shall fully utilize the available
data field capacity; hence, padding is unnecessary there.
The number of padding bytes is computed so that the length granularity constraints
for the last frame of the transfer are satisfied.

\begin{remark}
    Usage of padding bytes implies that when a serialized message is being deserialized by a receiving node,
    the byte sequence used for deserialization may be longer than the actual byte sequence generated by the
    emitting node during serialization.
    This behavior is compatible with the DSDL specification.

    The weak MTU requirement for CAN FD is designed to avoid compatibility issues.
\end{remark}

\subsubsection{Transfer CRC}\label{sec:transport_can_transfer_crc}

Payload of multi-frame transfers is extended with a transfer CRC for validating the correctness of their reassembly.
Transfer CRC is not used with single-frame transfers.

The transfer CRC is computed over the entire payload of the multi-frame transfer
plus the trailing padding bytes, if any.
The resulting CRC value is appended to the transfer payload after the padding bytes (if any)
in the \emph{big-endian byte order} (most significant byte first)\footnote{%
    This is the native byte order for this CRC function.
}.

The CRC function is the standard CRC-16-CCITT:
initial value $\mathrm{FFFF}_{16}$, polynomial $\mathrm{1021}_{16}$,
not reversed, no output XOR, big endian.
The value for an input sequence $\left(49, 50, \ldots, 56, 57\right)$ is $\mathrm{29B1}_{16}$.
The following code snippet provides a basic implementation of the transfer CRC algorithm in C++
(LUT-based alternatives exist).

\begin{samepage}
\begin{minted}{cpp}
#include <cstdint>
#include <cstddef>

/// UAVCAN/CAN transfer CRC function implementation. License: CC0, no copyright reserved.
class CANTransferCRC
{
    std::uint16_t value_ = 0xFFFFU;

public:
    void add(const std::uint8_t byte)
    {
        value_ ^= static_cast<std::uint16_t>(byte) << 8U;
        for (std::uint8_t bit = 8; bit > 0; --bit)
        {
            if ((value_ & 0x8000U) != 0)
            {
                value_ = (value_ << 1U) ^ 0x1021U;
            }
            else
            {
                value_ = value_ << 1U;
            }
        }
    }

    void add(const std::uint8_t* bytes, std::size_t length)
    {
        while (length-- > 0)
        {
            add(*bytes++);
        }
    }

    [[nodiscard]] std::uint16_t get() const { return value_; }
};
\end{minted}
\end{samepage}

\subsection{Examples}

\begin{remark}[breakable]
    Heartbeat from node-ID 42, nominal priority level,
    uptime starting from 0 and then incrementing by one every transfer,
    vendor-specific status code 3471:

    \begin{UAVCANCompactTable}{|l l|}
        CAN ID (hex)      & CAN data (hex)          \\
        \texttt{107D552A} & \texttt{00 00 00 00 04 78 68 E0} \\
        \texttt{107D552A} & \texttt{01 00 00 00 04 78 68 E1} \\
        \texttt{107D552A} & \texttt{02 00 00 00 04 78 68 E2} \\
        \texttt{107D552A} & \texttt{03 00 00 00 04 78 68 E3} \\
    \end{UAVCANCompactTable}

    \verb|uavcan.primitive.String.1.0| under subject-ID 4919 ($1337_{16}$) published by an anonymous node,
    the string is ``\verb|Hello world!|'' (ASCII); one byte of zero padding can be seen between
    the payload and the tail byte:

    \begin{UAVCANCompactTable}{|l l|}
        CAN ID (hex)      & CAN data (hex)                                           \\
        \texttt{11133775} & \texttt{00 18 48 65 6C 6C 6F 20 77 6F 72 6C 64 21 00 E0} \\
        \texttt{11133775} & \texttt{00 18 48 65 6C 6C 6F 20 77 6F 72 6C 64 21 00 E1} \\
        \texttt{11133775} & \texttt{00 18 48 65 6C 6C 6F 20 77 6F 72 6C 64 21 00 E2} \\
        \texttt{11133775} & \texttt{00 18 48 65 6C 6C 6F 20 77 6F 72 6C 64 21 00 E3} \\
    \end{UAVCANCompactTable}

    Node info request from node 123 to node 42 via Classic CAN, then response;
    notice how the transfer CRC is scattered across two frames:

    \begin{UAVCANCompactTable}{|l l l X|}
        CAN ID (hex)      & CAN data (hex)                                  & ASCII             & Comment \\

        \texttt{136B957B} & \texttt{E1}                                     & \texttt{.}        &
        The request contains no payload. \\

        \texttt{126BBDAA} & \texttt{01 00 00 00 01 00 00 A1}                & \texttt{........} &
        Start of response, toggle bit is set. \\

        \texttt{126BBDAA} & \texttt{00 00 00 00 00 00 00 01}                & \texttt{........} &
        Toggle bit is cleared. \\

        \texttt{126BBDAA} & \texttt{00 00 00 00 00 00 00 21}                & \texttt{.......!} &
        Toggle bit is set. \\

        \texttt{126BBDAA} & \texttt{00 00 00 00 00 00 00 01}                & \texttt{........} &
        Etc. \\

        \texttt{126BBDAA} & \texttt{00 00 \underline{24} 6F 72 67 2E 21}    & \texttt{..\underline{\$}org.!} &
        Array (string) length prefix. \\

        \texttt{126BBDAA} & \texttt{75 61 76 63 61 6E 2E 01}                & \texttt{uavcan..} &
        \\

        \texttt{126BBDAA} & \texttt{70 79 75 61 76 63 61 21}                & \texttt{pyuavca!} &
        \\

        \texttt{126BBDAA} & \texttt{6E 2E 64 65 6D 6F 2E 01}                & \texttt{n.demo..} &
        \\

        \texttt{126BBDAA} & \texttt{62 61 73 69 63 5F 75 21}                & \texttt{basic\_u!} &
        \\

        \texttt{126BBDAA} & \texttt{73 61 67 65 00 00 \underline{9A} 01}    & \texttt{sage..\underline{.}.} &
        Transfer CRC, MSB. \\

        \texttt{126BBDAA} & \texttt{\underline{E7} 61}                      & \texttt{\underline{.}a}       &
        Transfer CRC, LSB. \\
    \end{UAVCANCompactTable}

    \verb|uavcan.primitive.array.Natural8.1.0| under subject-ID 4919 ($1337_{16}$) published by node 59,
    the array contains an arithmetic sequence $\left(0, 1, 2, \ldots{}, 89, 90, 91\right)$;
    the transport MTU is 64 bytes:

    \begin{UAVCANCompactTable}{|l X[2] X|}
        CAN ID (hex)      & CAN data (hex) & Comment \\
        \texttt{1013373B} &
        \texttt{%
            00 B8 00 01 02 03 04 05 06 07 08 09 0A 0B 0C 0D 0E 0F 10 11 12 13 14 15 16 17 18 19 1A 1B 1C 1D 1E 1F 20
            21 22 23 24 25 26 27 28 29 2A 2B 2C 2D 2E 2F 30 31 32 33 34 35 36 37 38 39 3A 3B 3C A0
        } &
        First frame: 1.~payload (array length prefix is 92); 2.~tail byte. \\

        \texttt{1013373B} &
        \texttt{%
            3D 3E 3F 40 41 42 43 44 45 46 47 48 49 4A 4B 4C 4D 4E 4F 50 51 52 53 54 55 56 57 58 59 5A 5B
            \underline{00} \underline{00} \underline{00} \underline{00} \underline{00} \underline{00} \underline{00}
            \underline{00} \underline{00} \underline{00} \underline{00} \underline{00} \underline{00} \underline{00}
            \textbf{C0} \textbf{48} 40
        } &
        Last frame: 1.~payload; 2.~padding (underlined); 3.~transfer CRC (bold); 4.~tail byte. \\
    \end{UAVCANCompactTable}
\end{remark}

\subsection{Software design considerations}

\subsubsection{Ordered transmission}

The CAN controller driver software shall guarantee that CAN frames with identical CAN ID values
will be transmitted in their order of appearance in the transmission queue\footnote{%
    This is because multi-frame transfers use identical CAN ID for all frames of the transfer,
    and UAVCAN requires that all frames of a multi-frame transfer shall be transmitted in the correct order.
}.

\subsubsection{Transmission timestamping}

\begin{remark}[breakable]
    Certain application-level functions of UAVCAN may require the driver to timestamp outgoing transport frames,
    e.g., the time synchronization function.
    A sensible approach to transmission timestamping is built around the concept of \emph{loop-back frames},
    which is described here.

    If the application needs to timestamp an outgoing frame, it sets a special flag -- the \emph{loop-back flag} --
    on the frame before sending it to the driver.
    The driver would then automatically re-enqueue this frame back into the reception queue once it is transmitted
    (keeping the loop-back flag set so that the application is able to distinguish the loop-back
    frame from regular received traffic).
    The timestamp of the loop-backed frame would be of the moment when it was delivered to the bus.

    The advantage of the loop-back based approach is that it relies on the same interface between
    the application and the driver that is used for regular communications.
    No complex and dangerous callbacks or write-backs from interrupt handlers are involved.
\end{remark}

\subsubsection{Inner priority inversion}

Implementations should take necessary precautions against the problem of inner priority inversion.

\begin{remark}[breakable]
    Suppose the application needs to emit a frame with the CAN ID $X$.
    The frame is submitted to the CAN controller's registers and the transmission is started.
    Suppose that afterwards it turned out that there is a new frame with the CAN ID $(X-1)$ that needs to be sent,
    too, but the previous frame $X$ is in the way, and it is blocking the transmission of the new frame.
    This may turn into a problem if the lower-priority frame is losing arbitration on the bus due
    to the traffic on the bus having higher priority than the current frame,
    but lower priority than the next frame that is waiting in the queue.

    A naive solution to this is to continuously check whether the priority of the frame that is currently being
    transmitted by the CAN controller is lower than the priority of the next frame in the queue, and if it is,
    abort transmission of the current frame, move it back to the transmission queue,
    and begin transmission of the new one instead.
    This approach, however, has a hidden race condition:
    the old frame may be aborted at the moment when it has already been received by remote nodes,
    which means that the next time it is re-transmitted, the remote nodes will see it duplicated.
    Additionally, this approach increases the complexity of the driver and can possibly affect
    its throughput and latency.

    Most CAN controllers offer a robust solution to the problem:
    they have multiple transmission mailboxes (usually at least 3),
    and the controller always chooses for transmission the mailbox which contains the highest priority frame.
    This provides the application with a possibility to avoid the inner priority inversion problem:
    whenever a new transmission is initiated, the application should check whether the priority of the next frame
    is higher than any of the other frames that are already awaiting transmission.
    If there is at least one higher-priority frame pending,
    the application doesn't move the new one to the controller's transmission mailboxes,
    it remains in the queue.
    Otherwise, if the new frame has a higher priority level than all of the pending frames,
    it is pushed to the controller's transmission mailboxes and removed from the queue.
    In the latter case, if a lower-priority frame loses arbitration,
    the controller would postpone its transmission and try transmitting the higher-priority one instead.
    That resolves the problem.

    There is an interesting extreme case, however.
    Imagine a controller equipped with $N$ transmission mailboxes.
    Suppose the application needs to emit $N$ frames in the increasing order of priority,
    which leads to all of the transmission mailboxes of the controller being occupied.
    Now, if all of the conditions below are satisfied, the system ends up with a priority inversion condition
    nevertheless, despite the measures described above:

    \begin{itemize}
        \item The highest-priority pending CAN frame cannot be transmitted due to the bus being saturated
        with a higher-priority traffic.
        \item The application needs to emit a new frame which has a higher priority than that which saturates the bus.
    \end{itemize}

    If both hold, a priority inversion is afoot because there is no free transmission mailbox to
    inject the new higher-priority frame into.
    The scenario is extremely unlikely, however;
    it is also possible to construct the application in a way that would preclude the problem,
    e.g., by limiting the number of simultaneously used distinct CAN ID values.

    The following pseudocode demonstrates the principles explained above:

    \begin{samepage}
    \begin{minted}{cpp}
    // Returns the index of the TX mailbox that can be used for the transmission of the newFrame
    // If none are available, returns -1.
    getFreeMailboxIndex(newFrame)
    {
        chosen_mailbox = -1     // By default, assume that no mailboxes are available

        for i = 0...NumberOfTxMailboxes
        {
            if isTxMailboxFree(i)
            {
                chosen_mailbox = i
                // Note: cannot break here, shall check all other mailboxes as well.
            }
            else
            {
                if not isFramePriorityHigher(newFrame, getFrameFromTxMailbox(i))
                {
                    chosen_mailbox = -1
                    break   // Denied - shall wait until this mailbox has finished transmitting
                }
            }
        }

        return chosen_mailbox
    }
    \end{minted}
    \end{samepage}
\end{remark}

\subsubsection{Automatic hardware acceptance filter configuration}

\begin{remark}[breakable]
    Most CAN controllers are equipped with hardware acceptance filters.
    Hardware acceptance filters reduce the application workload by ignoring irrelevant CAN frames on the bus
    by comparing their ID values against the set of relevant ID values configured by the application.

    There exist two common approaches to CAN hardware filtering:
    list-based and mask-based.
    In the case of the list-based approach, every CAN frame detected on the bus is compared
    against the set of reference CAN ID values provided by the application;
    only those frames that are found in the reference set are accepted.
    Due to the complex structure of the CAN ID field used by UAVCAN,
    usage of the list-based filtering method with this protocol is impractical.

    Most CAN controller vendors implement mask-based filters,
    where the behavior of each filter is defined by two parameters: the mask $M$ and the reference ID $R$.
    Then, such filter accepts only those CAN frames for which the following bitwise logical condition holds
    true\footnote{Notation: $\land$ -- bitwise logical AND, $\oplus$ -- bitwise logical XOR,
    $\neg$ -- bitwise logical NOT.}:
    $$((X \land M) \oplus R) \leftrightarrow 0$$
    where $X$ is the CAN ID value of the evaluated frame.

    Complex UAVCAN applications are often required to operate with more distinct transfers than there are
    acceptance filters available in the hardware.
    That creates the challenge of finding the optimal configuration of the available filters that meets the
    following criteria:
    \begin{itemize}
        \item All CAN frames needed by the application are accepted.
        \item The number of irrelevant frames (i.e., not used by the application) accepted from the bus is minimized.
    \end{itemize}

    The optimal configuration is a function of the number of available hardware filters,
    the set of distinct transfers needed by the application,
    and the expected frequency of occurrence of all possible distinct transfers on the bus.
    The latter is important because if there are to be irrelevant transfers,
    it makes sense to optimize the configuration so that the acceptance of less common irrelevant transfers
    is preferred over the more common irrelevant transfers, as that reduces the processing load on the application.

    The optimal configuration depends on the properties of the network the node is connected to.
    In the absence of the information about the network,
    or if the properties of the network are expected to change frequently,
    it is possible to resort to a quasi-optimal configuration which assumes that
    the occurrence of all possible irrelevant transfers is equally probable.
    As such, the quasi-optimal configuration is a function of only the number of available hardware filters
    and the set of distinct transfers needed by the application.

    The quasi-optimal configuration can be easily found automatically.
    Certain implementations of the UAVCAN protocol stack include this functionality,
    allowing the application to easily adjust the configuration of the hardware acceptance filters
    using a very simple API.

    A quasi-optimal hardware acceptance filter configuration algorithm is described below.
    The approach was first proposed by P. Kirienko and I. Sheremet in 2015.

    First, the bitwise \emph{filter merge} operation is defined on filter configurations $A$ and $B$.
    The set of CAN frames accepted by the merged filter configuration is a superset of
    those accepted by $A$ and $B$.
    The definition is as follows:
    \begin{equation*}
    \begin{split}
        m_M(R_A, R_B, M_A, M_B) & = M_A \land M_B \land \neg (R_A \oplus R_B) \\
        m_R(R_A, R_B, M_A, M_B) & = R_A \land m_M(R_A, R_B, M_A, M_B)
    \end{split}
    \end{equation*}

    The \emph{filter rank} is a function of the mask of the filter.
    The rank of a filter is a unitless quantity that defines in relative terms how selective the filter
    configuration is.
    The rank of a filter is proportional to the likelihood that the filter will reject a random CAN ID.
    In the context of hardware filtering, this quantity is conveniently representable via the number of bits set in
    the filter mask parameter (also known as \emph{population count}):
    \begin{equation*}
    r(M) =
    \begin{cases}
        0                                   &\mid M < 1 \\
        r(\lfloor\frac{M}{2}\rfloor)        &\mid M \bmod 2 = 0 \\
        r(\lfloor\frac{M}{2}\rfloor) + 1    &\mid M \bmod 2 \neq 0 \\
    \end{cases}
    \end{equation*}

    Having the low-level operations defined, we can proceed to define the whole algorithm.
    First, construct the initial set of CAN acceptance filter configurations
    according to the requirements of the application.
    Then, as long as the number of configurations in the set exceeds the number of available
    hardware acceptance filters, repeat the following:
    \begin{enumerate}
        \item Find the pair $A$, $B$ of configurations in the set for which $r(m_M(R_A, R_B, M_A, M_B))$ is maximized.
        \item Remove $A$ and $B$ from the set of configurations.
        \item Add a new configuration $X$ to the set of configurations, where
        $M_X = m_M(R_A, R_B, M_A, M_B)$, and $R_X = m_R(R_A, R_B, M_A, M_B)$.
    \end{enumerate}

    The algorithm reduces the number of filter configurations by one at each iteration,
    until the number of available hardware filters is sufficient to accommodate the whole set of configurations.
\end{remark}

