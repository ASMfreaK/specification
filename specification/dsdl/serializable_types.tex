\section{Serializable types}\label{sec:dsdl_serializable_types}

Values of the serializable type category can be exchanged between nodes over the UAVCAN network.
This is opposed to the expression types (section \ref{sec:dsdl_expression_types}),
instances of which can only exist while DSDL definitions are being evaluated.
The data serialization rules are defined in the section \ref{sec:dsdl_data_serialization}.

\subsection{Void types}

Void types are used for padding purposes.
As will be explained in later sections, it is desirable to align fields at byte boundaries;
void types can be used to facilitate that.

Void-typed fields are set to zero when an object is serialized and ignored when it is deserialized.
Void types can be used to reserve space in data type definitions for possible use in later versions of the data type.

The DSDL name pattern for void types is as follows: ``\verb|void[1-9]\d*|'',
where the trailing integer represents the length of the void field, in bits,
ranging from 1 to 64, inclusive.

Void types can be referred to directly by their name from any namespace.

\subsection{Primitive types}\label{sec:dsdl_primitive_types}

Primitive types are assumed to be known to DSDL processing tools a priori,
and as such, they need not be defined by the user.
Primitive types can be referred to directly by their name from any namespace.

The following text defines DSDL name patterns using the POSIX Extended Regular Expression (ERE) notation.

\subsubsection{Hierarchy}

The hierarchy of primitive types is documented below.

\begin{itemize}
    \item \textbf{Boolean types.} A boolean-typed value represents a variable of the Boolean algebra.
    A Boolean-typed value can have two values: true and false.
    The corresponding DSDL data type name is ``\verb|bool|''.

    \item \textbf{Algebraic types.} Those are types for which conventional algebraic operators are defined.
    \begin{itemize}
        \item \textbf{Integer types} are used to represent signed and unsigned integer values.
        See table \ref{table:dsd_integer_properties}.
        \begin{itemize}
            \item \textbf{Signed integer types.} These are used to represent values which can be negative.
            The corresponding DSDL data type name pattern is ``\verb|int[1-9]\d*|'',
            where the trailing integer represents the length of the
            encoded representation of the value, in bits, ranging from 2 to 64, inclusive.

            \item \textbf{Unsigned integer types.} These are used to represent non-negative values.
            The corresponding DSDL data type name pattern is ``\verb|uint[1-9]\d*|'',
            where the trailing integer represents the length of the
            encoded representation of the value, in bits, ranging from 2 to 64, inclusive.
        \end{itemize}

        \item \textbf{Floating point types} are used to approximately represent real values.
        The underlying encoded representation follows the IEEE 754 standard.
        The corresponding DSDL data type name pattern is ``\verb~float(16|32|64)~'', where the trailing
        integer represents the type of the IEEE 754 representation.
        See table \ref{table:dsd_floating_point_properties}.
    \end{itemize}
\end{itemize}

\begin{UAVCANSimpleTable}{Properties of integer types}{|l X l|}
    Category &
    DSDL names &
    Range, $X$ is bit length
    \label{table:dsd_integer_properties} \\

    Signed integers &
    \texttt{int2}, \texttt{int3}, \texttt{int4} \ldots{} \texttt{int62}, \texttt{int63}, \texttt{int64} &
    $\left[-\frac{2^{X}}{2},\frac{2^{X}}{2}-1\right]$ \\

    Unsigned integers &
    \texttt{uint2}, \texttt{uint3}, \texttt{uint4} \ldots{} \texttt{uint62}, \texttt{uint63}, \texttt{uint64} &
    $\left[0,2^{X}-1\right]$ \\
\end{UAVCANSimpleTable}

\begin{UAVCANSimpleTable}{Properties of floating point types}{|X X X X|}
    DSDL name        & Representation    & Approximate epsilon   & Approximate range
    \label{table:dsd_floating_point_properties} \\

    \texttt{float16} & IEEE 754 binary16 & $0.001$               & $\pm{}65504$      \\
    \texttt{float32} & IEEE 754 binary32 & $10^{-7}$             & $\pm{}10^{39}$    \\
    \texttt{float64} & IEEE 754 binary64 & $2 \times{} 10^{-16}$ & $\pm{}10^{308}$   \\
\end{UAVCANSimpleTable}

\subsubsection{Cast mode}

The concept of \emph{cast mode} is defined for all primitive types.
The cast mode defines the behavior when a primitive-typed entity is assigned a value that exceeds its range.
Such assignment requires some of the information to be discarded;
due to the loss of information involved, it is called a \emph{lossy assignment}.

The following cast modes are defined:
\begin{description}
    \item[Truncated mode] --- denoted with the keyword ``\verb|truncated|'' placed before the primitive type name.
    \item[Saturated mode] --- denoted with the optional keyword ``\verb|saturated|''
    placed before the primitive type name. If neither cast mode is specified, saturated mode is assumed by default.
    This essentially makes the ``\verb|saturated|'' keyword redundant; it is provided only for consistency.
\end{description}

When a primitive-typed entity is assigned a value that exceeds its range,
the resulting value is chosen according to the lossy assignment rules
specified in the table \ref{table:dsdl_cast_mode}.
Cases that are marked as illegal are not permitted in DSDL definitions.

\begin{UAVCANSimpleTable}{Lossy assignment rules per cast mode}{|l X X|}
    Type category & Truncated mode & Saturated mode (default)
    \label{table:dsdl_cast_mode} \\

    Boolean &
    Illegal: boolean type with truncated cast mode is not allowed. &
    Falsity if the value is zero or false, truth otherwise. \\

    Signed integer &
    Illegal: signed integer types with truncated cast mode are not allowed. &
    Nearest reachable value. \\

    Unsigned integer &
    Most significant bits are discarded. &
    Nearest reachable value. \\

    Floating point &
    Infinity with the same sign, unless the original value is not-a-number, in which case it will be preserved. &
    If the original value is finite, the nearest finite value will be used.
    Otherwise, in the case of infinity or not-a-number, the original value will be preserved. \\
\end{UAVCANSimpleTable}

Rules of conversion between values of different type categories do not affect compatibility at the protocol level,
and as such, they are to be implementation-defined.

\subsubsection{Expressions}

At the time of DSDL definition processing,
values of primitive types are represented as instances of the \verb|rational| type (section \ref{sec:dsdl_rational}),
with the exception of the type \verb|bool|,
instances of which are usable in DSDL expressions as-is.

\begin{UAVCANSimpleTable}{Operators defined on instances of type boolean}{|l X X|}
    Op & Type & Description \\

    \texttt{\textbf{!}}     & $(\texttt{bool}) \rightarrow \texttt{bool}$ & Logical not. \\

    \texttt{\textbf{||}}    & $(\texttt{bool}, \texttt{bool}) \rightarrow \texttt{bool}$ & Logical or. \\
    \texttt{\textbf{\&\&}}  & $(\texttt{bool}, \texttt{bool}) \rightarrow \texttt{bool}$ & Logical and. \\

    \texttt{\textbf{==}}    & $(\texttt{bool}, \texttt{bool}) \rightarrow \texttt{bool}$ & Equality. \\
    \texttt{\textbf{!=}}    & $(\texttt{bool}, \texttt{bool}) \rightarrow \texttt{bool}$ & Inequality. \\

\end{UAVCANSimpleTable}

\subsubsection{Reference list}

An exhaustive list of all void and primitive types
ordered by bit length is provided below for reference.
Note that the cast mode specifier is omitted intentionally.

\immediate\write18{rm -f ../latex.tmp}
\immediate\write18{../render_list_of_void_and_primitive_types.py > ../latex.tmp}
\immediate\input{../latex.tmp}

\subsection{Array types}

An array type represents an ordered collection of values.
All values in the collection share the same type, which is referred to as \emph{array element type}.
The array element type can be any type except:
\begin{itemize}
    \item void type;
    \item array type\footnote{Meaning that nested arrays are not allowed;
    however, the array element type can be a composite type which in turn may contain arrays.
    In other words, indirect nesting of arrays is permitted.}.
\end{itemize}

The number of elements in the array can be specified as a constant at the data type definition time,
in which case the array is said to be a \emph{fixed-length array}.
Alternatively, the number of elements can vary between zero and some static limit specified
at the data type definition time,
in which case the array is said to be a \emph{variable-length array}.
Variable-length arrays with unbounded maximum number of elements are not allowed.

Arrays are defined by adding a pair of square brackets after the array element type specification,
where the brackets contain the \emph{array capacity expression}.
The array capacity expression must yield a positive integer upon its evaluation;
any other value or type renders the current DSDL definition invalid.

The array capacity expression can be prefixed with the following character sequences in order to define
a variable-length array:
\begin{itemize}
    \item ``\verb|<|'' (ASCII code 60) --- indicates that the integer value yielded by the array capacity expression
    specifies the non-inclusive upper boundary of the number of elements.

    \item ``\verb|<=|'' (ASCII code 60 followed by 61) --- same as above, but the upper boundary is inclusive.
\end{itemize}
If neither of the above prefixes are provided, the resultant definition is that of a fixed-length array.

\subsection{Composite types}

\subsubsection{Kinds}

There are two kinds of composite data type definitions: message types and service types.
All versions of a data type must be of the same kind\footnote{%
For example, if a data type version 0.1 is of a message kind, all later versions of it must be messages, too.}.

A message data type defines one data schema.

A service data type contains two data schema definitions:
one for service request object, and one for service response object, in that order.
The two schemas are separated by the service response marker (``\verb|---|'') on a separate line.

The presence of the service response marker indicates that the data type definition at hand is of the service kind.
There may be no more than one service response marker in a given definition.

\subsubsection{Dependencies}

In order to refer to a composite type from another composite type definition
(e.g., for nesting or for referring to an external constant),
one has to specify the full data type name of the referred data type followed by its
major and minor version numbers separated by the namespace separator character,
as demonstrated on the figure \ref{fig:dsdl_nested_reference}.

To facilitate look-up of external dependencies,
implementations are expected to obtain from external sources%
\footnote{For example, from user-provided configuration options.}
the list of directories that are the roots of namespaces containing the referred dependencies.

\begin{figure}[H]
    $$
    \underbrace{\texttt{\huge{uavcan.node.Heartbeat}}}_{\text{full data type name}}%
    \texttt{\huge{.}}%
    \underbrace{\texttt{\huge{1.0}}}_{\substack{\text{version} \\ \text{numbers}}}%
    $$
    \caption{Reference to an external composite data type definition.\label{fig:dsdl_nested_reference}}
\end{figure}

If the referred data type and the referring data type share the same full namespace name,
it is allowed to omit the namespace from the referred data type specification
leaving only the short data type name, as demonstrated on the figure \ref{fig:dsdl_nested_reference_short}.
In this case, the referred data type will be looked for in the namespace of the referrer.
Partial omission of namespace components is not permitted.

\begin{figure}[H]
    $$
    \underbrace{\texttt{\huge{Heartbeat}}}_{\text{short data type name}}%
    \texttt{\huge{.}}%
    \underbrace{\texttt{\huge{1.0}}}_{\substack{\text{version} \\ \text{numbers}}}%
    $$
    \caption{Reference to an external composite data type definition located in the same namespace.
    \label{fig:dsdl_nested_reference_short}}
\end{figure}

Circular dependencies are not permitted.
A circular dependency renders all of the definitions involved in the dependency loop invalid.

If any of the referred definitions are marked as deprecated,
the referring definition must be marked deprecated as well\footnote{%
Deprecation is indicated with the help of a special directive, as explained in section \ref{sec:dsdl_directives}}.
If a non-deprecated definition refers to a deprecated definition,
the referring definition is malformed\footnote{%
This tainting behavior is designed to prevent unexpected breakage of
type hierarchies when one of the deprecated dependencies reaches its end of life.}.

\subsubsection{Unions}

Any data schema definition can be supplied with a special directive indicating that
it forms a tagged union\footnote{Section \ref{sec:dsdl_directives}.}.

A tagged union schema must contain at least two fields.
The value of a tagged union object is a function of the field which value it is currently holding
and the value of the field itself.

\subsubsection{Expressions}

In the case of composite definitions of the message kind,
constant attributes of the message type are accessible through application of the
attribute reference operator ``\verb|.|'' to the corresponding instance of the serializable metatype.

\begin{remark}
    \begin{minted}{python}
        uint64 MY_CONSTANT = vendor.MessageType.1.0.OTHER_CONSTANT
    \end{minted}
\end{remark}

