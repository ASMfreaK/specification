\section{Abstract concepts}

The function of the transport layer is to facilitate exchange of serialized representations of DSDL objects\footnote{%
    DSDL and data serialization are reviewed in chapter \ref{sec:dsdl}.
} between UAVCAN nodes over the \emph{transport network}.

\subsection{Transport model}\label{sec:transport_model}

This section introduces an abstract implementation-agnostic model of the UAVCAN transport layer.
The core relations are depicted on figure \ref{fig:transport_layer_model}.
Some of the concepts introduced at this level may not be manifested in the design of concrete transport layer
implementations; despite that, they are convenient for an abstract discussion.

% Please do not remove the hard placement specifier [H], it is needed to keep elements ordered.
\begin{figure}[H]
    \centering
    \resizebox{\textwidth}{!}{
        \footnotesize
        \begin{tabu}{|l l l|X[c,2] X[c] X[c]|l|}\hline\rowfont{\bfseries}
            \multicolumn{3}{|c|}{Taxonomy} &
            Message transfers &
            \multicolumn{2}{|c|}{Service transfers} &
            Description \\\hline

            % TRANSFER PAYLOAD
            \multicolumn{3}{|c|}{Transfer payload} &
            \multicolumn{3}{c|}{\bfseries{} Serialized object} &
            The serialized instance of a specific DSDL data type. \\\hline

            % TRANSFER PRIORITY
            \multicolumn{1}{|c|}{\multirow{7}{*}{\rotatebox[origin=c]{90}{Transfer metadata}}} &
            &
            &
            \multicolumn{3}{c|}{\bfseries{} Transfer priority} &
            Defines the urgency (time sensitivity) of the transferred object.\\\cline{4-7}

            % TRANSFER ID
            \multicolumn{1}{|c|}{} &
            &
            &
            \multicolumn{3}{c|}{\bfseries{} Transfer-ID} &
            An integer that uniquely identifies a transfer within its category.\\\cline{2-7}

            % ROUTE SPECIFIER
            \multicolumn{1}{|c|}{} &
            \multicolumn{1}{c|}{\multirow{5}{*}{\rotatebox[origin=c]{90}{\shortstack{Session \\ specifier}}}} &
            \multirow{2}{*}{\shortstack{Route \\ specifier}} &
            \multicolumn{3}{c|}{\bfseries{} Source node-ID} &
            Source node-ID is not defined for anonymous message transfers. \\\cline{5-6}

            \multicolumn{1}{|c|}{} &
            \multicolumn{1}{c|}{} &
            &
            \multicolumn{1}{c|}{} &
            \multicolumn{2}{c|}{\bfseries{} Destination node-ID} &
            Destination node-ID is not defined for message transfers.\\\cline{3-7}

            % DATA SPECIFIER
            \multicolumn{1}{|c|}{} &
            \multicolumn{1}{c|}{} &
            \multirow{3}{*}{\shortstack{Data \\ specifier}} &
            \multicolumn{1}{c|}{\multirow{2}{*}{\bfseries{} Subject-ID}} &
            \multicolumn{2}{c|}{\bfseries{} Service-ID} &
            Port-ID specifies how the serialized object should be processed.\\\cline{5-6}

            \multicolumn{1}{|c|}{} &
            \multicolumn{1}{c|}{} &
            &
            \multicolumn{1}{c|}{} &
            {\bfseries{} Request} &
            {\bfseries{} Response} &
            Request/response specifier applies to services only.\\\cline{4-6}

            \multicolumn{1}{|c|}{} &
            \multicolumn{1}{c|}{} &
            &
            \multicolumn{3}{c|}{\bfseries{} Transfer kind} &
            Message (subject) or service transfer.\\\hline
        \end{tabu}
    }
    \caption{UAVCAN transport layer model.}\label{fig:transport_layer_model}
\end{figure}

\subsubsection{Transfer}

A \emph{transfer} is a singular act of data transmission from one UAVCAN node to zero or more other UAVCAN nodes
over the transport network.
A transfer carries zero or more bytes of \emph{transfer payload} together with the associated \emph{transfer metadata},
which encodes the semantic and temporal properties of the carried payload.
The elements comprising the metadata are reviewed below.

Transfers are distinguished between \emph{message transfers} and \emph{service transfers} depending on the kind
of the carried DSDL object.

A transfer is manifested on the transport network as one or more \emph{transport frames}.
A transport frame is an atomic entity carrying the entire transfer payload or a fraction thereof
with the associated transfer metadata --
possibly extended with additional elements specific to the concrete transport layer implementation --
over the transport network.
The exact definition of a transport frame and the mapping of the abstract transport model onto it
are dependent on the transport layer implementation\footnote{
    For example, the CAN bus transport layer implementation (introduced later) defines a particular CAN frame format.
    Frames that follow the format are UAVCAN transport frames of the CAN bus transport.
}.

\subsubsection{Transfer payload}

The transfer payload contains the serialized representation of the carried
DSDL object\footnote{Chapter \ref{sec:dsdl}.}.

Concrete transport layer implementations may extend the payload with a \emph{transfer CRC}
-- an additional metadata field used for validating its integrity.
The details of its implementation are dictated by the concrete transport layer specification.

Concrete transport layer implementations may extend the payload with zero-valued \emph{padding bytes} at the end
to meet the transport-specific data granularity constraints.
Usage of non-zero-valued padding bytes is prohibited for all implementations\footnote{%
    Non-zero padding bytes are disallowed because they would interfere with the implicit zero extension rule
    (section \ref{sec:dsdl_data_serialization}).
}.

The deterministic nature of UAVCAN in general and DSDL in particular allows implementations to statically determine
the maximum amount of memory that is required to contain the serialized representation
of a DSDL object of a particular type.
Consequently, an implementation that is interested in receiving data objects of a particular type
can statically determine the maximum length of the transfer payload.

Implementations shall handle incoming transfers containing a larger amount of payload data than expected.
In the event of such extra payload being received, a compliant implementation should
discard the excessive (unexpected) data at the end of the received payload\footnote{%
    Such occurrence is not indicative of a problem so it should not be reported as such.
}.
The transfer CRC, if applicable, shall be validated regardless of the presence of the extra payload in the transfer.

A \emph{transport-layer maximum transmission unit} (MTU) is the maximum amount of data with the associated metadata
that can be transmitted per transport frame for a particular transport layer implementation.
All nodes connected to a given transport network should share the same transport-layer MTU setting\footnote{%
    Failure to follow this rule may render nodes unable to communicate if a transmitting node emits larger transport
    frames than the receiving node is able to accept.
}.

In order to facilitate the implicit zero extension rule introduced in section \ref{sec:dsdl_data_serialization},
implementations shall not discard a transfer even if it is determined that it contains less payload
data than a predicted minimum.

\begin{figure}[H]
    $$
    \raisebox{1em}{\footnotesize{\text{first byte}}}
    \overbrace{%
        \underbrace{%
            \blacksquare\blacksquare\blacksquare\blacksquare\blacksquare\blacksquare%
            \blacksquare\blacksquare\blacksquare\blacksquare\blacksquare\blacksquare%
        }_{\substack{\text{Expected, accepted} \\ \text{payload}}}%
        \underbrace{%
            \boxtimes\boxtimes\boxtimes\boxtimes\boxtimes\boxtimes\boxtimes\boxtimes%
        }_{\substack{\text{Excessive, discarded} \\ \text{payload}}}%
    }^{\substack{%
        \text{Transfer CRC is validated} \\
        \text{for the entire transfer payload} \\
        \text{before the truncation}}
    }
    \raisebox{1em}{\footnotesize{\text{last byte}}}
    $$
    \caption{Transfer payload truncation.\label{fig:transport_payload_truncation}}
\end{figure}

\begin{remark}
    The requirement to discard the excessive payload data at the end of the transfer is motivated by
    the necessity to allow extensibility of data type definitions, as described in chapter \ref{sec:dsdl}.
    Additionally, excessive payload data may contain zero padding bytes if required by the particular
    transport layer implementation.

    Let node $A$ publish an object of the following type over the subject $x$:

    \begin{minted}{python}
        float32 parameter
        float32 variance
    \end{minted}

    Let node $B$ subscribe to the subject $x$ expecting an object of the following type:

    \begin{minted}{python}
        float32 parameter
    \end{minted}

    The payload truncation requirement guarantees that the two nodes will be able to interoperate despite
    relying on incompatible data type definitions.
    Under this example, the duty of ensuring the semantic compatibility lies on the system integrator.

    The requirement that all involved nodes use the same transport-layer MTU is crucial here.
    Suppose that the MTU expected by the node $B$ is four bytes and the MTU of the node $A$ is eight bytes.
    Under this setup, messages emitted by $A$ would be contained in single-frame transfers that are too large
    for $B$ to process, resulting in the nodes being unable to communicate.
    An attempt to optimize the memory utilization of $B$ by relying on the fact that the maximum length of a
    serialized representation of the message is four bytes would be a mistake, because this assumption ignores
    the existence of subtyping and introduces leaky abstractions throughout the protocol stack.
\end{remark}

\begin{remark}
    The implicit zero extension rule makes deserialization routines sensitive to the trailing unused data.
    For example, suppose that a publisher emits an object of type:

    \begin{minted}{python}
        uint16 foo
    \end{minted}

    Suppose that the transport layer at hand requires padding to 4 bytes, which is done with $55_{16}$
    (intentionally non-compliant for the sake of this example).
    Suppose that the published value is $1234_{16}$,
    so the resulting serialized representation is $\left[34_{16}, 12_{16}, 55_{16}, 55_{16}\right]$.
    Suppose that the receiving side relies on the implicit zero extension rule with the following definition:

    \begin{minted}{python}
        uint16 foo
        uint16 bar
    \end{minted}

    The expectation is that \verb|foo| will be deserialized as $1234_{16}$,
    and \verb|bar| will be zero-extended as $0000_{16}$.
    If arbitrary padding values were allowed, the value of \verb|bar| would become undefined;
    in this particular example it would be $5555_{16}$.

    Therefore, the implicit zero-extension rule requires that padding is done with zero bytes only.
\end{remark}

\subsubsection{Transfer priority}\label{sec:transport_transfer_priority}

Transfers are prioritized by means of the \emph{transfer priority} property,
which allows at least 8 (eight) distinct priority levels.
Concrete transport implementations may support more than eight priority levels.

Transmission of transport frames shall be ordered so that frames of higher priority are transmitted first.
It follows that higher-priority transfers may preempt transmission of lower-priority transfers.

Transmission of transport frames that share the same priority level should follow the order of their appearance in
the transmission queue.

\begin{remark}[breakable]
    Transfer prioritization is paramount for distributed real-time applications.

    The priority level mnemonics and their usage recommendations are specified in the following list.
    The mapping between the mnemonics and actual numeric identifiers is transport-dependent.

    % https://forum.uavcan.org/t/transfer-priority-level-mnemonics/218/6?u=pavel.kirienko
    \begin{description}
        \item[Exceptional] -- The bus designer can ignore these messages when calculating bus load since they
        should only be sent when a total system failure has occurred.
        For example, a self-destruct message on a rocket would use this priority.
        Another analogy is an NMI on a microcontroller.

        \item[Immediate] -- Immediate is a ``high priority message'' but with additional latency constraints.
        Since exceptional messages are not considered when designing a bus, the latency of immediate messages
        can be determined by considering only immediate messages.

        \item[Fast] -- Fast and immediate are both ``high priority messages'' but with additional latency constraints.
        Since exceptional messages are not considered when designing a bus,
        the latency of fast messages can be determined by considering only immediate and fast messages.

        \item[High] -- High priority messages are more important than nominal messages but have looser
        latency requirements than fast messages. This priority is used so that,
        in the presence of rogue nominal messages, important commands can be received.
        For example, one might envision a failure mode where a temperature sensor starts to
        load a vehicle bus with nominal messages.
        The vehicle remains operational (for a time) because the controller is exchanging fast and
        immediate messages with sensors and actuators.
        A system safety monitor is able to detect the distressed bus and command the vehicle to a
        safe state by sending high priority messages to the controller.

        \item[Nominal] -- This is what all messages should use by default.
        Specifically the heartbeat messages should use this priority.

        \item[Low] -- Low priority messages are expected to be sent on a bus under all conditions but cannot
        prevent the delivery of nominal messages.
        They are allowed to be delayed but latency should be constrained by the bus designer.

        \item[Slow] -- Slow messages are low priority messages that have no time sensitivity at all.
        The bus designer need only ensure that, for all possible system states,
        these messages will eventually be sent.

        \item[Optional] -- These messages might never be sent (theoretically) for some possible system states.
        The system must tolerate never exchanging optional messages in every possible state.
        The bus designer can ignore these messages when calculating bus load.
        This should be the priority used for diagnostic or debug messages that are not required on an
        operational system.
    \end{description}
\end{remark}

\subsubsection{Transfer-ID}\label{sec:transport_transfer_id}

Cyclic vs. monotonic.

\subsubsection{Route specifier}

\subsubsection{Data specifier}

\subsubsection{Session specifier}

UAVCAN is stateless. Sessions are purely local constructs.

\subsection{Transfers}

\subsubsection{Single-frame and multi-frame transfers}


