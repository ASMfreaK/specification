\section{Application-level requirements}\label{sec:application_requirements}

This section describes a set of high-level rules that shall be obeyed by all UAVCAN implementations.

\subsection{Port identifier distribution}

An overview of related concepts is provided in chapter~\ref{sec:basic}.

The subject and service identifier values are segregated into three ranges:
\begin{itemize}
    \item unregulated port identifiers that can be freely chosen by users and integrators (both fixed and non-fixed);
    \item regulated fixed identifiers for non-standard data type definitions
          that are assigned by the UAVCAN maintainers for publicly released data types;
    \item regulated identifiers of the standard data types that are an integral part of the UAVCAN specification.
\end{itemize}

More information on the subject of data type regulation is provided in section~\ref{sec:basic_data_type_regulation}.

The ranges are summarized in table~\ref{table:application_requirements_port_id_distribution}.
The ranges may be expanded, but not contracted, in a future version of the document.

\begin{UAVCANSimpleTable}{Port identifier distribution}{|l l X|}%
    \label{table:application_requirements_port_id_distribution}
    Subject-ID      & Service-ID        & Purpose \\
    $[0, 6143]$     & $[0, 255]$        & Unregulated identifiers (both fixed and non-fixed). \\
    $[6144, 7167]$  & $[256, 383]$      & Non-standard fixed regulated identifiers (i.e., vendor-specific). \\
    $[7168, 8191]$  & $[384, 511]$      & Standard fixed regulated identifiers. \\
\end{UAVCANSimpleTable}

\subsection{Port compatibility}

% https://github.com/UAVCAN/specification/pull/64#discussion_r357771739

The system integrator shall ensure that nodes participating in data exchange via a given port\footnote{%
    I.e., subject or service.
} use data type definitions that are sufficiently congruent so that the resulting behavior of the involved nodes
is predictable and the possibility of unintended behaviors caused by misinterpretation of exchanged serialized
objects is eliminated.

\begin{remark}
    Let there be type $A$:

    \begin{minted}{python}
        void1
        uint7 demand_factor_pct  # [percent]
        # Values above 100% are not allowed.
    \end{minted}

    And type $B$:

    \begin{minted}{python}
        uint8 demand_factor_pct  # [percent]
        # Values above 100% indicate overload.
    \end{minted}

    The data types are not semantically compatible, but they can be used on the same subject nevertheless:
    a subscriber expecting $B$ can accept $A$.
    The reverse is not true, however.

    This example shows that even semantically incompatible types can facilitate
    behaviorally correct interoperability of nodes.
\end{remark}

\begin{remark}
    Compatibility of subjects and services is completely independent from the names of the involved data types.
    Data types can be moved between namespaces and freely renamed and re-versioned
    without breaking compatibility with existing deployments.
    Nodes provided by different vendors that utilize differently named data types may
    still interoperate if such data types happen to be compatible.
    The duty of ensuring the compatibility lies on the system integrator.
\end{remark}

\subsection{Standard namespace}

An overview of related concepts is provided in chapter~\ref{sec:dsdl}.

This specification defines a set of standard regulated DSDL data types located under
the root namespace named ``\verb"uavcan"'' (section~\ref{sec:sdt}).

Vendor-specific, user-specific, or any other data types not defined by this specification
shall not be defined inside the standard root namespace\footnote{Custom data type definitions shall be located
inside vendor-specific or user-specific namespaces instead.}.
