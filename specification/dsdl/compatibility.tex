\section{Compatibility and versioning}\label{sec:dsdl_versioning}

\subsection{Rationale}

Data type definitions may evolve over time as they are refined to better address the needs of their applications.
UAVCAN defines a set of rules that allow data type designers to modify and advance their
data type definitions while ensuring backward compatibility and functional safety.

\subsection{Semantic compatibility}\label{sec:dsdl_semantic_compatibility}

A data type $A$ is \emph{semantically compatible} with a data type $B$
if relevant behavioral properties of the application are invariant under the substitution of $A$ with $B$.
The property of semantic compatibility is commutative.

\begin{remark}[breakable]
    The following two definitions are semantically compatible and can be used interchangeably:

    \begin{minted}{python}
        uint16 FLAG_A = 1
        uint16 FLAG_B = 256
        uint16 flags
    \end{minted}

    \begin{minted}{python}
        uint8 FLAG_A = 1
        uint8 FLAG_B = 1
        uint8 flags_a
        uint8 flags_b
    \end{minted}

    It should be noted here that due to different set of field and constant attributes,
    the source code auto-generated from the provided definitions may be not drop-in replaceable,
    requiring changes in the application;
    however, source-code-level application compatibility is orthogonal to data type compatibility.

    The following supertype may or may not be semantically compatible with the above
    depending on the semantics of the removed field:

    \begin{minted}{python}
        uint8 FLAG_A = 1
        uint8 flags_a
    \end{minted}
\end{remark}

\begin{remark}
    Let node $A$ publish messages of the following type:

    \begin{minted}{python}
        float32 foo
        float64 bar
    \end{minted}

    Let node $B$ subscribe to the same subject using the following data type definition:

    \begin{minted}{python}
        float32 foo
        float64 bar
        int16   baz  # Extra field; implicit zero extension rule applies.
    \end{minted}

    Let node $C$ subscribe to the same subject using the following data type definition:

    \begin{minted}{python}
        float32 foo
        # The field 'bar' is missing; implicit truncation rule applies.
    \end{minted}

    Provided that the semantics of the added and omitted fields allow it,
    the nodes will be able to interoperate successfully despite using different data type definitions.
\end{remark}

\subsection{Versioning}

\subsubsection{General assumptions}

The concept of versioning applies only to composite data types.
As such, unless specifically stated otherwise, every reference to ``data type''
in this section implies a composite data type.

A data type is uniquely identified by its full name,
assuming that every root namespace is uniquely named.
There is one or more versions of every data type.

A data type definition is uniquely identified by its full name and the version number pair.
In other words, there may be multiple definitions of a data type differentiated by their version numbers.

\subsubsection{Versioning principles}

Every data type definition has a pair of version numbers ---
a major version number and a minor version number, following the principles of semantic versioning.

For the purposes of the following definitions, a \emph{release} of a data type definition stands for
the disclosure of the data type definition to its intended users or to the general public,
or for the commencement of usage of the data type definition in a production system.

In order to ensure a deterministic application behavior and ensure a robust migration path
as data type definitions evolve, all data type definitions that share the same
full name and the same major version number shall be semantically compatible with each other.

The versioning rules do not extend to scenarios where the name of a data type is changed,
because that would essentially construe the release of a new data type,
which relieves its designer from all compatibility requirements.
When a new data type is first released,
the version numbers of its first definition shall be assigned ``1.0'' (major 1, minor 0).

In order to ensure predictability and functional safety of applications that leverage UAVCAN,
it is required that once a data type definition is released,
its DSDL source text, name, version numbers, fixed port-ID, extent, finality, and other properties cannot undergo any
modifications whatsoever, with the following exceptions:
\begin{itemize}
    \item Whitespace changes of the DSDL source text are allowed,
          excepting string literals and other semantically sensitive contexts.

    \item Comment changes of the DSDL source text are allowed as long as such changes
          do not affect semantic compatibility of the definition.

    \item A deprecation marker directive (section~\ref{sec:dsdl_directives}) can be added or removed\footnote{%
              Removal is useful when a decision to deprecate a data type definition is withdrawn.
          }.
\end{itemize}
Addition or removal of the fixed port identifier is not permitted after a data type definition
of a particular version is released.

Therefore, substantial changes can be introduced only by releasing new definitions (i.e., new versions)
of the same data type.
If it is desired and possible to keep the same major version number for a new definition of the data type,
the minor version number of the new definition shall be one greater than the newest existing minor version
number before the new definition is introduced.
Otherwise, the major version number shall be incremented by one and the minor version shall be set to zero.

An exception to the above rules applies when the major version number is zero.
Data type definitions bearing the major version number of zero are not subjected to any compatibility requirements.
Released data type definitions with the major version number of zero are permitted to change in arbitrary
ways without any regard for compatibility.
It is recommended, however, to follow the principles of immutability, releasing every subsequent definition
with the minor version number one greater than the newest existing definition.

For any data type, there shall be at most one definition per version.
In other words, there shall be exactly one or zero definitions
per combination of data type name and version number pair.

All data types under the same name shall be also of the same kind.
In other words, if the first released definition of a data type is of the message kind,
all other versions shall also be of the message kind.

All data types under the same name and major version number should share the same extent and the same finality status.

\subsubsection{Fixed port identifier assignment constraints}

The following constraints apply to fixed port-ID assignments:
\begin{align*}
    \exists P(x_{a.b})                          &\rightarrow \exists P(x_{a.c})
    &\mid&\ b < c;\ x \in (M \cup S)
    \\
    \exists P(x_{a.b})                          &\rightarrow         P(x_{a.b}) =    P(x_{a.c})
    &\mid&\ b < c;\ x \in (M \cup S)
    \\
    \exists P(x_{a.b}) \land \exists P(x_{c.d}) &\rightarrow         P(x_{a.b}) \neq P(x_{c.d})
    &\mid&\ a \neq c;\ x \in (M \cup S)
    \\
    \exists P(x_{a.b}) \land \exists P(y_{c.d}) &\rightarrow         P(x_{a.b}) \neq P(y_{c.d})
    &\mid&\ x \neq y;\ x \in T;\ y \in T;\ T = \left\{ M, S \right\}
\end{align*}
where $t_{a.b}$ denotes a data type $t$ version $a.b$ ($a$ major, $b$ minor);
$P(t)$ denotes the fixed port-ID (whose existence is optional) of data type $t$;
$M$ is the set of message types, and $S$ is the set of service types.

\subsubsection{Data type version selection}

DSDL compilers should compile every available data type version separately,
allowing the application to choose from all available major and minor version combinations.

When emitting a transfer, the major version of the data type is chosen at the discretion of the application.
The minor version should be the newest available one under the chosen major version.

When receiving a transfer, the node deduces which major version of the data type to use
from its port identifier (either fixed or non-fixed).
The minor version should be the newest available one under the deduced major version\footnote{%
    Such liberal minor version selection policy poses no compatibility risks since all definitions under the same
    major version are compatible with each other.
}.

It follows from the above two rules that when a node is responding to a service request,
the major data type version used for the response transfer shall be the same that is used for the request transfer.
The minor versions may differ, which is acceptable due to the major version compatibility requirements.

\begin{remark}[breakable]
    A simple usage example is provided in this intermission.

    Suppose a vendor named ``Sirius Cybernetics Corporation'' is contracted to design a
    cryopod management data bus for a colonial spaceship ``Golgafrincham B-Ark''.
    Having consulted with applicable specifications and standards, an engineer came up with the following
    definition of a cryopod status message type (named \verb|sirius_cyber_corp.b_ark.cryopod.Status|):

    \begin{minted}{python}
        # sirius_cyber_corp.b_ark.cryopod.Status.0.1

        float16 internal_temperature    # [kelvin]
        float16 coolant_temperature     # [kelvin]

        uint8 FLAG_COOLING_SYSTEM_A_ACTIVE = 1
        uint8 FLAG_COOLING_SYSTEM_B_ACTIVE = 2
        # Status flags in the lower bits.
        uint8 FLAG_PSU_MALFUNCTION = 32
        uint8 FLAG_OVERHEATING     = 64
        uint8 FLAG_CRYOBOX_BREACH  = 128
        # Error flags in the higher bits.
        uint8 flags  # Storage for the above defined flags (this is not the recommended practice).
    \end{minted}

    The definition is then deployed to the first prototype for initial laboratory testing.
    Since the definition is experimental, the major version number is set to zero in order to signify the
    tentative nature of the definition.
    Suppose that upon completion of the first trials it is identified that the units should track their
    power consumption in real time for each of the three redundant power supplies independently.

    It is easy to see that the amended definition shown below is not semantically compatible
    with the original definition; however, it shares the same major version number of zero, because the backward
    compatibility rules do not apply to zero-versioned data types to allow for low-overhead experimentation
    before the system is deployed and fielded.

    \begin{minted}{python}
        # sirius_cyber_corp.b_ark.cryopod.Status.0.2

        truncated float16 internal_temperature    # [kelvin]
        truncated float16 coolant_temperature     # [kelvin]

        saturated float32 power_consumption_0     # [watt] Power consumption by the redundant PSU 0
        saturated float32 power_consumption_1     # [watt] likewise for PSU 1
        saturated float32 power_consumption_2     # [watt] likewise for PSU 2
        # breaking compatibility with Status.0.1 is okay because the major version is 0

        uint8 FLAG_COOLING_SYSTEM_A_ACTIVE = 1
        uint8 FLAG_COOLING_SYSTEM_B_ACTIVE = 2
        # Status flags in the lower bits.
        uint8 FLAG_PSU_MALFUNCTION = 32
        uint8 FLAG_OVERHEATING     = 64
        uint8 FLAG_CRYOBOX_BREACH  = 128
        # Error flags in the higher bits.
        uint8 flags  # Storage for the above defined flags (this is not the recommended practice).
    \end{minted}

    The last definition is deemed sufficient and is deployed to the production system
    under the version number of 1.0: \verb|sirius_cyber_corp.b_ark.cryopod.Status.1.0|.

    Having collected empirical data from the fielded systems, the Sirius Cybernetics Corporation has
    identified a shortcoming in the v1.0 definition, which is corrected in an updated definition.
    Since the updated definition, which is shown below, is semantically compatible\footnote{%
        The topic of data serialization is explored in detail in section~\ref{sec:dsdl_data_serialization}.
    } with v1.0, the major version number is kept the same and the minor version number is incremented by one:

    \begin{minted}{python}
        # sirius_cyber_corp.b_ark.cryopod.Status.1.1

        saturated float16 internal_temperature    # [kelvin]
        saturated float16 coolant_temperature     # [kelvin]

        float32[3] power_consumption    # [watt] Power consumption by the PSU

        bool flag_cooling_system_a_active
        bool flag_cooling_system_b_active
        # Status flags (this is the recommended practice).

        void3   # Reserved for other flags

        bool flag_psu_malfunction
        bool flag_overheating
        bool flag_cryobox_breach
        # Error flags (this is the recommended practice).
    \end{minted}

    Since the definitions v1.0 and v1.1 are semantically compatible,
    UAVCAN nodes using either of them can successfully interoperate on the same bus.

    Suppose further that at some point a newer version of the cryopod module,
    equipped with better temperature sensors, is released.
    The definition is updated accordingly to use \verb|float32| for the temperature fields instead of \verb|float16|.
    Seeing as that change breaks the compatibility, the major version number has to be incremented by one,
    and the minor version number has to be reset back to zero:

    \begin{minted}{python}
        # sirius_cyber_corp.b_ark.cryopod.Status.2.0

        float32 internal_temperature    # [kelvin]
        float32 coolant_temperature     # [kelvin]

        float32[3] power_consumption    # [watt] Power consumption by the PSU

        bool flag_cooling_system_a_active
        bool flag_cooling_system_b_active
        void3
        bool flag_psu_malfunction
        bool flag_overheating
        bool flag_cryobox_breach
    \end{minted}

    Imagine that later it was determined that the module should report additional status information
    relating to the coolant pump.
    Thanks to the implicit truncation and the implicit zero extension rules,
    the new fields can be introduced in a semantically-compatible way without releasing
    a new major version of the data type:

    \begin{minted}{python}
        # sirius_cyber_corp.b_ark.cryopod.Status.2.1

        float32 internal_temperature    # [kelvin]
        float32 coolant_temperature     # [kelvin]

        float32[3] power_consumption    # [watt] Power consumption by the PSU

        bool flag_cooling_system_a_active
        bool flag_cooling_system_b_active
        void3
        bool flag_psu_malfunction
        bool flag_overheating
        bool flag_cryobox_breach

        float32 rotor_angular_velocity  # [radian/second] (usage of RPM would be non-compliant)
        float32 volumetric_flow_rate    # [meter^3/second]
        # Coolant pump fields (extension over v2.0; implicit truncation/extension rules apply)
        # If zero, assume that the values are unavailable.
    \end{minted}

    It is also possible to add an optional field at the end wrapped into a variable-length
    array of up to one element, or a tagged union where the first field is empty
    and the second field is the wrapped value.
    In this way, the implicit truncation/extension rules would automatically make such optional field
    appear/disappear depending on whether it is supported by the receiving node.

    Nodes using v1.0, v1.1, v2.0, and v2.1 definitions can coexist on the same network,
    and they can interoperate successfully as long as they all support at least v1.x or v2.x.
    The correct version can be determined at runtime from the port identifier assignment as described in
    section~\ref{sec:basic_subjects_and_services}.

    In general, nodes that need to maximize their compatibility are likely to employ all existing major versions of
    each used data type.
    If there are more than one minor versions available, the highest minor version within the major version should
    be used in order to take advantage of the latest changes in the data type definition.
    It is also expected that in certain scenarios some nodes may resort to publishing the same message type
    using different major versions concurrently to circumvent compatibility issues
    (in the example reviewed here that would be v1.1 and v2.1).

    The examples shown above rely on the primitive scalar types for reasons of simplicity.
    Real applications should use the type-safe physical unit definitions available in the SI namespace instead.
    This is covered in section~\ref{sec:application_functions_si}.
\end{remark}
